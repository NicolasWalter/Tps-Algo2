% Diseño del tipo T
\newpage

% Diseño del Tipo
\disDisenio{Iterador sobre Lista Extendido($\alpha)$}
% La especificación

\disAspectosDeLaInterfaz

\disInterfaz
Se extiende la interfaz del Iterador sobre Lista dada en el apunte de m\'odulos b\'asicos, el cual recorrer\'a, una lista de tuplas, por lo que las operaciones Siguiente y Anterior, devuelven el primer elemento.

\disOperaciones{b\'asicas del Iterador Extendido}

\disDeclaraFuncion{CrearIt}{\paramIn{l}{lista(\alpha)}}{res: itListaE(\alpha)}{true}{alias(res \igobs crearItBi(<>, l) $\wedge$ alias(SecuSuby(it) $=$ l)}{\Ode{1}}{Crea un iterador bidireccional de la lista, de forma tal que al pedir Siguiente se obtenga el primer elemento de l..}
\disComentAliasing{el iterador se invalida si y s\'olo si se elimina el elemento siguiente del iterador sin utilizar la funci\'on EliminarSiguiente.}

\disDeclaraFuncion{Siguiente}{\paramIn{it}{itListaE(\alpha)}}{res: \alpha}{HaySiguiente?(it)}{alias(res \igobs $\Pi_{1}$(Siguiente(it)))}{\Ode{1}}{Devuelve el elemento siguiente a la posici\'on del iterador.}
\disComentAliasing{Res es modificable si y s\'olo si it es modificable.}

\disDeclaraFuncion{Anterior}{\paramIn{it}{itListaE(\alpha)}}{res: \alpha}{HayAnterior?(it)}{alias(res \igobs $\Pi_{1}$(Anterior(it)))}{\Ode{1}}{Devuelve el elemento siguiente a la posici\'on del iterador.}
\disComentAliasing{Res es modificable si y s\'olo si it es modificable.}


\disAlgoritmos
%\hspace*{\disSubSubSecMargen}{Texto}
% HACK: SGA 28/05/2011. Para quitar el titulo Algorithm del caption \floatname{algorithm}{}
\floatname{algorithm}{}
% WARNING: SGA 27/05/2011. La opción [H] indica a LaTex que el algoritmo lo queremos AQUI!
% Ver 4.4.1 Placement of Algorithms de algorithms.pdf.



\begin{algorithm}[H]
\begin{algorithmic}[1]
\Function {\textsc{$i$CrearItE}}{\paramIn{l}{lista(\alpha)}}{$\disFlecha$ res: itListaE} \Comment{$\Ode{1}$}
	\State res $\gets$ $<$l.primero, l$>$ \Comment{$\Ode{1}$}
\EndFunction
\end{algorithmic}
\end{algorithm}
	

\begin{algorithm}[H]
\begin{algorithmic}[1]
\Function {\textsc{$i$Siguiente}}{\paramIn{it}{itListaE(\alpha)}}{$\disFlecha$ res: \alpha} \Comment{$\Ode{1}$}
	\State res $\gets$ $\Pi_{1}$(it.siguiente\rightarrow dato) \Comment{$\Ode{1}$}
\EndFunction
\end{algorithmic}
\end{algorithm}
	
	
\begin{algorithm}[H]
\begin{algorithmic}[1]	
\Function {\textsc{$i$Anterior}}{\paramIn{it}{itListaE(\alpha)}}{$\disFlecha$ res: $\alpha$} \Comment{$\Ode{1}$}
	\State res $\gets$ $\Pi_{1}$(SiguienteReal(it)\rightarrow anterior\rightarrow dato)	\Comment{$\Ode{1}$}
\EndFunction
\end{algorithmic}
\end{algorithm}