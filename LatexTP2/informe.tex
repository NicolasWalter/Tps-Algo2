\documentclass[a4paper,spanish, 10pt]{article}
\usepackage[paper=a4paper, left=1.2cm, right=1.2cm, bottom=2cm, top=2cm]{geometry}
\usepackage[spanish]{babel}
\usepackage[latin1]{inputenc}
\usepackage{fancyhdr}
\usepackage{caratula, aed2-symb, aed2-itef, aed2-tad}
\usepackage{algorithm}
\usepackage{algpseudocode}
\usepackage{setspace}
\usepackage{color}
\usepackage{dis}
\usepackage{setspace}

\begin{document}

% 0Acomodo fancyhdr.
\pagestyle{fancy}
\thispagestyle{fancy}
\addtolength{\headheight}{1pt}
\lhead{Algoritmos y Estructuras de Datos II}
\rhead{$2^{\mathrm{do}}$ cuatrimestre de 2015}
\cfoot{\thepage /\pageref{LastPage}}
\renewcommand{\footrulewidth}{0.4pt}

\author{Algoritmos y Estructuras de Datos II, DC, UBA.}
\date{}
\title{}



% Datos de caratula
\materia{Algoritmos y Estructuras de Datos II}
\titulo{Trabajo Pr\'actico N\'umero 2}
%\subtitulo{}
\grupo{Grupo: 21}

\integrante{Langberg, Andr\'es}{249/14}{andreslangberg@gmail.com}
\integrante{Walter, Nicol\'as}{272/14}{nicowalter25@gmail.com}
\integrante{Sticco, Patricio Bernardo}{337/14}{pbsticco@hotmail.com}
\integrante{Len, Juli\'an}{467/14}{julianlen@gmail.com}

\maketitle



 %\section{Observaciones}

 	\begin{enumerate}
 	    \item \textbf{TAD} POSICION \textbf{ES} TUPLA(X:NAT, Y:NAT)
	    \item \textbf{TAD} DIRECCION \textbf{ES} ENUM\{ IZQ,DER,ARRIBA,ABAJO\}
	    \item \textbf{TAD} AGENTE \textbf{ES} NAT
	    \item \textbf{TAD} NOMBRE \textbf{ES} STRING
	    \item Suponemos que contamos con el TAD DiccionarioM, donde la funcion vacio() toma como par\'ametro un 'k', cuyo valor acota superiormente a la cantidad de claves.
	\end{enumerate}

%% Diseño del tipo T
\newpage

% Diseño del Tipo
\disDisenio{Campus}
% La especificación
\disEspecificacion
\hspace*{\disSubSecMargen}Se usa el {\sc Tad Campus} especificado por la c\'atedra.

\disAspectosDeLaInterfaz

\disInterfaz

\disSeExplicaCon{Campus}

\disGenero{campus}

\disOperaciones{b\'asicas de Campus}

\disDeclaraFuncion{CrearCampus}{\paramIn{c}{nat}, \paramIn{f}{nat}}{res : campus}{true}{res $\igobs$ crearCampus(c,f)}{\Ode{1}}{Crea un campus de c columnas y f filas.}

\disDeclaraFuncion{AgregarObstaculo}{\paramInOut{c}{campus}, \paramIn{p}{posicion}}{}{c $\igobs$ $c_{0}$ $\land$ posValida(p,c) $\yluego$ $\¬$ocupada?(p,c)}{c $\igobs$ agregarObstaculo(p,$c_{0}$)}{\Ode{1}}{Devuelve $true$ sii p esta ocupada por un obstaculo.}


\disDeclaraFuncion{Filas?}{\paramIn{c}{campus}}{res : nat}{true}{res $\igobs$ filas(c)}{\Ode{1}}{Devuelve la cantidad de filas en el campus.}

\disDeclaraFuncion{Columnas?}{\paramIn{c}{campus}}{res : nat}{true}{res $\igobs$ columnas(c)}{\Ode{1}}{Devuelve la cantidad de columnas en el campus.}

\disDeclaraFuncion{Ocupada?}{\paramIn{c}{campus}, \paramIn{p}{posicion}}{res : bool}{posValida(p,c)}{res $\igobs$ ocupada?(p,c)}{\Ode{1}}{Devuelve $true$ sii p esta ocupada por un obstaculo.}

\disDeclaraFuncion{PosValida?}{\paramIn{c}{campus}, \paramIn{p}{posicion}}{res : bool}{true}{res $\igobs$ posValida?(p,c)}{\Ode{1}}{Devuelve $true$ sii p es parte del mapa.}

\disDeclaraFuncion{EsIngreso?}{\paramIn{c}{campus}, \paramIn{p}{posicion}}{res : bool}{true}{res $\igobs$ esIngreso?(p,c)}{\Ode{1}}{Devuelve $true$ sii p es un ingreso.}
\newpage
 
\disDeclaraFuncion{Vecinos}{\paramIn{c}{campus}, \paramIn{p}{posicion}}{res : conj(posicion)}{posValida(p,c)}{res $\igobs$ vecinos(p,c)}{\Ode{1}}{Devuelve el conjunto de posiciones vecinas a p.}

\disDeclaraFuncion{ProxPosicion}{\paramIn{c}{campus}, \paramIn{dir}{direccion}, \paramIn{p}{posicion}}{res : posicion}{posValida(p,c)}{res $\igobs$ proxPosicion(p,d,c)}{\Ode{1}}{Devuelve la posicion vecina a p que esta en la direccion dir.}

\disDeclaraFuncion{IngresosMasCercanos}{\paramIn{c}{campus}, \paramIn{p}{posicion}}{res : conj(posicion)}{posValida(p,c)}{res $\igobs$ ingresosMasCercanos(p,c)}{\Ode{1}}{Devuelve el conjunto de ingresos mas cercanos a p.}




\disPautasDeImplementacion

\disEstructuraDeRepresentacion

\disSeRepresentaCon{campus}{estr}
\disDondeEs{estr}{\disTuplaEstr{nat/filas, nat/columnas, vector(vector(bool))/mapa}}

\disJustificacionDeLaEstructuraElegida

\disInvarianteDeRepresentacion
\hspace*{\disSubSubSecMargen}\textbf{\textsf{Informal}}

\hspace*{\disSubSubSecMargen}
\begin{enumerate}
\setlength{\itemindent}{3em}
  \item El mapa debe tener tantas filas como indica la estructura, lo mismo con las columnas.

\end{enumerate}

\hspace*{\disSubSubSecMargen}\textbf{\textsf{Formal}}
\onehalfspace
\disRep{estr}{e}{$true$ $\Longleftrightarrow$ 
\\\hspace*{3.75em}(1) e.filas = longitud(e.mapa) $\yluego$ 
($\forall$ i : nat)(i $\leq$ e.filas $\implies$ longitud(e.mapa[i])= e.columnas)}
\disFuncionDeAbstraccion
\vspace*{-1em}
%\hspace*{\disSubSubSecMargen}{Texto}
\disAbs{estr}{e}{campus}{c}{\Big(filas(c) = e.filas $\land$ columnas(c) = e.columnas $\yluego$ ($\forall$ p : posicion)(p.X $\leq$ e.filas $\land$
\\\hspace*{3.75em} p.Y $\leq$ e.columnas $\impluego$ ocupada?(p,c) $\Leftrightarrow$ (e.mapa[f])[c]\Big) }


%\disFuncionDeAbsFuncionesAux

\newpage

\disAlgoritmos
%\hspace*{\disSubSubSecMargen}{Texto}
% HACK: SGA 28/05/2011. Para quitar el titulo Algorithm del caption \floatname{algorithm}{}
\floatname{algorithm}{}
% WARNING: SGA 27/05/2011. La opción [H] indica a LaTex que el algoritmo lo queremos AQUI!
% Ver 4.4.1 Placement of Algorithms de algorithms.pdf.
\begin{algorithm}\phantom{[H]}
\begin{algorithmic}[1]
\Function {\textsc{$i$CrearCampus}}{\paramIn{c}{nat}, \paramIn{f}{nat}}{$\disFlecha$ res : estr} \Comment{$\Ode{f^{2}*c^{2}}$}
  \State var vector(vector(bool)) mapa $\gets$ vacia(vacia()) \Comment{$\Ode{1}$}
  \State var nat i $\gets$ 0 \Comment{$\Ode{1}$}
  \While {i$\leq$f} \Comment{$\Ode{f}$}
    \State var vector(bool) nuevo $\gets$ vacia() \Comment{$\Ode{1}$}
    \State var nat j $\gets$ 0 \Comment{$\Ode{1}$}
    \While {j$\leq$c} \Comment{$\Ode{c}$}
      \State AgregarAtras(nuevo, false) \Comment{$\Ode{c}$}
      \State j++  \Comment{$\Ode{1}$}
    \EndWhile
    \State AgregarAtras(mapa, nuevo) \Comment{$\Ode{f}$}
    \State i++ \Comment{$\Ode{1}$}
  \EndWhile
  \State res $\gets$ $<$ f, c, mapa$>$ \Comment{$\Ode{1}$}
  
\EndFunction
\end{algorithmic}
\end{algorithm}

\begin{algorithm}\phantom{[H]}
\begin{algorithmic}[1]
\Function {\textsc{$i$AgregarObstaculo}}{\paramInOut{e}{estr}, \paramIn{p}{posicion}}{$\disFlecha$ res : estr} \Comment{$\Ode{longitud(e.mapa[p.X]}$}
  \State Agregar(e.mapa[p.X], p.Y, true) \Comment{$\Ode{longitud(e.mapa[p.X])}$}
\EndFunction
\end{algorithmic}
\end{algorithm}

\begin{algorithm}\phantom{[H]}
\begin{algorithmic}[1]
\Function {\textsc{$i$Filas?}}{\paramIn{e}{estr}}{$\disFlecha$ res : nat} \Comment{$\Ode{1}$}
  \State res $\gets$ e.filas  \Comment{$\Ode{1}$}
\EndFunction
\end{algorithmic}
\end{algorithm}

\begin{algorithm}\phantom{[H]}
\begin{algorithmic}[1]
\Function {\textsc{$i$Columnas?}}{\paramIn{e}{estr}}{$\disFlecha$ res : nat} \Comment{$\Ode{1}$}
  \State res $\gets$ e.columnas  \Comment{$\Ode{1}$}
\EndFunction
\end{algorithmic}
\end{algorithm}

\begin{algorithm}\phantom{[H]}
\begin{algorithmic}[1]
\Function {\textsc{$i$Ocupada?}}{\paramIn{e}{estr}, \paramIn{p}{posicion}}{$\disFlecha$ res : bool} \Comment{$\Ode{1}$}
  \State res $\gets$ (e.mapa[p.X])[p.Y]  \Comment{$\Ode{1}$}
\EndFunction
\end{algorithmic}
\end{algorithm}

\begin{algorithm}\phantom{[H]}
\begin{algorithmic}[1]
\Function {\textsc{$i$PosValida?}}{\paramIn{e}{estr}, \paramIn{p}{posicion}}{$\disFlecha$ res : bool} \Comment{$\Ode{1}$}
  \State res $\gets$ (0 $<$ p.X) $\land$ (p.X $\leq$ e.filas) $\land$ (0 $<$ p.Y) $\land$ (p.Y $\leq$ e.columnas) \Comment{$\Ode{1}$}
\EndFunction
\end{algorithmic}
\end{algorithm}

\begin{algorithm}\phantom{[H]}
\begin{algorithmic}[1]
\Function {\textsc{$i$EsIngreso?}}{\paramIn{e}{estr}, \paramIn{p}{posicion}}{$\disFlecha$ res : bool} \Comment{$\Ode{1}$}
  \State res $\gets$ (p.Y = 1) $\lor$ (p.Y = e.filas) \Comment{$\Ode{1}$}
\EndFunction
\end{algorithmic}
\end{algorithm}

\begin{algorithm}\phantom{[H]}
\begin{algorithmic}[1]
\Function {\textsc{$i$Vecinos}}{\paramIn{e}{estr}, \paramIn{p}{posicion}}{$\disFlecha$ res : bool} \Comment{$\Ode{1}$}
  \State var conj(posicion) nuevo $\gets$ vacio() \Comment{$\Ode{1}$}
  \State Agregar(nuevo, (p.X+1,p.Y))
  \State Agregar(nuevo, (p.X-1,p.Y))
  \State Agregar(nuevo, (p.X,p.Y+1))
  \State Agregar(nuevo, (p.X,p.Y-1))
  \State var $itConj(posicion)$ it $\gets$ crearIt(nuevo)
    \While {haySiguiente(it)} \Comment{$\Ode{c}$}
      \If {$i$PosValida?(e,siguiente(it))} \Comment{$\Ode{1}$}
        \State avanzar(it) \Comment{$\Ode{1}$}
      \Else
      \State eliminarSiguiente(it)  \Comment{$\Ode{1}$}
      \EndIf
    \EndWhile
\EndFunction
\end{algorithmic}
\end{algorithm}



\end{algorithm}

























%% Diseño del tipo T
\newpage

% Diseño del Tipo
\disDisenio{Rastrillaje}
% La especificación
\disEspecificacion
\hspace*{\disSubSecMargen}Se usa el {\sc Tad CampusSeguro} especificado por la c\'atedra.

\disAspectosDeLaInterfaz

\disInterfaz

\disSeExplicaCon{CampusSeguro}

\disGenero{rastr}

\disOperaciones{b\'asicas de Rastrillaje}

\disDeclaraFuncion{Campus}{\paramIn{r}{rastr}}{res : campus}{true}{res $\igobs$ campus(r)}{\Ode{1}}{Devuelve el campus.}

\disDeclaraFuncion{Estudiantes}{\paramIn{r}{rastr}}{res : conj(nombre)}{true}{res $\igobs$ estudiantes(r)}{\Ode{1}}{Devuelve el conjunto de estudiantes presentes en el campus.}

\disDeclaraFuncion{Hippies}{\paramIn{r}{rastr}}{res : conj(nombre)}{true}{res $\igobs$ hippies(r)}{\Ode{1}}{Devuelve el conjunto de hippies presentes en el campus.}

\disDeclaraFuncion{Agentes}{\paramIn{r}{rastr}}{res : conj(agente)}{true}{res $\igobs$ agentes(r)}{\Ode{1}}{Devuelve el conjunto de agentes presentes en el campus.}

\disDeclaraFuncion{PosEstudianteYHippie}{\paramIn{r}{rastr}, \paramIn{id}{nombre}}{res : posicion}{id $\in$ (estudiantes(r) $\cup$ hippies(cs))}{res $\igobs$ posEstudianteYHippie(id,r)}{\Ode{1}}{Devuelve la posici\'on del estudiante/hippie pasado como par\'ametro.}

\disDeclaraFuncion{PosAgente}{\paramIn{r}{rastr}, \paramIn{a}{agente}}{res : posicion}{a $\in$ posAgente(a,r)}{res $\igobs$ posAgente(a,r)}{\Ode{1}}{Devuelve la posici\'on del agente pasado como par\'ametro.}

\disDeclaraFuncion{CantSanciones}{\paramIn{r}{rastr}, \paramIn{a}{agente}}{res : nat}{a $\in$ cantSanciones(a,r)}{res $\igobs$ cantSanciones(a,r)}{\Ode{1}}{Devuelve la cantidad de sanciones recibidas por el agente pasado como par\'ametro.}

\disDeclaraFuncion{CantHippiesAtrapados}{\paramIn{r}{rastr}, \paramIn{a}{agente}}{res : nat}{a $\in$ agentes(r)}{res $\igobs$ cantHippiesAtrapados(a,r)}{\Ode{1}}{Devuelve la cantidad de hippies atrapados por el agente pasado como par\'ametro.}

\disDeclaraFuncion{ComenzarRastrillaje}{\paramIn{c}{campus}, \paramIn{d}{dicc(agente,posicion)}}{res : rastr}{($\forall$ a : agente)(def?(a,d) $\impluego$ (posValida?(obtener(a,d))) $\land$ $\¬$ocupada?(obtener(a,d),c)) $\land$ ($\forall$ a, $a_{2}$ : agente)((def?(a,d) $\land$ def?($a_{2}$,d) $\land$ a $\not=a_{2}$) $\impluego$ obtener(a,d)$\not=$ obtener($a_{2}$,d))}{res $\igobs$ comenzarRastrillaje(c,d)}{\Ode{1}}{Crea un Rastrillaje.}

\disDeclaraFuncion{IngresarEstudiante}{\paramInOut{r}{rastr}, \paramIn{e}{nombre}, \paramIn{p}{posicion}}{}{r = $r_{0}$ $\land$ e $\notin$ (estudiantes(r)$\cup$hippies(r)) $\land$ esIngreso?(p,campus(r)) $\land$ $\¬$estaOcupada?(p,r)}{r $\igobs$ ingresarEstudiante(e,p,$r_{0}$)}{\Ode{1}}{Modifica el rastrillaje, ingresando un estudiante al campus.}

\disDeclaraFuncion{IngresarHippie}{\paramInOut{r}{rastr}, \paramIn{h}{nombre}, \paramIn{p}{posicion}}{}{r = $r_{0}$ $\land$ h $\notin$ (estudiantes(r)$\cup$hippies(r)) $\land$ esIngreso?(p,campus(r)) $\land$ $\¬$estaOcupada?(p,r)}{r $\igobs$ ingresarHippie(h,p,$r_{0}$)}{\Ode{1}}{Modifica el rastrillaje, ingresando un hippie al campus.}

\disDeclaraFuncion{MoverEstudiante}{\paramInOut{r}{rastr}, \paramIn{e}{nombre}, \paramIn{dir}{direccion}}{}{r = $r_{0}$ $\land$ e $\in$ estudiantes(r) $\land$ (seRetira(e,dir,r) $\lor$ (posValida?(proxPosicion(posEstudianteYHippie(e,r),dir,campus(r)),campus(r)) $\land$ $\¬$estaOcupada?(proxPosicion(posEstudianteYHippie(e,r),dir,campus(r)),r)))}{r $\igobs$ moverEstudiante(e,d,$r_{0}$)}{\Ode{1}}{Modifica el rastrillaje, al mover un estudiante del campus.}

\disDeclaraFuncion{MoverHippie}{\paramInOut{r}{rastr}, \paramIn{h}{nombre}}{}{r = $r_{0}$ $\land$ h $\in$ hippies(r) $\land$ $\¬$todasOcupadas?(vecinos(posEstudianteYHippie(h,r),campus(r)),r) }{r $\igobs$ moverHippie(r,$r_{0}$)}{\Ode{1}}{Modifica el rastrillaje, al mover un hippie del campus.}

\disDeclaraFuncion{MoverAgente}{\paramInOut{r}{rastr}, \paramIn{a}{agente}}{}{r = $r_{0}$ $\land$ a $\in$ agentes(r) $\yluego$ cantSanciones(a,r) $\leq$ 3 $\land$ $\¬$todasOcupadas?(vecinos(posAgente(a,r),campus(r)),r)}{r $\igobs$ moverAgente(a,$r_{0}$)}{\Ode{1}}{Modifica el rastrillaje, al mover un agente del campus.}

\disDeclaraFuncion{MasVigilante}{\paramIn{r}{rastr}}{res : agente}{true}{res $\igobs$ masVigilante(r)}{\Ode{1}}{Devuelve el agente con mas capturas.}

\disDeclaraFuncion{ConKSanciones}{\paramIn{r}{rastr}, \paramIn{k}{nat}}{res : conj(agente)}{true}{res $\igobs$ conKSanciones(k,r)}{\Ode{1}}{Devuelve el agente con mas capturas.}

\disDeclaraFuncion{ConMismasSanciones}{\paramIn{r}{rastr}, \paramIn{a}{agente}}{res : conj(agente)}{a $\in$ agentes(r)}{res $\igobs$ conMismasSanciones(a,r)}{\Ode{1}}{Devuelve el conjunto de agentes con la misma cantidad de sanciones que a.}





\disPautasDeImplementacion

\disEstructuraDeRepresentacion

\disSeRepresentaCon{campus}{estr}
\disDondeEs{estr}{\disTuplaEstr{campus/campo, diccPromedio(agente ; datosAg)/agentes, arreglo(tupla(placa;posicion))/posAgentesLog, conjLineal(datosHoE)/hippies, conjLineal(datosHoE)/estudiantes, diccString(nombre;posicion)/posCiviles, diccLineal(nombre;posicion)/posRapida, vector(vector(datosPos))/quienOcupa, itConj(agente)/masVigilante, lista(datosK))/agregoEn1, vector(datosK)/buscoEnLog}}

\disDondeEs{datosAg}{\disTuplaEstr{nat/QSanciones, nat/premios, posicion/posActual, itConj(agente)/grupoSanciones, itLista(datosK)/verK}}

\disDondeEs{datosHoE}{\disTuplaEstr{nombre/ID, itDicc(nombre;posicion)/posActual}}

\disDondeEs{datosPos}{\disTuplaEstr{bool/ocupada?, clases/queHay, itDicc(agente)/hayCana, itConj(nombre)/hayHoE}}

\disDondeEs{clases}{enum\{``agente'',``estudiante'',``hippie'',``obstaculo'',``nada''\}}

\disDondeEs{datosK}{\disTuplaEstr{nat/K, conjLineal(agente)/grupoK}}

\disJustificacionDeLaEstructuraElegida
\newpage
\disInvarianteDeRepresentacion
\hspace*{\disSubSubSecMargen}\textbf{\textsf{Informal}}

\hspace*{\disSubSubSecMargen}
\begin{enumerate}
\setlength{\itemindent}{3em}
  \item El mapa debe tener tantas filas como indica la estructura, lo mismo con las columnas.

\end{enumerate}

\hspace*{\disSubSubSecMargen}\textbf{\textsf{Formal}}
\onehalfspace
\disRep{estr}{e}{$true$ $\Longleftrightarrow$ 
\\\hspace*{3.75em}(1)(2)(3) ($\forall$ a,a2: Agente)(def?(a,e.agentes) $\wedge$ def?(a2,e.agentes)) $\impluego$ \\ (obtener(a,e.agentes).Qsanciones=siguiente(obtener(a,e.agentes).verK).K \\ $\wedge$ obtener(a,e.agentes).grupoSanciones=siguiente(obtener(a,e.agentes).verK).grupoK \\ $\wedge$ (a2 $\in$ obtener(a,e.agentes).grupoSanciones) $\Longleftrightarrow$ (obtener(a,e.agentes).Qsanciones = obtener(a2,e.agentes).Qsanciones) $\wedge$ (4) PosValida(e,obtener(a,e.agentes).PosActual)) $\wedge$ (5) TodasLasPlacas(e,e.PosAgentesLog())}
\disFuncionDeAbstraccion
\vspace*{-1em}
%\hspace*{\disSubSubSecMargen}{Texto}
\disAbs{estr}{e}{campus}{c}{\Big(filas(c) = e.filas $\land$ columnas(c) = e.columnas $\yluego$ ($\forall$ p : posicion)(p.X $\leq$ e.filas $\land$
\\\hspace*{3.75em} p.Y $\leq$ e.columnas $\impluego$ ocupada?(p,c) $\Leftrightarrow$ (e.mapa[f])[c]\Big) }


%\disFuncionDeAbsFuncionesAux


\newpage

\disAlgoritmos
%\hspace*{\disSubSubSecMargen}{Texto}
% HACK: SGA 28/05/2011. Para quitar el titulo Algorithm del caption \floatname{algorithm}{}
\floatname{algorithm}{}
% WARNING: SGA 27/05/2011. La opción [H] indica a LaTex que el algoritmo lo queremos AQUI!
% Ver 4.4.1 Placement of Algorithms de algorithms.pdf.

\begin{algorithm}[H]
\begin{algorithmic}[1]
\Function {\textsc{$i$Campus}}{\paramIn{e}{estr}}{$\disFlecha$ res : campus} \Comment{$\Ode{1}$}
  \State res $\gets$ e.campo
\EndFunction
\end{algorithmic}
\end{algorithm}


\begin{algorithm}[H]
\begin{algorithmic}[1]
\Function {\textsc{$i$Estudiantes}}{\paramIn{e}{estr}}{$\disFlecha$ res : itConj(nombre)} \Comment{$\Ode{1}$}
  \State res $\gets$ crearIt (e.estudiantes)
\EndFunction
\end{algorithmic}
\end{algorithm}


\begin{algorithm}[H]
\begin{algorithmic}[1]
\Function {\textsc{$i$Hippies}}{\paramIn{e}{estr}}{$\disFlecha$ res : itConj(nombre)} \Comment{$\Ode{1}$}
  \State res $\gets$ crearIt (e.hippies)
\EndFunction
\end{algorithmic}
\end{algorithm}

\begin{algorithm}[H]
\begin{algorithmic}[1]
\Function {\textsc{$i$Agentes}}{\paramIn{e}{estr}}{$\disFlecha$ res : itConj(agente)} \Comment{$\Ode{1}$}
  \State res $\gets$ crearIt (e.agentes) 
\EndFunction
\end{algorithmic}
\end{algorithm}

\begin{algorithm}[H]
\begin{algorithmic}[1]
\Function {\textsc{$i$posEstudiantesYHippie}}{\paramIn{n}{nombre} \paramIn{e}{estr}}{$\disFlecha$ res : posicion} \Comment{$\Ode{\longitud{n_{m}}}$}
    \State res $\gets$ obtener(n,e.posCiviles)
\EndFunction
\end{algorithmic}
\end{algorithm}

\begin{algorithm}[H]
\begin{algorithmic}[1]
\Function {\textsc{$i$posAgente}}{\paramIn{a}{agente} \paramIn{e}{estr}}{$\disFlecha$ res : posicion} \Comment{$\Ode{1}(promedio)$}
    \State res $\gets$ obtener(a,e.agentes).posActual
\EndFunction
\end{algorithmic}
\end{algorithm}

\begin{algorithm}[H]
\begin{algorithmic}[1]
\Function {\textsc{$i$cantSanciones}}{\paramIn{a}{agente} \paramIn{e}{estr}}{$\disFlecha$ res : nat} \Comment{$\Ode{1}(promedio)$}
    \State res $\gets$ obtener(a,e.agentes).Qsanciones
\EndFunction
\end{algorithmic}
\end{algorithm}

\begin{algorithm}[H]
\begin{algorithmic}[1]
\Function {\textsc{$i$cantHippiesAtrapados}}{\paramIn{a}{agente} \paramIn{e}{estr}}{$\disFlecha$ res : nat} \Comment{$\Ode{1}(promedio)$}
    \State res $\gets$ obtener(a,e.agentes).premios
\EndFunction
\end{algorithmic}
\end{algorithm}

























%% Dise�o del tipo T
\newpage

% Dise�o del Tipo
\disDisenio{DiccionarioString($\sigma$)}
% La especificaci�n
\disEspecificacion
\hspace*{\disSubSecMargen}Se usa el {\sc Tad Diccionario($\kappa$, $\sigma$)} especificado en el apunte de Tads b\'asicos.

\disAspectosDeLaInterfaz

\disInterfaz

\disParametrosFormales{$\kappa,\sigma$}

\disParametrosFormalesDeclaraFunc{\puntito = \puntito}{\paramIn{a_1}{\kappa}, \paramIn{a_2}{\kappa}}{res : bool}{true}{res \igobs (a_1 = a_2)}{\ThetaDe{equals(a_1, a_2)}}{funci\'on de igualdad de $\kappa$'s}
\disParametrosFormalesDeclaraFunc{COPIAR}{\paramIn{k}{\kappa}}{res : \kappa}{true}{res \igobs k}{\ThetaDe{copy(k)}}{funci\'on de copia de $\kappa$'s}
\disParametrosFormalesDeclaraFunc{COPIAR}{\paramIn{s}{\sigma}}{res : \sigma}{true}{res \igobs s}{\ThetaDe{copy(s)}}{funci\'on de copia de $\sigma$'s}

\disSeExplicaCon{Diccionario($\kappa,\sigma$), Iterador Bidireccional(Tupla($\kappa,\sigma$))}

\disGenero{diccString($\kappa$,$\sigma$)}

\disOperaciones{b\'asicas de diccionario}

\disDeclaraFuncion{Definido?}{\paramIn{d}{diccString(\kappa,\sigma)}, \paramIn{k}{\kappa}}{res : bool}{true}{res \igobs def?(d, k)}{\Ode{|k|}\ |k|\ es\ la\ longitud\ de\ la\ clave.}{Devuelve true si y s\'olo si $k$ est\'a definido en el diccionario.}

\disDeclaraFuncion{Obtener}{\paramIn{d}{diccString(\kappa,\sigma)}, \paramIn{k}{\kappa}}{res : \sigma}{def?(d, k)}{alias(res \ $\igobs$ obtener(d, k))}{\Ode{|k|}\ |k|\ es\ la\ longitud\ de\ la\ clave.}{Devuelve el significado de la clave $k$ en $d$.}
\disComentAliasing{ res no es modificable.}


\disDeclaraFuncion{Vacio}{}{res : diccString(\kappa,\sigma)}{true}{res $\igobs$ vacio()}{\Ode{1}}{Genera un diccionario vac\'io.}

\disDeclaraProc{Definir}{\paramInOut{d}{diccString(\kappa,\sigma)}, \paramIn{k}{\kappa}, \paramIn{s}{\sigma}}{d \igobs d_0}{d \igobs definir(k, s, d_0)}{\Ode{|k|}\ |k|\ es\ la\ longitud\ de\ la\ clave.}{Define la clave $k$ con el significado $s$ en el diccionario.}

\disDeclaraFuncion{Borrar}{\paramInOut{d}{diccString(\kappa,\sigma)}, \paramIn{k}{\kappa}}{res : bool}{d=$d_{0}$ \land def?(k,d)}{d \igobs borrar(k,$d_{0}$)}{\Ode{|k|}\ |k|\ es\ la\ longitud\ de\ la\ clave.}{Elimina la clave k del diccionario.}



\disOperaciones{b\'asicas del iterador}

\disDeclaraFuncion{CrearIt}{\paramIn{d}{diccString(\kappa,\sigma)}}{res : itdiccString(\kappa,\sigma)}{true}{alias(esPermutacion(SecuSuby(res),d)) \ \land vacia?(Anteriores(res))}{\Ode{n}\ n\ es\ la\ cantidad\ de\ claves.}{Crea un iterador del diccionario de forma tal que se puedan recorrer sus elementos aplicando iterativamente SIGUIENTE(no ponemos la operacion SIGUIENTE en la interfaz pues no la usamos).}

\disDeclaraFuncion{HaySiguiente}{\paramIn{it}{itdiccString(\kappa,\sigma)}}{res : bool}{true}{res $\igobs$ HaySiguiente?(it)}{\Ode{1}}{Devuelve true si y solo si en el iterador quedan elementos para avanzar.}

\disDeclaraFuncion{SiguienteSignificado}{\paramIn{it}{itdiccString(\kappa,\sigma)}}{res : \sigma}{HaySiguiente?(it)}{alias(res $\igobs$ Siguiente(it).significado)}{\Ode{1}}{Devuelve el significado del elemento siguiente del iterador.}
\disComentAliasing{ res no es modificable.}


\disDeclaraProc{Avanzar}{\paramInOut{it}{itdiccString(\kappa,\sigma)}}{it \ $\igobs$ it_{0} \ $\land$ HaySiguiente?(it)}{it \ $\igobs$ Avanzar(it_{0})}{\Ode{1}}{Avanza a la posicion siguiente del iterador.}




\disPautasDeImplementacion

\disEstructuraDeRepresentacion

\disSeRepresentaCon{dicc\_trie(\kappa,\sigma)}{puntero(nodo)}
\disDondeEs{nodo}{\disTuplaEstr{Puntero($\sigma$)/significado, arreglo[256] $de \ puntero (nodo)$/caracteres, Puntero(nodo)/padre}}

\disJustificacionDeLaEstructuraElegida

%\disEstructurasAlternativas

\disInvarianteDeRepresentacion
\hspace*{\disSubSubSecMargen}\textbf{\textsf{Informal}}

\hspace*{\disSubSubSecMargen}
\begin{itemize}
% HACK: SGA 20/06/2011. Para identar correctmente los items.
\setlength{\itemindent}{3em}
  \item Todas las posiciones del arreglo de caracteres est�n definidas.
  \item No hay claves de 0 caracteres. El significado de la ra�z es NULL.
  \item No hay ciclos en la estructura. Es decir, existe una cota superior sobre la cantidad de niveles posibles del �rbol.
  \item Dado un nodo cualquiera del trie, existe un �nico camino desde la ra�z hasta el nodo.
\end{itemize}

\hspace*{\disSubSubSecMargen}\textbf{\textsf{Formal}}

\disRep{estr}{e}{$true$ $\Longleftrightarrow$ 
\\\hspace*{3.75em}(1)(\forall i:nat)(i<256 $\implies$ definido?(e $\rightarrow$ caracteres,i)) \yluego
\\\hspace*{3.75em}(2)(e \rightarrow significado = NULL) \yluego
\\\hspace*{3.75em}(2)($\exists$ n:nat)(finaliza(e,n)) \yluego
\\\hspace*{3.75em}(3)($\forall$ p,q: puntero(nodo))(p $\in$ punteros(e) \land q \in (punteros(e) - \lbrace p\rbrace ) $\implies$ p\not=q) \yluego
\\\hspace*{3.75em}}


\disFuncionDeAbstraccion
\vspace*{-1em}
%\hspace*{\disSubSubSecMargen}{Texto}
\disAbs{roseTree(estrDato)}{r}{dicc\_trie($\sigma$)}{d}{($\forall$ $k$ : secu($letra$))(def?(k, d) $\igobs$ esta?(k, r)) $\land$ (def?(c, d) $\implies$ (obtener(k, d) $\igobs$ buscar(k, r)))}


\disFuncionDeAbsFuncionesAux


\newpage
\disAlgoritmos
%\hspace*{\disSubSubSecMargen}{Texto}
% HACK: SGA 28/05/2011. Para quitar el titulo Algorithm del caption \floatname{algorithm}{}
\floatname{algorithm}{}
% WARNING: SGA 27/05/2011. La opci�n [H] indica a LaTex que el algoritmo lo queremos AQUI!
% Ver 4.4.1 Placement of Algorithms de algorithms.pdf.
\begin{algorithm}\phantom{[H]}
\begin{algorithmic}[1]
\Function {\textsc{iVacio}}{ }{$\disFlecha$ res : estr} \Comment{$\Ode{1}$}
	\State var arreglo(puntero(nodo)) letras $\gets$ crearArreglo[256]
	\State \textbf{for} i $\gets$ 0 \textbf{to} 255 \textbf{do} \Comment{$\Ode{1}$}
  	\State letras[i] $\gets$ NULL \Comment{\Ode{1}}
  	\State \textbf{end for}
  	\State var nodo nuevo $\gets$ $<$NULL,letras,NULL$>$ \Comment{$\Ode{1}$}
  	\State res $\gets$ \&nuevo \Comment{$\Ode{1}$}
\EndFunction
\end{algorithmic}
\end{algorithm}

\begin{algorithm}\phantom{[H]}
\begin{algorithmic}[1]
\Function {\textsc{iDefinir}}{\paramInOut{d}{estr}, \paramIn{k}{string}, \paramIn{s}{\sigma}}
    \State nat $i$ $\gets$ 0 \Comment{$\Ode{1}$}
    \State puntero(nodo) actual $\gets$ d \Comment{$\Ode{1}$}
    \While {($i$ $<$ $|k|$)} \Comment{$\Ode{|k|}$}
        \If {actual $\disFlecha$ caracteres[ord(k[i])] = NULL} \Comment{$\Ode{1}$}
        	\State puntero(nodo) anterior $\gets$ actual
            \State actual $\disFlecha$ caracteres[ord(k[i])] $\gets$ $i$Vacio() \Comment{$\Ode{1}$}
            \State actual $\disFlecha$ padre $\gets$ anterior
        \Else
        \State actual $\gets$ (actual $\disFlecha$ caracteres[ord(k[i])]) \Comment{$\Ode{1}$}        
        \EndIf
        \State $i$ $\gets$ $i + 1$ \Comment{$\Ode{1}$}
    \EndWhile
        \State actual $\disFlecha$ significado $\gets$ $\&$copiar(s) \Comment{$\Ode{1}$}
\EndFunction
\end{algorithmic}
\end{algorithm}

\begin{algorithm}\phantom{[H]}
\begin{algorithmic}[1]
\Function {\textsc{iObtener}}{\paramIn{d}{estr}, \paramIn{k}{string}}{$\disFlecha$ res : $\sigma$}
  \State nat $i$ $\gets$ 0 \Comment{$\Ode{1}$}
  \State puntero actual $\gets$ d \Comment{$\Ode{1}$}
  \While {$i$ $<$ $|k|$} \Comment{$\Ode{|k|}$}
      \State actual $\gets$ (actual $\disFlecha$ caracteres[ord(k[i])]) \Comment{$\Ode{1}$}
      \State $i$ $\gets$ $i+1$
  \EndWhile
  \State res $\gets$ *(actual $\disFlecha$ significado)
\EndFunction
\end{algorithmic}
\end{algorithm}

\begin{algorithm}\phantom{[H]}
\begin{algorithmic}[1]
\Function {\textsc{iBorrar}}{\paramInOut{d}{estr}, \paramIn{k}{string}}{}
	\State puntero(nodo) actual $\gets$ d
	\State \textbf{for} i $\gets$ 0 to $|k|$ \Comment{$\Ode{1}$}
  	\State actual $\gets$ (actual\disFlecha caracteres[ord(k[i])]) \Comment{\Ode{1}}
  	\State \textbf{end for}
  	\State (actual\disFlecha significado) $\gets$ NULL
  	var puntero(nodo) camino $\gets$ NULL
  	\While {(actual\disFlecha significado = NULL) \textbf{or} todosNULL(actual\disFlecha caracteres)} \Comment{$\Ode{|k|}$}
    	\State camino $\gets$ actual \Comment{$\Ode{1}$}
    	\State actual $\gets$ (actual \disFlecha padre)
    	\State delete camino
  	\EndWhile
\end{algorithmic}
\end{algorithm}


\begin{algorithm}\phantom{[H]}
\begin{algorithmic}[1]
\Function {\textsc{iDefinido?}}{\paramIn{d}{estr}, \paramIn{k}{string}}{$\disFlecha$ res : bool}
  \State nat $i$ $\gets$ 0 \Comment{$\Ode{1}$}
  \State puntero actual $\gets$ d \Comment{$\Ode{1}$}
  \State bool def $\gets$ \textsf{true} \Comment{$\Ode{1}$}
  \While {($i$ $<$ $|k|$ \textbf{and} def)} \Comment{$\Ode{|k|}$}
    \If {actual $\disFlecha$ caracteres[ord(k[i])] = NULL} \Comment{$\Ode{1}$}
      \State def $\gets$ \textsf{false} \Comment{$\Ode{1}$}
      \Else
        \State actual $\gets$ actual $\disFlecha$ caracteres[ord(k[i])] \Comment{$\Ode{1}$}
        \State $i$ $\gets$ $i+1$ \Comment{$\Ode{1}$}
      \EndIf
  \EndWhile
  \State res $\gets$ def $\land$ $�$(actual $\disFlecha$ significado(NULL)) \Comment{$\Ode{1}$}

\EndFunction
\end{algorithmic}
\end{algorithm}

\newpage
\disServiciosUsados
\vspace*{-1em}
 
\disRequerimientosSobreElTipo{}
\disItemRequerimientosSobreElTipo{La funci\'on \textbf{$|$x$|$} debe tener complejidad $\Ode{1}$ en el caso peor.}
\disItemRequerimientosSobreElTipo{La funci\'on \textbf{$|$x$|$} debe tener complejidad $\Ode{1}$ en el caso peor.}
\disItemRequerimientosSobreElTipo{Las operaciones deben realizarse por referencia.}
\disItemRequerimientosSobreElTipo{Debe proveer una operaci\'on \textbf{Copia} que devuelve una nueva instancia de la secuencia pero que es\\ \hspace*{5.75em}independiente de la actual, con complejidad $\Ode{n}$ en el caso peor.}
\disItemRequerimientosSobreElTipo{Debe proveer un \textbf{iterador} para avanzar que comienza en el primero elemento de la secuencia.}
\disItemRequerimientosSobreElTipo{Debe proveer un \textbf{iterador} para retroceder que comienza en el �ltimo elemento de la secuencia.}
\disItemRequerimientosSobreElTipo{Las operaciones \textbf{CrearIt, Siguiente, Anterior, TieneSiguiente, TieneAnterior} deben tener complejidad \\
\hspace*{5.75em}$\Ode{1}$ en el caso peor.}
\vspace*{1em}
\hspace*{2em}Donde $n$ es la longitud de la palabra.


% Diseño del tipo T
\newpage

% Diseño del Tipo
\disDisenio{DiccionarioProm($\sigma$)}
% La especificación
\disEspecificacion
\hspace*{\disSubSecMargen}Se usa el {\sc Tad Diccionario($\kappa$, $\sigma$)} especificado en el apunte de Tads b\'asicos.

\disAspectosDeLaInterfaz

\disInterfaz

\disParametrosFormales{$\kappa,\sigma$}

\disParametrosFormalesDeclaraFunc{\puntito = \puntito}{\paramIn{a_1}{\kappa}, \paramIn{a_2}{\kappa}}{res : bool}{true}{res \igobs (a_1 = a_2)}{\ThetaDe{equals(a_1, a_2)}}{funci\'on de igualdad de $\kappa$'s}
\disParametrosFormalesDeclaraFunc{COPIAR}{\paramIn{k}{\kappa}}{res : \kappa}{true}{res \igobs k}{\ThetaDe{copy(k)}}{funci\'on de copia de $\kappa$'s}
\disParametrosFormalesDeclaraFunc{COPIAR}{\paramIn{s}{\sigma}}{res : \sigma}{true}{res \igobs s}{\ThetaDe{copy(s)}}{funci\'on de copia de $\sigma$'s}

\disSeExplicaCon{Diccionario($\kappa,\sigma$)}

\disGenero{diccProm($\kappa$,$\sigma$)}

\disOperaciones{b\'asicas de diccionario}

\disDeclaraFuncion{Definido?}{\paramIn{d}{diccProm(\kappa,\sigma)}, \paramIn{k}{\kappa}}{res : bool}{true}{res \igobs def?(d, k)}{\Ode{Na}\ Na\ es\ la\ cantidad\ de\ agentes.}{Devuelve true si y s\'olo si $k$ est\'a definido en el diccionario.}

\disDeclaraFuncion{Obtener}{\paramIn{d}{diccString(\kappa,\sigma)}, \paramIn{k}{\kappa}}{res : \sigma}{def?(d, k)}{alias(res \ $\igobs$ obtener(d, k))}{\Ode{Na}\ Na\ es\ la\ cantidad\ de\ agentes.}{Devuelve el significado de la clave $k$ en $d$.}
\disComentAliasing{ res no es modificable.}


\disDeclaraFuncion{Vacio}{\paramIn{cantClaves}{nat}}{res : diccString(\kappa,\sigma)}{true}{res $\igobs$ vacio()}{\Ode{Na}\ Na\ es\ la\ cantidad\ de\ agentes.}{Genera un diccionario vac\'io.}

\disDeclaraProc{Definir}{\paramInOut{d}{diccProm(\kappa,\sigma)}, \paramIn{k}{\kappa}, \paramIn{s}{\sigma}}{d \igobs d_0}{d \igobs definir(k, s, d_0)}{\Ode{1}}{Define la clave $k$ con el significado $s$ en el diccionario.}

%\disDeclaraFuncion{Borrar}{\paramInOut{d}{diccProm(\kappa,\sigma)}, \paramIn{k}{\kappa}}{res : bool}{d=$d_{0}$ \land def?(k,d)}{d \igobs borrar(k,$d_{0}$)}{\Ode{|k|}\ |k|\ es\ la\ longitud\ de\ la\ clave.}{Elimina la clave k del diccionario.}





\disPautasDeImplementacion

\disEstructuraDeRepresentacion

\disSeRepresentaCon{diccProm(\kappa,\sigma)}{estr}
\disDondeEs{estr}{\disTuplaEstr{nat/CantClaves, arreglo[CantClaves] $de \ lista (datos)$/tabla}}
\disDondeEs{datos}{\disTuplaEstr{\kappa /clave, \sigma /significado}}
\disJustificacionDeLaEstructuraElegida

%\disEstructurasAlternativas

\disInvarianteDeRepresentacion
\hspace*{\disSubSubSecMargen}\textbf{\textsf{Informal}}

\hspace*{\disSubSubSecMargen}
\begin{itemize}
% HACK: SGA 20/06/2011. Para identar correctmente los items.
\setlength{\itemindent}{3em}
  \item Todas las posiciones del arreglo de caracteres están definidas.
  \item No hay claves de 0 caracteres. El significado de la raíz es NULL.
  \item No hay ciclos en la estructura. Es decir, existe una cota superior sobre la cantidad de niveles posibles del árbol.
  \item Dado un nodo cualquiera del trie, existe un único camino desde la raíz hasta el nodo.
\end{itemize}

\hspace*{\disSubSubSecMargen}\textbf{\textsf{Formal}}

\disRep{estr}{e}{$true$ $\Longleftrightarrow$ 
\\\hspace*{3.75em}(1)(\forall i:nat)(i<256 $\implies$ definido?(e $\rightarrow$ caracteres,i)) \yluego
\\\hspace*{3.75em}(2)(e \rightarrow significado = NULL) \yluego
\\\hspace*{3.75em}(2)($\exists$ n:nat)(finaliza(e,n)) \yluego
\\\hspace*{3.75em}(3)($\forall$ p,q: puntero(nodo))(p $\in$ punteros(e) \land q \in (punteros(e) - \lbrace p\rbrace ) $\implies$ p\not=q) \yluego
\\\hspace*{3.75em}}


\disFuncionDeAbstraccion
\vspace*{-1em}
%\hspace*{\disSubSubSecMargen}{Texto}
\disAbs{roseTree(estrDato)}{r}{dicc\_trie($\sigma$)}{d}{($\forall$ $k$ : secu($letra$))(def?(k, d) $\igobs$ esta?(k, r)) $\land$ (def?(c, d) $\implies$ (obtener(k, d) $\igobs$ buscar(k, r)))}


\disFuncionDeAbsFuncionesAux


\newpage
\disAlgoritmos
%\hspace*{\disSubSubSecMargen}{Texto}
% HACK: SGA 28/05/2011. Para quitar el titulo Algorithm del caption \floatname{algorithm}{}
\floatname{algorithm}{}
% WARNING: SGA 27/05/2011. La opción [H] indica a LaTex que el algoritmo lo queremos AQUI!
% Ver 4.4.1 Placement of Algorithms de algorithms.pdf.
\begin{algorithm}\phantom{[H]}
\begin{algorithmic}[1]
\Function {\textsc{iVacio}}{\paramIn{cantClaves}{nat} }{$\disFlecha$ res : estr} \Comment{$\Ode{cantClaves}$}
	\State var arreglo(lista(datos)) tabla $\gets$ crearArreglo[cantClaves] \Comment{$\Ode{cantClaves}$}
	\State \textbf{for} i $\gets$ 0 \textbf{to} cantClaves \textbf{do} \Comment{$\Ode{cantClaves}$}
  	\State tabla[i] $\gets$ Vacia() \Comment{\Ode{1}}
  	\State \textbf{end for}
  	\State var datos nuevo $\gets$ $<$tabla,cantClaves$>$ \Comment{$\Ode{1}$}
  	\State res $\gets$ \&nuevo \Comment{$\Ode{1}$}
\EndFunction
\end{algorithmic}
\end{algorithm}

\begin{algorithm}\phantom{[H]}
\begin{algorithmic}[1]
\Function {\textsc{iDefinir}}{\paramInOut{d}{estr}, \paramIn{k}{nat}, \paramIn{s}{\sigma}} \Comment{$\Ode{1}$}
    \State nat $i$ $\gets$ fHash(k, e.cantClaves) \Comment{$\Ode{1}$}
    \State e.tabla[i] $\gets$ AgAtras(e.tabla[i],$<$k,s$>$) \Comment{$\Ode{1}$}
    
\EndFunction
\end{algorithmic}
\end{algorithm}

\begin{algorithm}\phantom{[H]}
\begin{algorithmic}[1]
\Function {\textsc{iObtener}}{\paramIn{d}{estr}, \paramIn{k}{nat}}{$\disFlecha$ res : $\sigma$} \Comment{$\Ode{longitud(tabla[i])}$}
  \State nat $i$ $\gets$ fHash(k, e.cantClaves) \Comment{$\Ode{1}$}
  \State bool flag $\gets$ false \Comment{$\Ode{1}$}
  \State nat $j$ $\gets$ 0 \Comment{$\Ode{1}$}

  \While { $\neg$ flag $\wedge$ j $<$ longitud(tabla[i]) } \Comment{$\Ode{longitud(tabla[i])}$}
	  \If {datos.clave.tabla[i][j]=k} \Comment{$\Ode{1}$}
      \State res $\gets$ datos.significado.tabla[i][j] \Comment{$\Ode{1}$}
      \State aux $\gets$ true \Comment{$\Ode{1}$}
     % \Else
      %  \State actual $\gets$ actual $\disFlecha$ caracteres[ord(k[i])] \Comment{$\Ode{1}$}
       % \State $i$ $\gets$ $i+1$ \Comment{$\Ode{1}$}
      \EndIf      
      \State $j$ $\gets$ $j+1$
  \EndWhile
\EndFunction
\end{algorithmic}
\end{algorithm}



\begin{algorithm}\phantom{[H]}
\begin{algorithmic}[1]
\Function {\textsc{iDefinido?}}{\paramIn{d}{estr}, \paramIn{k}{nat}}{$\disFlecha$ res : bool} \Comment{$\Ode{longitud(tabla[i])}$}
  \State nat $i$ $\gets$ fHash(k, e.cantClaves) \Comment{$\Ode{1}$}
  \State nat $j$ $\gets$ 0 \Comment{$\Ode{1}$}
  \State bool def $\gets$ \textsf{false} \Comment{$\Ode{1}$}
  \While {($j$ $<$ longitud(tabla[i]) \textbf{and} $\neg$ def)} \Comment{$\Ode{longitud(tabla[i])}$}
    \If {datos.clave.tabla[i][j]=k} \Comment{$\Ode{1}$}
      \State def $\gets$ \textsf{true} \Comment{$\Ode{1}$}
     
      \EndIf
  \EndWhile
  \State res $\gets$ def  \Comment{$\Ode{1}$}

\EndFunction
\end{algorithmic}
\end{algorithm}

\begin{algorithm}\phantom{[H]}
\begin{algorithmic}[1]
\Function {\textsc{fHash}}{\paramIn{k}{nat}, \paramIn{cantClaves}{nat}}{$\disFlecha$ res : nat} \Comment{$\Ode{1}$}
  \State res $\gets$ k mod cantClaves \Comment{$\Ode{1}$}
\EndFunction
\end{algorithmic}
\end{algorithm}


\newpage
\disServiciosUsados
\vspace*{-1em}
 
\disRequerimientosSobreElTipo{}
\disItemRequerimientosSobreElTipo{La funci\'on \textbf{$|$x$|$} debe tener complejidad $\Ode{1}$ en el caso peor.}
\disItemRequerimientosSobreElTipo{La funci\'on \textbf{$|$x$|$} debe tener complejidad $\Ode{1}$ en el caso peor.}
\disItemRequerimientosSobreElTipo{Las operaciones deben realizarse por referencia.}
\disItemRequerimientosSobreElTipo{Debe proveer una operaci\'on \textbf{Copia} que devuelve una nueva instancia de la secuencia pero que es\\ \hspace*{5.75em}independiente de la actual, con complejidad $\Ode{n}$ en el caso peor.}
\disItemRequerimientosSobreElTipo{Debe proveer un \textbf{iterador} para avanzar que comienza en el primero elemento de la secuencia.}
\disItemRequerimientosSobreElTipo{Debe proveer un \textbf{iterador} para retroceder que comienza en el último elemento de la secuencia.}
\disItemRequerimientosSobreElTipo{Las operaciones \textbf{CrearIt, Siguiente, Anterior, TieneSiguiente, TieneAnterior} deben tener complejidad \\
\hspace*{5.75em}$\Ode{1}$ en el caso peor.}
\vspace*{1em}
\hspace*{2em}Donde $n$ es la longitud de la palabra.





\end{document}

