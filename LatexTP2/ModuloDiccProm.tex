% Diseño del tipo T
\newpage

% Diseño del Tipo
\disDisenio{DiccionarioProm}
% La especificación
\disEspecificacion
\hspace*{\disSubSecMargen}Se usa el {\sc Tad DiccionarioM} (Nota al corrector: leer observaciones).

\disAspectosDeLaInterfaz

\disInterfaz

\disSeExplicaCon{DiccionarioM($\kappa,\sigma$)}

\disGenero{diccProm($\kappa$,$\sigma$)}

\disOperaciones{b\'asicas de diccionario}

\disDeclaraFuncion{Definido?}{\paramIn{d}{diccProm(\kappa,\sigma)}, \paramIn{k}{\kappa}}{res : bool}{true}{res \igobs def?(d, k)}{\Ode{n}\ n\ es\ la\ cantidad\ de\ claves.}{Devuelve true si y s\'olo si $k$ est\'a definido en el diccionario.}

\disDeclaraFuncion{Obtener}{\paramIn{d}{diccProm(\kappa,\sigma)}, \paramIn{k}{\kappa}}{res : \sigma}{def?(d, k)}{alias(res \ $\igobs$ obtener(d, k))}{\Ode{n}\ n\ es\ la\ cantidad\ de\ claves.}{Devuelve el significado de la clave $k$ en $d$.}
\disComentAliasing{ se devuelve una referencia al significado de la clave.}

\disDeclaraFuncion{Claves}{\paramIn{d}{diccProm(\kappa,\sigma)}}{res : itConj($\kappa$)}{true}{res \igobs claves(d)}{\Ode{1}}{Devuelve el conjunto con las claves definidas en d.}

\disDeclaraFuncion{Vacio}{\paramIn{n}{nat}}{res : diccProm(\kappa,\sigma)}{true}{res $\igobs$ vacio(n)}{\Ode{n}}{Genera un diccionario vac\'io, donde n acota superiormente a la cantidad de claves.}

\disDeclaraProc{Definir}{\paramInOut{d}{diccProm(\kappa,\sigma)}, \paramIn{k}{\kappa}, \paramIn{s}{\sigma}}{d \igobs d_0}{d \igobs definir(k, s, d_0)}{\Ode{1}}{Define la clave $k$ con el significado $s$ en el diccionario.}

%\disDeclaraFuncion{Borrar}{\paramInOut{d}{diccProm(\kappa,\sigma)}, \paramIn{k}{\kappa}}{res : bool}{d=$d_{0}$ \land def?(k,d)}{d \igobs borrar(k,$d_{0}$)}{\Ode{|k|}\ |k|\ es\ la\ longitud\ de\ la\ clave.}{Elimina la clave k del diccionario.}


\disPautasDeImplementacion

\disEstructuraDeRepresentacion

\disSeRepresentaCon{diccProm(\kappa,\sigma)}{estr}
\disDondeEs{estr}{\disTuplaEstr{conjLineal($\kappa$)/Cclaves, nat/clavesMax, arreglo $de \ lista (datos)$/tabla}}
\disDondeEs{datos}{\disTuplaEstr{$\kappa$/clave, $\sigma$/significado}}
\disJustificacionDeLaEstructuraElegida

%\disEstructurasAlternativas

\disInvarianteDeRepresentacion
\hspace*{\disSubSubSecMargen}\textbf{\textsf{Informal}}

\hspace*{\disSubSubSecMargen}
\begin{itemize}
% HACK: SGA 20/06/2011. Para identar correctmente los items.
\setlength{\itemindent}{3em}
  \item clavesMax es mayor que cero.
  \item La longitud del arreglo es igual a clavesMax.
  \item Todas las posiciones del arreglo estan definidas.
  \item Todos los elementos de Cclaves estan definidos en la tabla y viceversa.  
  \item Todas las claves de la tabla estan definidos en Cclaves.  

\end{itemize}

\hspace*{\disSubSubSecMargen}\textbf{\textsf{Formal}}

\disRep{estr}{e}{$true$ $\Longleftrightarrow$ 
\\\hspace*{3.75em}(1) e.clavesMax $>$ 0 \yluego
\\\hspace*{3.75em}(2) longitud(e.tabla) == e.clavesMax $\land$
\\\hspace*{3.75em}(3) ($\forall$ i: nat)(i $\leq$ e.clavesMax \impluego definido?(e.tabla,i)) $\land$
\\\hspace*{3.75em}(3) ($\forall$ k: $\kappa$)(k $\in$ e.Cclaves \implies ($\exists$ j : nat)(estaEn?(e.tabla[j],k))) $\land$
\\\hspace*{3.75em}(4) ($\forall$ i: nat)($\forall$ k: $\kappa$)(i \< e.clavesMax \yluego estaEn?(e.tabla[i],k) \implies k $\in$ e.Ccclaves )
}

\singlespace
\disFuncionDeAbsFuncionesAux
\tadOperacion{\tadNombreFuncion{\hspace*{3.7em}estaEn?}}{lista(datos), $\kappa$}{bool}{}
\tadAxioma{\hspace*{2.3em}estaEn?(l,k)}{($\exists$ i : nat)(i$<$longitud(l) \impluego l[i].clave == k)}\mbox{}




\disFuncionDeAbstraccion
\vspace*{-1em}
%\hspace*{\disSubSubSecMargen}{Texto}
\disAbs{estr}{e}{DiccionarioProm($\kappa$, $\sigma$)}{d}{cla}


\disFuncionDeAbsFuncionesAux


\newpage
\disAlgoritmos
%\hspace*{\disSubSubSecMargen}{Texto}
% HACK: SGA 28/05/2011. Para quitar el titulo Algorithm del caption \floatname{algorithm}{}
\floatname{algorithm}{}
% WARNING: SGA 27/05/2011. La opción [H] indica a LaTex que el algoritmo lo queremos AQUI!
% Ver 4.4.1 Placement of Algorithms de algorithms.pdf.
\begin{algorithm}[H]
\begin{algorithmic}[1]
\Function {\textsc{iVacio}}{\paramIn{n}{nat} }{$\disFlecha$ res : estr} \Comment{$\Ode{clavesMax}$}
	\State var arreglo(lista(datos)) tabla $\gets$ crearArreglo[n] \Comment{$\Ode{clavesMax}$}
	\State \textbf{for} i $\gets$ 0 \textbf{to} n \textbf{do} \Comment{$\Ode{clavesMax}$}
  	\State tabla[i] $\gets$ Vacia() \Comment{$\Ode{1}$}
  	\State \textbf{end for}
  	\State res $\gets$ $<$n,tabla$>$ \Comment{$\Ode{1}$}
\EndFunction
\end{algorithmic}
\end{algorithm}

\begin{algorithm}[H]
\begin{algorithmic}[1]
\Function {\textsc{iDefinir}}{\paramInOut{d}{estr}, \paramIn{k}{nat}, \paramIn{s}{\sigma}} \Comment{$\Ode{1}$}
    \State nat $i$ $\gets$ fHash(k, e.clavesMax) \Comment{$\Ode{1}$}
    \State e.tabla[i] $\gets$ AgregarAtras(e.tabla[i],$<$k,s$>$) \Comment{$\Ode{1}$}
\EndFunction
\end{algorithmic}
\end{algorithm}

\begin{algorithm}[H]
\begin{algorithmic}[1]
\Function {\textsc{iObtener}}{\paramIn{d}{estr}, \paramIn{k}{nat}}{$\disFlecha$ res : $\sigma$} \Comment{$\Ode{longitud(tabla[i])}$}
  \State nat $i$ $\gets$ fHash(k, e.clavesMax) \Comment{$\Ode{1}$}
  \State var itLista(datos) it $\gets$ crearIt(tabla[i])
  \While {haySiguiente(it)} 
  \If {siguiente(it).clave = k}
    \State res $\gets$ &siguiente(it).significado
  \EndIf
  \EndWhile
\EndFunction
\end{algorithmic}
\end{algorithm}



\begin{algorithm}[H]
\begin{algorithmic}[1]
\Function {\textsc{iDefinido?}}{\paramIn{d}{estr}, \paramIn{k}{nat}}{$\disFlecha$ res : bool} \Comment{$\Ode{longitud(tabla[i])}$}
  \State nat $i$ $\gets$ fHash(k, e.clavesMax) \Comment{$\Ode{1}$}
  \State var itLista(datos) it $\gets$ crearIt(tabla[i])
  \State bool aux $\gets$ false
  \While {haySiguiente(it)}
  \If {siguiente(it).clave = k}
    \State aux $\gets$ true
  \EndIf
  \EndWhile
  \State res $\gets$ aux
\EndFunction
\end{algorithmic}
\end{algorithm}

\begin{algorithm}[H]
\begin{algorithmic}[1]
\Function {\textsc{fHash}}{\paramIn{k}{nat}, \paramIn{clavesMax}{nat}}{$\disFlecha$ res : nat} \Comment{$\Ode{1}$}
  \State res $\gets$ k mod clavesMax \Comment{$\Ode{1}$}
\EndFunction
\end{algorithmic}
\end{algorithm}

\begin{algorithm}[H]
\begin{algorithmic}[1]
\Function {\textsc{iClaves}}{\paramIn{d}{estr} }{$\disFlecha$ res : itConj($\kappa$)} \Comment{$\Ode{clavesMax}$}
  \State res $\gets$ crearIt(e.Cclaves)
\EndFunction
\end{algorithmic}
\end{algorithm}

\newpage
\disServiciosUsados
\vspace*{-1em}
