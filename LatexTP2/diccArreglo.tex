\section{Modulo DiccArreglo(nat, significado)}

\begin{Interfaz}
  \textbf{par�metros formales}\hangindent=2\parindent\\
  \parbox{1.7cm}{\textbf{g�neros}} nat, significado\\
  
  \textbf{se explica con}: \tadNombre{dicc($clave, significado$)}, \tadNombre{Iterador Unidireccional($\alpha$)}.\\
  \indent\textbf{g�neros}: \TipoVariable{DiccArreglo(nat, significado)}, \TipoVariable{itDicc(nat)}.

  \titlex{Operaciones b�sicas del diccionario arreglo}
  
  \InterfazFuncion{Vacio}{}{DiccArreglo(nat, significado)}%
  {$res \igobs$ vacio()}%
  [$\Theta(1)$]
  [genera un nuevo diccionario arreglo.]
  
  \InterfazFuncion{Definir}{\Inout{DA}{DiccArreglo(nat, significado)}, \In{s}{significado}}{}%
  [$DA_0 \igobs DA$]
  {$DA \igobs$ definir($\#claves(DA_0) + e$, s, $DA_0$)}%
  [$\Theta(n)$]
  [Define una nueva clave con su significado. Donde e es la cantidad de elementos eliminados del diccionario 	historicamente\\
   n es la cantidad de elementos en el diccionario en el estado anterior a Definir.] 

  \InterfazFuncion{Def?}{\In{a}{nat}, \In{DA}{DiccArreglo(nat, significado)}}{bool}%
  {$res \igobs$ def?(a, DA)}%
  [$\Theta(1)$]
  [Devuelve $True$ si el la clave est� definida.]

  \InterfazFuncion{Obtener}{\In{a}{nat}, \In{DA}{DiccArreglo(nat, significado)}}{significado}%
  [Def?(a, DA)]  
  {$res \igobs$ obtener(a, DA)}%
  [$\Theta(1)$]
  [Devuelve el significado de la clave a.]
  
  \InterfazFuncion{Borrar}{\In{a}{nat}, \Inout{DA}{DiccArreglo(nat, significado)}}{}%
  [Def?(a, DA) $\land$ $DA_0 \igobs DA$]  
  {$DA \igobs$ borrar(a, $DA_0$)}%
  [$\Theta(1)$]
  [Borra el significado y su clave a.]
  
  \InterfazFuncion{Claves}{\Inout{DA}{DiccArreglo(nat, significado)}}{itDicc(nat)}%  
  {$res \igobs$ CrearItUni(toSecu(claves($DA$)))}%
  [$\Theta(1)$]
  [Devuelve el conjunto de claves del diccionario.]
  
\pagebreak  
  
  \titlex{Operaciones b�sicas del iterador}
  
  \InterfazFuncion{CrearIt}{\In{DA}{DiccArreglo(nat, significado)}}{itDicc(nat)}%  
  {$res$ $\leftarrow$ CrearItUni(toSecu(claves($DA$)))}%
  [$\Theta(1)$]
  [Crea un iterador unidireccional del conjunto de claves \\
  de forma tal que siguiente devuelva la siguiente clave del diccionario.]
    
  \InterfazFuncion{Avanzar}{\Inout{it}{itDicc(nat)}}{}%  
  [$it = it_0$ $\land$ hayMas?(it)]
  {$it \igobs$ avanzar($it_0$)}%
  [$\Theta(1)$]
  [Avanza a la posici�n siguiente del iterador.]  
  
  \InterfazFuncion{Actual}{\In{it}{itDicc(nat)}}{nat}%  
  [hayMas?(it)]
  {$res \igobs$ actual($it$)}%
  [$\Theta(1)$]
  [devuelve el elemento siguiente a la posici�n del iterador.] 
  [res es modificable si y solo si it es modificable.]  
  
  \InterfazFuncion{HayMas?}{\In{it}{itDicc(nat)}}{bool}%  
  {$res \igobs$ hayMas?($it$)}%
  [$\Theta(1)$]
  [devuelve $true$ si y solo si en el iterador quedan elementos para avanzar.]   
    
\end{Interfaz}

\subsection{Representacion}

\subsubsection{Representaci�n del diccionario arreglo}

El objetivo de este modulo es implementar un arreglo de elementos que son una tupla con un significado y un booleano.
La idea es que si el booleano vale True significa que el elemento est� borrado. 

 \begin{Estructura}{DiccArreglo(nat, significado)}[vec : vector(valor)]
  \begin{Tupla}[valor]
   \tupItem{sig}{significado}
   \tupItem{esta?}{bool}
  \end{Tupla}
 \end{Estructura}

  \Rep[vec][v]{true}\mbox{}\

  \AbsFc[vector(valor)]{dicc(nat, significado)}[vec]{\IF long($vec$) $=$ $0$ THEN vacio ELSE 
  \textbf{if} prim($vec$).esta? = $true$ \textbf{then}\\ definir(long($vec$) $-$ 1, prim(($vec$).significado, Abs(fin($vec$))) \\\textbf{else} \\
   Abs(fin($vec$)) \\\textbf{fi} FI} 

\pagebreak

\subsubsection{Representaci�n del iterador}
Este iterador recorre las claves que son los naturales en el rango del vector. Para esto mantiene una variable\\
actualizada con la posici�n siguiente. Este iterador se indefine cada vez que el vector se redefine. En ese caso deber� crearse un nuevo iterador.

 \begin{Estructura}{itDicc(nat)}[itTupla]
  \begin{Tupla}[itTupla]
   \tupItem{posicion}{nat}
   \tupItem{pVec}{puntero(vector(valor))}   
  \end{Tupla}
 \end{Estructura}
 
 Fijarse que en el rep no puedo indexar porque pVec en el mundo de TAD�s es una secuencia!.
 \Rep[itTupla][it]{$\neg$($it$.pVec  $=$  NULL) $\yluego$ \\ 
 ($it$.posicion $<$ long(*($i$.pVec)) \&\& 0 $\leq$ $it$.posicion) $\yluego$ \\ 
 buscar((*($it$.pVec)), $it$.posicion).esta? = true) 
 }\mbox{} \\

  \Abs[itTupla]{itUni(nat)}[it]{iu}{siguientes($iu$) $=$ convertir(long(*($it$.pVec))) $\land$ \\ 
   siguiente($iu$) $=$ $it$.posici�n}\\

 ~  

 \tadOperacion{convertir}{nat/n}{s : secu(nat)}{}
  \tadAxioma{convertir($n$)}{if $n = 0$ then \\ 
  $< >$ else \\
  convertir(n - 1)) o (n - 1) \\
  fi} 
  
 ~    
  
 \tadOperacion{buscar}{vector(valor)/$v$, nat/$n$}{va : valor}{n $\leq$ tam($v$)}
  \tadAxioma{buscar($v$,$n$)}{if $n = 0$ then \\ 
  prim(v) else \\
  buscar(fin(v), n-1) \\
  fi}

\pagebreak

\subsection{Algoritmos}

  \subsubsection{Algoritmos del diccionario arreglo}

  \TipoFuncion{iVacio}{}{vector(valor)} \\
  \indent\indent res $\leftarrow$ Vac�a();\\
  
  \TipoFuncion{iDefinir}{\Inout{vec}{vector(valor)}, \In{s}{significado}}{} \\
  \indent\indent\indent\indent res  $\leftarrow$ AgregarAtras(vec, s);\\
  
  \TipoFuncion{iDef?}{\In{a}{nat}, \Inout{vec}{vector(valor)}}{bool} \\
  \indent\indent res $\leftarrow$ false; \\ 
  \indent\indent if a $<$ Longitud(vec) \&\& a $>$ 0 \{ \\
  \indent\indent\indent\indent if vec[a].esta? true \{ \\ 
  \indent\indent\indent\indent\indent\indent res $\leftarrow$ true; \\
  \indent\indent\indent\indent \} \\ 
  \indent\indent\} \\

  \TipoFuncion{iObtener}{\In{a}{nat}, \Inout{vec}{vector(valor)}}{valor} \\
  \indent\indent res $\leftarrow$ vec[a].significado; \\

  \TipoFuncion{iBorrar}{\In{a}{nat}, \Inout{vec}{vector(valor)}}{} \\
  \indent\indent vec[a].esta? $\leftarrow$ false; \\ 
  
  \TipoFuncion{iClaves}{\Inout{vec}{vector(valor)}}{itDicc(nat)} \\
  \indent\indent res $\leftarrow$ iCrearIt(vec);
 
  \subsubsection{Algoritmos del iterador}
  
  \TipoFuncion{iCrearIt}{\In{vec}{vector(valor)}}{itTupla} \\
  \indent\indent res.posicion $\leftarrow$ 0; \\
  \indent\indent res.pVec $\leftarrow$ \&vec;\\
  
  \TipoFuncion{iAvanzar}{\Inout{it}{itTupla}}{} \\
  \indent\indent while iHaySiguiente(it) \&\& iObtener(i,*(it.pVec)).esta? != true \{ \\
  \indent\indent\indent\indent it.posicion++; \\
  \indent\indent \} \\
 
  \TipoFuncion{iActual}{\In{it}{itTupla}}{nat} \\
  \indent\indent res $\leftarrow$ it.posicion; \\

  \TipoFuncion{iHayMas?}{\In{it}{itTupla}}{bool} \\
  \indent\indent if it.posicion $<$ Longitud(*(it.pVec)) \{ \\
  \indent\indent\indent\indent res $\leftarrow$ true; \\
  \indent\indent \} else \{ \\
  \indent\indent\indent\indent res $\leftarrow$ false; \\
  \indent\indent \}