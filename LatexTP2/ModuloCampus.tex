% Diseño del tipo T
\newpage

% Diseño del Tipo
\disDisenio{Campus}
% La especificación
\disEspecificacion
\hspace*{\disSubSecMargen}Se usa el {\sc Tad Campus} especificado por la c\'atedra.

\disAspectosDeLaInterfaz

\disInterfaz

\disSeExplicaCon{Campus}

\disGenero{campus}

\disOperaciones{b\'asicas de Campus}

\disDeclaraFuncion{CrearCampus}{\paramIn{c}{nat}, \paramIn{f}{nat}}{res : campus}{true}{res $\igobs$ crearCampus(c,f)}{\Ode{1}}{Crea un campus de c columnas y f filas.}




\disDeclaraFuncion{Filas?}{\paramIn{c}{campus}}{res : nat}{true}{res $\igobs$ filas(c)}{\Ode{1}}{Devuelve la cantidad de filas en el campus.}

\disDeclaraFuncion{Columnas?}{\paramIn{c}{campus}}{res : nat}{true}{res $\igobs$ columnas(c)}{\Ode{1}}{Devuelve la cantidad de columnas en el campus.}

\disDeclaraFuncion{Ocupada?}{\paramIn{c}{campus}, \paramIn{p}{posicion}}{res : bool}{posValida(p,c)}{res $\igobs$ ocupada?(p,c)}{\Ode{1}}{Devuelve $true$ sii p esta ocupada por un obstaculo.}

\disDeclaraFuncion{AgregarObstaculo}{\paramInOut{c}{campus}, \paramIn{p}{posicion}}{}{c $\igobs$ $c_{0}$ $\land$ posValida(p,c) $\yluego$ $\¬$ocupada?(p,c)}{c $\igobs$ agregarObstaculo(p,$c_{0}$)}{\Ode{1}}{Devuelve $true$ sii p esta ocupada por un obstaculo.}

\disDeclaraFuncion{PosValida?}{\paramIn{c}{campus}, \paramIn{p}{posicion}}{res : bool}{true}{res $\igobs$ posValida?(p,c)}{\Ode{1}}{Devuelve $true$ sii p es parte del mapa.}

\disDeclaraFuncion{EsIngreso?}{\paramIn{c}{campus}, \paramIn{p}{posicion}}{res : bool}{true}{res $\igobs$ esIngreso?(p,c)}{\Ode{1}}{Devuelve $true$ sii p es un ingreso.}
\newpage
 
\disDeclaraFuncion{Vecinos}{\paramIn{c}{campus}, \paramIn{p}{posicion}}{res : conj(posicion)}{posValida(p,c)}{res $\igobs$ vecinos(p,c)}{\Ode{1}}{Devuelve el conjunto de posiciones vecinas a p.}

\disDeclaraFuncion{VecinosComunes}{\paramIn{c}{campus}, \paramIn{p}{posicion}, \paramIn{p2}{posicion}}{res : conj(posicion)}{posValida(p,c) $\land$ posValida(p2,c)}{res $\igobs$ vecinos(p,c)$\cap$vecinos(p2,c)}{\Ode{1}}{Devuelve el conjunto de vecinos comunes. La complejidad es \Ode{1} dado que los vecinos son a lo sumo 4, o sea, constantes}

\disDeclaraFuncion{VecinosComunes}{\paramIn{c}{campus}, \paramIn{p}{posicion}, \paramIn{p2}{posicion}}{res : conj(posicion)}{posValida(p,c) $\land$ posValida(p2,c)}{res $\igobs$ vecinos(p,c)$\cap$vecinos(p2,c)}{\Ode{1}}{Devuelve el conjunto de vecinos comunes entre dos posiciones. La complejidad es \Ode{1} dado que los vecinos son a lo sumo 4, o sea, constantes.}

\disDeclaraFuncion{ProxPosicion}{\paramIn{c}{campus}, \paramIn{dir}{direccion}, \paramIn{p}{posicion}}{res : posicion}{posValida(p,c)}{res $\igobs$ proxPosicion(p,d,c)}{\Ode{1}}{Devuelve la posicion vecina a p que esta en la direccion dir.}

\disDeclaraFuncion{IngresosMasCercanos}{\paramIn{c}{campus}, \paramIn{p}{posicion}}{res : conj(posicion)}{posValida(p,c)}{res $\igobs$ ingresosMasCercanos(p,c)}{\Ode{1}}{Devuelve el conjunto de ingresos mas cercanos a p.}




\disPautasDeImplementacion

\disEstructuraDeRepresentacion

\disSeRepresentaCon{campus}{estr}
\disDondeEs{estr}{\disTuplaEstr{nat/filas, nat/columnas, vector(vector(bool))/mapa}}

\disJustificacionDeLaEstructuraElegida

\disInvarianteDeRepresentacion
\hspace*{\disSubSubSecMargen}\textbf{\textsf{Informal}}

\hspace*{\disSubSubSecMargen}
\begin{enumerate}
\setlength{\itemindent}{3em}
  \item El mapa debe tener tantas filas como indica la estructura, lo mismo con las columnas.

\end{enumerate}

\hspace*{\disSubSubSecMargen}\textbf{\textsf{Formal}}
\onehalfspace
\disRep{estr}{e}{$true$ $\Longleftrightarrow$ 
\\\hspace*{3.75em}(1) e.filas = longitud(e.mapa) $\yluego$ 
($\forall$ i : nat)(i $\leq$ e.filas $\implies$ longitud(e.mapa[i])= e.columnas)}
\disFuncionDeAbstraccion
\vspace*{-1em}
%\hspace*{\disSubSubSecMargen}{Texto}
\disAbs{estr}{e}{campus}{c}{\Big(filas(c) = e.filas $\land$ columnas(c) = e.columnas $\yluego$ ($\forall$ p : posicion)(p.X $\leq$ e.filas $\land$
\\\hspace*{3.75em} p.Y $\leq$ e.columnas $\impluego$ ocupada?(p,c) $\Leftrightarrow$ (e.mapa[f])[c]\Big) }


%\disFuncionDeAbsFuncionesAux

\newpage

\disAlgoritmos
%\hspace*{\disSubSubSecMargen}{Texto}
% HACK: SGA 28/05/2011. Para quitar el titulo Algorithm del caption \floatname{algorithm}{}
\floatname{algorithm}{}
% WARNING: SGA 27/05/2011. La opción [H] indica a LaTex que el algoritmo lo queremos AQUI!
% Ver 4.4.1 Placement of Algorithms de algorithms.pdf.
\begin{algorithm}\phantom{[H]}
\begin{algorithmic}[1]
\Function {\textsc{$i$CrearCampus}}{\paramIn{c}{nat}, \paramIn{f}{nat}}{$\disFlecha$ res : estr} \Comment{$\Ode{f^{2}*c^{2}}$}
  \State var vector(vector(bool)) mapa $\gets$ vacia(vacia()) \Comment{$\Ode{1}$}
  \State var nat i $\gets$ 0 \Comment{$\Ode{1}$}
  \While {i$\leq$f} \Comment{$\Ode{f}$}
    \State var vector(bool) nuevo $\gets$ vacia() \Comment{$\Ode{1}$}
    \State var nat j $\gets$ 0 \Comment{$\Ode{1}$}
    \While {j$\leq$c} \Comment{$\Ode{c}$}
      \State AgregarAtras(nuevo, false) \Comment{$\Ode{c}$}
      \State j++  \Comment{$\Ode{1}$}
    \EndWhile
    \State AgregarAtras(mapa, nuevo) \Comment{$\Ode{f}$}
    \State i++ \Comment{$\Ode{1}$}
  \EndWhile
  \State res $\gets$ $<$ f, c, mapa$>$ \Comment{$\Ode{1}$}
  
\EndFunction
\end{algorithmic}
\end{algorithm}

\begin{algorithm}\phantom{[H]}
\begin{algorithmic}[1]
\Function {\textsc{$i$AgregarObstaculo}}{\paramInOut{e}{estr}, \paramIn{p}{posicion}}{$\disFlecha$ res : estr} \Comment{$\Ode{longitud(e.mapa[p.X]}$}
  \State Agregar(e.mapa[p.X], p.Y, true) \Comment{$\Ode{longitud(e.mapa[p.X])}$}
\EndFunction
\end{algorithmic}
\end{algorithm}

\begin{algorithm}\phantom{[H]}
\begin{algorithmic}[1]
\Function {\textsc{$i$Filas?}}{\paramIn{e}{estr}}{$\disFlecha$ res : nat} \Comment{$\Ode{1}$}
  \State res $\gets$ e.filas  \Comment{$\Ode{1}$}
\EndFunction
\end{algorithmic}
\end{algorithm}

\begin{algorithm}\phantom{[H]}
\begin{algorithmic}[1]
\Function {\textsc{$i$Columnas?}}{\paramIn{e}{estr}}{$\disFlecha$ res : nat} \Comment{$\Ode{1}$}
  \State res $\gets$ e.columnas  \Comment{$\Ode{1}$}
\EndFunction
\end{algorithmic}
\end{algorithm}

\begin{algorithm}\phantom{[H]}
\begin{algorithmic}[1]
\Function {\textsc{$i$Ocupada?}}{\paramIn{e}{estr}, \paramIn{p}{posicion}}{$\disFlecha$ res : bool} \Comment{$\Ode{1}$}
  \State res $\gets$ (e.mapa[p.X])[p.Y]  \Comment{$\Ode{1}$}
\EndFunction
\end{algorithmic}
\end{algorithm}

\begin{algorithm}\phantom{[H]}
\begin{algorithmic}[1]
\Function {\textsc{$i$PosValida?}}{\paramIn{e}{estr}, \paramIn{p}{posicion}}{$\disFlecha$ res : bool} \Comment{$\Ode{1}$}
  \State res $\gets$ (0 $<$ p.X) $\land$ (p.X $\leq$ e.filas) $\land$ (0 $<$ p.Y) $\land$ (p.Y $\leq$ e.columnas) \Comment{$\Ode{1}$}
\EndFunction
\end{algorithmic}
\end{algorithm}

\begin{algorithm}\phantom{[H]}
\begin{algorithmic}[1]
\Function {\textsc{$i$EsIngreso?}}{\paramIn{e}{estr}, \paramIn{p}{posicion}}{$\disFlecha$ res : bool} \Comment{$\Ode{1}$}
  \State res $\gets$ (p.Y = 1) $\lor$ (p.Y = e.filas) \Comment{$\Ode{1}$}
\EndFunction
\end{algorithmic}
\end{algorithm}

\begin{algorithm}\phantom{[H]}
\begin{algorithmic}[1]
\Function {\textsc{$i$Vecinos}}{\paramIn{e}{estr}, \paramIn{p}{posicion}}{$\disFlecha$ res : conj(posicion)} \Comment{$\Ode{1}$}
  \State var conj(posicion) nuevo $\gets$ vacio() \Comment{$\Ode{1}$}
  \State Agregar(nuevo, (p.X+1,p.Y))
  \State Agregar(nuevo, (p.X-1,p.Y))
  \State Agregar(nuevo, (p.X,p.Y+1))
  \State Agregar(nuevo, (p.X,p.Y-1))
  \State var $itConj(posicion)$ it $\gets$ crearIt(nuevo)
    \While {haySiguiente(it)} \Comment{$\Ode{c}$}
      \If {$i$PosValida?(e,siguiente(it))} \Comment{$\Ode{1}$}
        \State avanzar(it) \Comment{$\Ode{1}$}
      \Else
      \State eliminarSiguiente(it)  \Comment{$\Ode{1}$}
      \EndIf
    \EndWhile
 \State res $\gets$ nuevo \Comment{$\Ode{1}$}
\EndFunction
\end{algorithmic}
\end{algorithm}

\begin{algorithm}\phantom{[H]}
\begin{algorithmic}[1]
\Function {\textsc{$i$VecinosComunes}}{\paramIn{e}{estr}, \paramIn{p}{posicion}, \paramIn{p2}{posicion}}{$\disFlecha$ res : conj(posicion)} \Comment{$\Ode{1}$}
    \State  var conj(posicion) v $\gets$ vecinos(e,p) \Comment{$\Ode{1}$}
    \State  var conj(posicion) v2 $\gets$ vecinos(e,p2) \Comment{$\Ode{1}$} 
    \State  var conj(posicion) nuevo $\gets$ vacio() \Comment{$\Ode{1}$}
    \State var $itConj(posicion)$ it $\gets$ crearIt(v) \Comment{$\Ode{1}$}
    \While {haySiguiente(it)} \Comment{$\Ode{1}$}
      \If {Pertenece?(v2,Siguiente(it))} \Comment{$\Ode{1}$}
        \State Agregar(nuevo, Siguiente(it))	\Comment{$\Ode{1}$}
       \EndIf
       \State Avanzar(it) \Comment{$\Ode{1}$}
    \EndWhile
 \State res $\gets$ nuevo \Comment{$\Ode{1}$}    
\EndFunction
\end{algorithmic}
\end{algorithm}

\begin{algorithm}\phantom{[H]}
\begin{algorithmic}[1]
\Function {\textsc{$i$VecinosValidos}}{\paramIn{e}{estr}, \paramIn{ps}{conj(posicion)}} {$\disFlecha$ res : conj(posicion)} \Comment{$\Ode{1}$}
    \State  var conj(posicion) nuevo $\gets$ vacio() \Comment{$\Ode{1}$}
    \State var $itConj(posicion)$ it $\gets$ crearIt(ps) \Comment{$\Ode{1}$}
    \While {haySiguiente(it)} \Comment{$\Ode{1}$} 
      \If {PosValida?(e,siguiente(it))} \Comment{$\Ode{1}$}
        \State Agregar(nuevo, siguiente(it))\Comment{$\Ode{1}$}
      \EndIf
      \State avanzar(it) \Comment{$\Ode{1}$}
    \EndWhile
 \State res $\gets$ nuevo \Comment{$\Ode{1}$}    
\EndFunction
\end{algorithmic}
\end{algorithm}

\begin{algorithm}\phantom{[H]}
\begin{algorithmic}[1]
\Function {\textsc{$i$Distancia}}{\paramIn{e}{estr}, \paramIn{p}{posicion}, \paramIn{p2}{posicion}}{$\disFlecha$ res : nat} \Comment{$\Ode{1}$}
    \State res $\gets$ $\mid$p.X - p2.X$\mid$ + $\mid$p.Y - p2.Y$\mid$ \Comment{$\Ode{1}$}    
\EndFunction
\end{algorithmic}
\end{algorithm}

\begin{algorithm}\phantom{[H]}
\begin{algorithmic}[1]
\Function {\textsc{$i$ProxPosicion}}{\paramIn{e}{estr}, \paramIn{d}{direccion}, \paramIn{p}{posicion}}{$\disFlecha$ res : posicion} \Comment{$\Ode{1}$}
		  \State var posicion p2 $\gets$ p	      
	  \If {d$==$izq}
	        \State p2 $\gets$ $<$p2.X+1, p2.Y$>$   \Comment{$\Ode{1}$}
	  \Else
	      \If {d$==$der}
            \State p2 $\gets$ $<$p2.X, p2.Y$>$   \Comment{$\Ode{1}$}
          \Else
             \If {d$==$arriba} \Comment{$\Ode{1}$}
          		\State p2 $\gets$ $<$p2.X, p2.Y-1$>$ \Comment{$\Ode{1}$}
          	  \Else 
          		\State p2 $\gets$ $<$p2.X, p2.Y+1$>$ \Comment{$\Ode{1}$}
   			 \EndIf		   
		  \EndIf		
      \EndIf
      
    \State res $\gets$ p2 \Comment{$\Ode{1}$}    
\EndFunction
\end{algorithmic}
\end{algorithm}

\begin{algorithm}\phantom{[H]}
\begin{algorithmic}[1]
\Function {\textsc{$i$IngresosMasCercanos}}{\paramIn{e}{estr}, \paramIn{p}{posicion}}{$\disFlecha$ res : conj(posicion)} \Comment{$\Ode{1}$}
		  \State var conj(posicion) nuevo $\gets$ Vacio()	      \Comment{$\Ode{1}$}
	      \If {distancia(e, p, $<$p.x,1$>$) $<$ distancia(e, p, $<$p.x,e.filas$>$)} \Comment{$\Ode{1}$}
		        \State Agregar(nuevo, $<$p.x,1$>$) \Comment{$\Ode{1}$}
	      \Else
		      	\If {distancia(e, p, $<$p.x,1$>$) $>$ distancia(e, p, $<$p.x,filas(e)$>$)} \Comment{$\Ode{1}$}
        		    \State Agregar(nuevo, $<$p.x,e.filas$>$) \Comment{$\Ode{1}$}
		        \Else
        	  		\State Agregar(nuevo, $<$p.x,1$>$) \Comment{$\Ode{1}$}
					\State Agregar(nuevo, $<$p.x,e.filas$>$) \Comment{$\Ode{1}$}
				\EndIf
     	 \EndIf
    \State res $\gets$ nuevo \Comment{$\Ode{1}$}    
\EndFunction
\end{algorithmic}
\end{algorithm}




\end{algorithm}
























