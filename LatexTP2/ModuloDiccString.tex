% Dise�o del tipo T
\newpage

% Dise�o del Tipo
\disDisenio{DiccionarioString($\sigma$)}
% La especificaci�n
\disEspecificacion
\hspace*{\disSubSecMargen}Se usa el {\sc Tad Diccionario($\kappa$, $\sigma$)} especificado en el apunte de Tads b\'asicos.

\disAspectosDeLaInterfaz

\disInterfaz

\disParametrosFormales{$\kappa,\sigma$}

\disParametrosFormalesDeclaraFunc{\puntito = \puntito}{\paramIn{a_1}{\kappa}, \paramIn{a_2}{\kappa}}{res : bool}{true}{res \igobs (a_1 = a_2)}{\ThetaDe{equals(a_1, a_2)}}{funci\'on de igualdad de $\kappa$'s}
\disParametrosFormalesDeclaraFunc{COPIAR}{\paramIn{k}{\kappa}}{res : \kappa}{true}{res \igobs k}{\ThetaDe{copy(k)}}{funci\'on de copia de $\kappa$'s}
\disParametrosFormalesDeclaraFunc{COPIAR}{\paramIn{s}{\sigma}}{res : \sigma}{true}{res \igobs s}{\ThetaDe{copy(s)}}{funci\'on de copia de $\sigma$'s}

\disSeExplicaCon{Diccionario($\kappa,\sigma$), Iterador Bidireccional(Tupla($\kappa,\sigma$))}

\disGenero{diccString($\kappa$,$\sigma$)}

\disOperaciones{b\'asicas de diccionario}

\disDeclaraFuncion{Definido?}{\paramIn{d}{diccString(\kappa,\sigma)}, \paramIn{k}{\kappa}}{res : bool}{true}{res \igobs def?(d, k)}{\Ode{|k|}\ |k|\ es\ la\ longitud\ de\ la\ clave.}{Devuelve true si y s\'olo si $k$ est\'a definido en el diccionario.}

\disDeclaraFuncion{Obtener}{\paramIn{d}{diccString(\kappa,\sigma)}, \paramIn{k}{\kappa}}{res : \sigma}{def?(d, k)}{alias(res \ $\igobs$ obtener(d, k))}{\Ode{|k|}\ |k|\ es\ la\ longitud\ de\ la\ clave.}{Devuelve el significado de la clave $k$ en $d$.}
\disComentAliasing{ res no es modificable.}


\disDeclaraFuncion{Vacio}{}{res : diccString(\kappa,\sigma)}{true}{res $\igobs$ vacio()}{\Ode{1}}{Genera un diccionario vac\'io.}

\disDeclaraProc{Definir}{\paramInOut{d}{diccString(\kappa,\sigma)}, \paramIn{k}{\kappa}, \paramIn{s}{\sigma}}{d \igobs d_0}{d \igobs definir(k, s, d_0)}{\Ode{|k|}\ |k|\ es\ la\ longitud\ de\ la\ clave.}{Define la clave $k$ con el significado $s$ en el diccionario.}

\disDeclaraFuncion{Borrar}{\paramInOut{d}{diccString(\kappa,\sigma)}, \paramIn{k}{\kappa}}{res : bool}{d=$d_{0}$ \land def?(k,d)}{d \igobs borrar(k,$d_{0}$)}{\Ode{|k|}\ |k|\ es\ la\ longitud\ de\ la\ clave.}{Elimina la clave k del diccionario.}



\disOperaciones{b\'asicas del iterador}

\disDeclaraFuncion{CrearIt}{\paramIn{d}{diccString(\kappa,\sigma)}}{res : itdiccString(\kappa,\sigma)}{true}{alias(esPermutacion(SecuSuby(res),d)) \ \land vacia?(Anteriores(res))}{\Ode{n}\ n\ es\ la\ cantidad\ de\ claves.}{Crea un iterador del diccionario de forma tal que se puedan recorrer sus elementos aplicando iterativamente SIGUIENTE(no ponemos la operacion SIGUIENTE en la interfaz pues no la usamos).}

\disDeclaraFuncion{HaySiguiente}{\paramIn{it}{itdiccString(\kappa,\sigma)}}{res : bool}{true}{res $\igobs$ HaySiguiente?(it)}{\Ode{1}}{Devuelve true si y solo si en el iterador quedan elementos para avanzar.}

\disDeclaraFuncion{SiguienteSignificado}{\paramIn{it}{itdiccString(\kappa,\sigma)}}{res : \sigma}{HaySiguiente?(it)}{alias(res $\igobs$ Siguiente(it).significado)}{\Ode{1}}{Devuelve el significado del elemento siguiente del iterador.}
\disComentAliasing{ res no es modificable.}


\disDeclaraProc{Avanzar}{\paramInOut{it}{itdiccString(\kappa,\sigma)}}{it \ $\igobs$ it_{0} \ $\land$ HaySiguiente?(it)}{it \ $\igobs$ Avanzar(it_{0})}{\Ode{1}}{Avanza a la posicion siguiente del iterador.}




\disPautasDeImplementacion

\disEstructuraDeRepresentacion

\disSeRepresentaCon{diccString(\kappa,\sigma)}{puntero(nodo)}
\disDondeEs{nodo}{\disTuplaEstr{Puntero($\sigma$)/significado, arreglo[256] $de \ puntero (nodo)$/caracteres, Puntero(nodo)/padre}}

\disJustificacionDeLaEstructuraElegida

%\disEstructurasAlternativas

\disInvarianteDeRepresentacion
\hspace*{\disSubSubSecMargen}\textbf{\textsf{Informal}}

\hspace*{\disSubSubSecMargen}
\begin{itemize}
% HACK: SGA 20/06/2011. Para identar correctmente los items.
\setlength{\itemindent}{3em}
  \item Todas las posiciones del arreglo de caracteres est�n definidas.
  \item No hay claves de 0 caracteres. El significado de la ra�z es NULL.
  \item No hay ciclos en la estructura. Es decir, existe una cota superior sobre la cantidad de niveles posibles del �rbol.
  \item Dado un nodo cualquiera del trie, existe un �nico camino desde la ra�z hasta el nodo.
\end{itemize}

\hspace*{\disSubSubSecMargen}\textbf{\textsf{Formal}}

\disRep{estr}{e}{$true$ $\Longleftrightarrow$ 
\\\hspace*{3.75em}(1)(\forall i:nat)(i<256 $\implies$ definido?(e $\rightarrow$ caracteres,i)) \yluego
\\\hspace*{3.75em}(2)(e \rightarrow significado = NULL) \yluego
\\\hspace*{3.75em}(2)($\exists$ n:nat)(finaliza(e,n)) \yluego
\\\hspace*{3.75em}(3)($\forall$ p,q: puntero(nodo))(p $\in$ punteros(e) \land q \in (punteros(e) - \lbrace p\rbrace ) $\implies$ p\not=q) \yluego
\\\hspace*{3.75em}}


\disFuncionDeAbstraccion
\vspace*{-1em}
%\hspace*{\disSubSubSecMargen}{Texto}
\disAbs{roseTree(estrDato)}{r}{dicc\_trie($\sigma$)}{d}{($\forall$ $k$ : secu($letra$))(def?(k, d) $\igobs$ esta?(k, r)) $\land$ (def?(c, d) $\implies$ (obtener(k, d) $\igobs$ buscar(k, r)))}


\disFuncionDeAbsFuncionesAux


\newpage
\disAlgoritmos
%\hspace*{\disSubSubSecMargen}{Texto}
% HACK: SGA 28/05/2011. Para quitar el titulo Algorithm del caption \floatname{algorithm}{}
\floatname{algorithm}{}
% WARNING: SGA 27/05/2011. La opci�n [H] indica a LaTex que el algoritmo lo queremos AQUI!
% Ver 4.4.1 Placement of Algorithms de algorithms.pdf.
\begin{algorithm}[H]
\begin{algorithmic}[1]
\Function {\textsc{iVacio}}{ }{$\disFlecha$ res : estr} \Comment{$\Ode{1}$}
	\State var arreglo(puntero(nodo)) letras $\gets$ crearArreglo[256]
	\State \textbf{for} i $\gets$ 0 \textbf{to} 255 \textbf{do} \Comment{$\Ode{1}$}
  	\State letras[i] $\gets$ NULL \Comment{\Ode{1}}
  	\State \textbf{end for}
  	\State var nodo nuevo $\gets$ $<$NULL,letras,NULL$>$ \Comment{$\Ode{1}$}
  	\State res $\gets$ \&nuevo \Comment{$\Ode{1}$}
\EndFunction
\end{algorithmic}
\end{algorithm}

\begin{algorithm}[H]
\begin{algorithmic}[1]
\Function {\textsc{iDefinir}}{\paramInOut{d}{estr}, \paramIn{k}{string}, \paramIn{s}{\sigma}}
    \State nat $i$ $\gets$ 0 \Comment{$\Ode{1}$}
    \State puntero(nodo) actual $\gets$ d \Comment{$\Ode{1}$}
    \While {($i$ $<$ $|k|$)} \Comment{$\Ode{|k|}$}
        \If {actual $\disFlecha$ caracteres[ord(k[i])] = NULL} \Comment{$\Ode{1}$}
        	\State puntero(nodo) anterior $\gets$ actual
            \State actual $\disFlecha$ caracteres[ord(k[i])] $\gets$ $i$Vacio() \Comment{$\Ode{1}$}
            \State actual $\disFlecha$ padre $\gets$ anterior
        \Else
        \State actual $\gets$ (actual $\disFlecha$ caracteres[ord(k[i])]) \Comment{$\Ode{1}$}        
        \EndIf
        \State $i$ $\gets$ $i + 1$ \Comment{$\Ode{1}$}
    \EndWhile
        \State actual $\disFlecha$ significado $\gets$ $\&$copiar(s) \Comment{$\Ode{1}$}
\EndFunction
\end{algorithmic}
\end{algorithm}

\begin{algorithm}[H]
\begin{algorithmic}[1]
\Function {\textsc{iObtener}}{\paramIn{d}{estr}, \paramIn{k}{string}}{$\disFlecha$ res : $\sigma$}
  \State nat $i$ $\gets$ 0 \Comment{$\Ode{1}$}
  \State puntero actual $\gets$ d \Comment{$\Ode{1}$}
  \While {$i$ $<$ $|k|$} \Comment{$\Ode{|k|}$}
      \State actual $\gets$ (actual $\disFlecha$ caracteres[ord(k[i])]) \Comment{$\Ode{1}$}
      \State $i$ $\gets$ $i+1$
  \EndWhile
  \State res $\gets$ *(actual $\disFlecha$ significado)
\EndFunction
\end{algorithmic}
\end{algorithm}

\begin{algorithm}[H]
\begin{algorithmic}[1]
\Function {\textsc{iBorrar}}{\paramInOut{d}{estr}, \paramIn{k}{string}}{}
	\State puntero(nodo) actual $\gets$ d
	\State \textbf{for} i $\gets$ 0 to $|k|$ \Comment{$\Ode{1}$}
  	\State actual $\gets$ (actual\disFlecha caracteres[ord(k[i])]) \Comment{\Ode{1}}
  	\State \textbf{end for}
  	\State (actual\disFlecha significado) $\gets$ NULL
  	var puntero(nodo) camino $\gets$ NULL
  	\While {(actual\disFlecha significado = NULL) \textbf{or} todosNULL(actual\disFlecha caracteres)} \Comment{$\Ode{|k|}$}
    	\State camino $\gets$ actual \Comment{$\Ode{1}$}
    	\State actual $\gets$ (actual \disFlecha padre)
    	\State delete camino
  	\EndWhile
\EndFunction
\end{algorithmic}
\end{algorithm}


\begin{algorithm}[H]
\begin{algorithmic}[1]
\Function {\textsc{iDefinido?}}{\paramIn{d}{estr}, \paramIn{k}{string}}{$\disFlecha$ res : bool}
  \State nat $i$ $\gets$ 0 \Comment{$\Ode{1}$}
  \State puntero actual $\gets$ d \Comment{$\Ode{1}$}
  \State bool def $\gets$ \textsf{true} \Comment{$\Ode{1}$}
  \While {($i$ $<$ $|k|$ \textbf{and} def)} \Comment{$\Ode{|k|}$}
    \If {actual $\disFlecha$ caracteres[ord(k[i])] = NULL} \Comment{$\Ode{1}$}
      \State def $\gets$ \textsf{false} \Comment{$\Ode{1}$}
      \Else
        \State actual $\gets$ actual $\disFlecha$ caracteres[ord(k[i])] \Comment{$\Ode{1}$}
        \State $i$ $\gets$ $i+1$ \Comment{$\Ode{1}$}
      \EndIf
  \EndWhile
  \State res $\gets$ def $\land$ $�$(actual $\disFlecha$ significado(NULL)) \Comment{$\Ode{1}$}

\EndFunction
\end{algorithmic}
\end{algorithm}

\newpage
\disServiciosUsados
\vspace*{-1em}
 
\disRequerimientosSobreElTipo{}
\disItemRequerimientosSobreElTipo{La funci\'on \textbf{$|$x$|$} debe tener complejidad $\Ode{1}$ en el caso peor.}
\disItemRequerimientosSobreElTipo{La funci\'on \textbf{$|$x$|$} debe tener complejidad $\Ode{1}$ en el caso peor.}
\disItemRequerimientosSobreElTipo{Las operaciones deben realizarse por referencia.}
\disItemRequerimientosSobreElTipo{Debe proveer una operaci\'on \textbf{Copia} que devuelve una nueva instancia de la secuencia pero que es\\ \hspace*{5.75em}independiente de la actual, con complejidad $\Ode{n}$ en el caso peor.}
\disItemRequerimientosSobreElTipo{Debe proveer un \textbf{iterador} para avanzar que comienza en el primero elemento de la secuencia.}
\disItemRequerimientosSobreElTipo{Debe proveer un \textbf{iterador} para retroceder que comienza en el �ltimo elemento de la secuencia.}
\disItemRequerimientosSobreElTipo{Las operaciones \textbf{CrearIt, Siguiente, Anterior, TieneSiguiente, TieneAnterior} deben tener complejidad \\
\hspace*{5.75em}$\Ode{1}$ en el caso peor.}
\vspace*{1em}
\hspace*{2em}Donde $n$ es la longitud de la palabra.

