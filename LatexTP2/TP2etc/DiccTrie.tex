% Dise�o del tipo T
\newpage

% Dise�o del Tipo
\disDisenio{DiccionarioTrie($\sigma$)}
% La especificaci�n
\disEspecificacion
\hspace*{\disSubSecMargen}Se usa el {\sc Tad Diccionario($\kappa$, $\sigma$)} especificado en el apunte de Tads b\'asicos.

\disAspectosDeLaInterfaz

\disInterfaz

\disParametrosFormales{}

\disParametrosFormalesDeclaraFunc{\puntito = \puntito}{\paramIn{a_1}{\alpha}, \paramIn{a_2}{\alpha}}{res : bool}{true}{res \igobs (a_1 = a_2)}{\ThetaDe{equals(a_1, a_2)}}{funci\'on de igualdad de $\alpha$'s}
\disParametrosFormalesDeclaraFunc{COPIAR}{\paramIn{k}{\kappa}}{res : \kappa}{true}{res \igobs k}{\ThetaDe{copy(k)}}{funci\'on de copia de $\kappa$'s}
\disParametrosFormalesDeclaraFunc{COPIAR}{\paramIn{s}{\sigma}}{res : \sigma}{true}{res \igobs s}{\ThetaDe{copy(s)}}{funci\'on de copia de $\sigma$'s}

\disSeExplicaCon{Diccionario($k,\sigma$), Iterador Bidireccional(Tupla($k,\sigma$))}

\disGenero{dicc\_trie($\kappa,\sigma$)}

\disOperaciones{b\'asicas de diccionario}

\disDeclaraFuncion{Definido?}{\paramIn{d}{dicc\_trie(\kappa,\sigma)}, \paramIn{k}{string}}{res : bool}{true}{res \igobs def?(d, k)}{\Ode{|k|}}{Devuelve true si y s\'olo si $k$ est\'a definido en el diccionario.}

\disDeclaraFuncion{Obtener}{\paramIn{d}{dicc\_trie(\kappa,\sigma)}, \paramIn{k}{string}}{res : \sigma}{def?(d, k)}{esAlias(res, obtener(d, k))}{\Ode{|k|}}{Devuelve el puntero al significado de la clave $k$ en $d$.}
\disComentAliasing{ res no es modificable.}


\disDeclaraFuncion{Vacio}{}{res : dicc\_trie(\kappa,\sigma)}{true}{res $\igobs$ vacio}{\Ode{1}}{Genera un diccionario vac\'io.}

\disDeclaraProc{Definir}{\paramInOut{d}{dicc\_trie(\kappa,\sigma)}, \paramIn{k}{string}, \paramIn{s}{\sigma}}{d \igobs d_0}{d \igobs definir(k, s, d_0)}{\Ode{|k|}}{Define la clave $k$ con el significado $s$ en el diccionario.}






\disPautasDeImplementacion

\disEstructuraDeRepresentacion

\disSeRepresentaCon{dicc\_trie(\kappa,\sigma)}{puntero(nodo)}
\disDondeEs{nodo}{\disTuplaEstr{Puntero($\sigma$)/significado, arreglo[256] $de \ puntero (nodo)$/caracteres }}

\disJustificacionDeLaEstructuraElegida

%\disEstructurasAlternativas

\disInvarianteDeRepresentacion
\hspace*{\disSubSubSecMargen}\textbf{\textsf{Informal}}

\hspace*{\disSubSubSecMargen}
\begin{itemize}
% HACK: SGA 20/06/2011. Para identar correctmente los items.
\setlength{\itemindent}{3em}
  \item Todas las posiciones del arreglo de caracteres est�n definidas.
  \item No hay claves de 0 caracteres. El significado de la ra�z es NULL.
  \item No hay ciclos en la estructura. Es decir, existe una cota superior sobre la cantidad de niveles posibles del �rbol.
  \item Dado un nodo cualquiera del trie, existe un �nico camino desde la ra�z hasta el nodo.
\end{itemize}

\hspace*{\disSubSubSecMargen}\textbf{\textsf{Formal}}

\disRep{estr}{e}{$true$ $\Longleftrightarrow$ 
\\\hspace*{3.75em}(1)(\forall i:nat)(i<256 $\implies$ definido?(e $\rightarrow$ caracteres,i)) \yluego
\\\hspace*{3.75em}(2)(e \rightarrow significado = NULL) \yluego
\\\hspace*{3.75em}(2)($\exists$ n:nat)(finaliza(e,n)) \yluego
\\\hspace*{3.75em}(3)($\forall$ p,q: puntero(nodo))(p $\in$ punteros(e) \land q \in (punteros(e) - \lbrace p\rbrace ) $\implies$ p\not=q) \yluego
\\\hspace*{3.75em}}


\disFuncionDeAbstraccion
\vspace*{-1em}
%\hspace*{\disSubSubSecMargen}{Texto}
\disAbs{roseTree(estrDato)}{r}{dicc\_trie($\sigma$)}{d}{($\forall$ $k$ : secu($letra$))(def?(k, d) $\igobs$ esta?(k, r)) $\land$ (def?(c, d) $\implies$ (obtener(k, d) $\igobs$ buscar(k, r)))}


\disFuncionDeAbsFuncionesAux


\newpage
\disAlgoritmos
%\hspace*{\disSubSubSecMargen}{Texto}
% HACK: SGA 28/05/2011. Para quitar el titulo Algorithm del caption \floatname{algorithm}{}
\floatname{algorithm}{}
% WARNING: SGA 27/05/2011. La opci�n [H] indica a LaTex que el algoritmo lo queremos AQUI!
% Ver 4.4.1 Placement of Algorithms de algorithms.pdf.
\begin{algorithm}\phantom{[H]}
\begin{algorithmic}[1]
\Function {\textsc{iVacio}}{ }{$\disFlecha$ res : estr}
  \State res $\gets$ $<Significado: NULL, Caracteres: crearArreglo()>$ \Comment{$\Ode{1}$}
  \State for i $\gets$ 0 to 256 do \Comment{$\Ode{1}$}
  \State res.caracteres[i] $\gets$ NULL \Comment{\Ode{1}}
\EndFunction
\end{algorithmic}
\end{algorithm}

\begin{algorithm}\phantom{[H]}
\begin{algorithmic}[1]
\Function {\textsc{iDefinir}}{\paramInOut{d}{estr}, \paramIn{k}{string}, \paramIn{s}{\sigma}}
  \State nat $i$ $\gets$ 0 \Comment{$\Ode{1}$}
  \State puntero actual $\gets$ d \Comment{$\Ode{1}$}
  \While {($i$ $<$ $|k|$)} \Comment{$\Ode{|k|}$}
    \If {actual $\disFlecha$ caracteres[ord(k[i])] = NULL} \Comment{$\Ode{1}$}
      \State actual $\disFlecha$ caracteres[ord(k[i]) $\gets$ $i$Vacio() \Comment{$\Ode{1}$}
      \EndIf
      \State actual $\gets$ (actual $\disFlecha$ caracteres[ord(k[i])]) \Comment{$\Ode{1}$}
      \State $i$ $\gets$ $i + 1$ \Comment{$\Ode{1}$}
  \EndWhile
    \State actual $\disFlecha$ significado $\gets$ $\&$copiar(s) \Comment{$\Ode{1}$}
\EndFunction
\end{algorithmic}
\end{algorithm}

\begin{algorithm}\phantom{[H]}
\begin{algorithmic}[1]
\Function {\textsc{iObtener}}{\paramIn{d}{estr}, \paramIn{k}{string}}{$\disFlecha$ res : $\sigma$}
  \State nat $i$ $\gets$ 0 \Comment{$\Ode{1}$}
  \State puntero actual $\gets$ d \Comment{$\Ode{1}$}
  \While {$i$ $<$ $|k|$} \Comment{$\Ode{|k|}$}
      \State actual $\gets$ (actual $\disFlecha$ caracteres[ord(k[i])]) \Comment{$\Ode{1}$}
      \State $i$ $\gets$ $i+1$
  \EndWhile
  \State res $\gets$ *(actual $\disFlecha$ significado)
\EndFunction
\end{algorithmic}
\end{algorithm}


\begin{algorithm}\phantom{[H]}
\begin{algorithmic}[1]
\Function {\textsc{iDefinido?}}{\paramIn{d}{estr}, \paramIn{k}{string}}{$\disFlecha$ res : bool}
  \State nat $i$ $\gets$ 0 \Comment{$\Ode{1}$}
  \State puntero actual $\gets$ d \Comment{$\Ode{1}$}
  \State bool def $\gets$ \textsf{true} \Comment{$\Ode{1}$}
  \While {($i$ $<$ $|k|$ \textbf{and} def)} \Comment{$\Ode{|k|}$}
    \If {actual $\disFlecha$ caracteres[ord(k[i])] = NULL} \Comment{$\Ode{1}$}
      \State def $\gets$ \textsf{false} \Comment{$\Ode{1}$}
      \Else
        \State actual $\gets$ actual $\disFlecha$ caracteres[ord(k[i])] \Comment{$\Ode{1}$}
        \State $i$ $\gets$ $i+1$ \Comment{$\Ode{1}$}
      \EndIf
  \EndWhile
  \State res $\gets$ def $\land$ $�$(actual $\disFlecha$ significado(NULL)) \Comment{$\Ode{1}$}

\EndFunction
\end{algorithmic}
\end{algorithm}

\newpage
\disServiciosUsados
\vspace*{-1em}
 
\disRequerimientosSobreElTipo{Secuencia}
\disItemRequerimientosSobreElTipo{La funci\'on \textbf{$|$x$|$} debe tener complejidad $\Ode{1}$ en el caso peor.}
\disItemRequerimientosSobreElTipo{La funci\'on \textbf{$|$x$|$} debe tener complejidad $\Ode{1}$ en el caso peor.}
\disItemRequerimientosSobreElTipo{Las operaciones deben realizarse por referencia.}
\disItemRequerimientosSobreElTipo{Debe proveer una operaci\'on \textbf{Copia} que devuelve una nueva instancia de la secuencia pero que es\\ \hspace*{5.75em}independiente de la actual, con complejidad $\Ode{n}$ en el caso peor.}
\disItemRequerimientosSobreElTipo{Debe proveer un \textbf{iterador} para avanzar que comienza en el primero elemento de la secuencia.}
\disItemRequerimientosSobreElTipo{Debe proveer un \textbf{iterador} para retroceder que comienza en el �ltimo elemento de la secuencia.}
\disItemRequerimientosSobreElTipo{Las operaciones \textbf{CrearIt, Siguiente, Anterior, TieneSiguiente, TieneAnterior} deben tener complejidad \\
\hspace*{5.75em}$\Ode{1}$ en el caso peor.}
\vspace*{1em}
\hspace*{2em}Donde $n$ es la longitud de la secuencia.

\disJustificacionDeLosOrdenes

\vspace*{-1em}

\disComplejidadDeLaOperacion {Vacio}
\hspace*{\disSubSecMargen}Esta funci\'on tiene complejidad $\Ode{1}$ porque se hace una operaci\'on de $\Ode{1}$.

\disComplejidadDeLaOperacion {Definir}
\hspace*{\disSubSecMargen}Tiene complejidad $\Ode{|k|}$, ya que hacemos $|k|$ veces el ciclo while. 

\disComplejidadDeLaOperacion {Definido?}
\hspace*{\disSubSecMargen}Como la funci\'on \textsc{QuienTieneLaLetra} y \textsc{EstaLaLetra?} tienen $\Ode{1}$ esto no influye en la complejidad del ciclo.
\\\hspace*{3.5em}Como se recorre el ciclo $|k|$ veces, y las operaciones dentro del mismo son $\Ode{1}$ entonces la funci\'on tiene 
\\\hspace*{3.5em}complejidad $\Ode{|k|}$.

\disComplejidadDeLaOperacion {Obtener}
\hspace*{\disSubSecMargen}Esta funci�n va a tener complejidad $\Ode{|k|}$. Como la guarda del while se calcula en $\Ode{1}$, y adem\'as, la funci\'on
\\\hspace*{3.5em}\textsc{QuienTieneEseAlfa} es $\Ode{1}$, y lo mismo para las asignaciones que est\'an dentro del if, entonces todas las
\\\hspace*{3.5em}operaciones del while tienen complejidad $\Ode{1}$. Como este ciclo se hace |k| veces, este es la complejidad de la
\\\hspace*{3.5em}funci\'on.




