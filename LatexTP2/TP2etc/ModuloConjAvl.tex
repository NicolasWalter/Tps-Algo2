% Diseño del tipo T
\newpage

% Diseño del Tipo
\disDisenio{ConjAvl($\alpha$)}
% La especificación
\disEspecificacion
\hspace*{\disSubSecMargen}Se usa el {\sc Tad Conjunto($\alpha$)} especificado por la c\'atedra.

\disAspectosDeLaInterfaz

\disInterfaz

\disSeExplicaCon{Conjunto($\alpha$)}

\disGenero{conjAvl($\alpha$)}

\disOperaciones{b\'asicas de ConjAvl($\alpha$)}

\disDeclaraFuncion{Vacio}{}{res : conjAvl(\alpha)}{true}{res \ $\igobs$ vacio()}{\Ode{1}}{Devuelve un conjAvl($\alpha$) vacio.}

\disDeclaraProc{Agregar}{\paramInOut{c}{conjAvl(\alpha)}, \paramIn{\ a}{\alpha}}{c \ $\igobs$ c_{0}}{res \ $\igobs$ Ag(a, c_{0})}{\Ode{log(n)}}{Agrega el elemento $a$ al conjAvl $c$.}

\disDeclaraFuncion{Vacio?}{\paramIn{c}{conjAvl(\alpha)}}{res: bool}{true}{res \ $\igobs$ \emptyset ?(c)}{\Ode{}}{Devuelve true si y solo si c esta vacio.}

\disDeclaraFuncion{Pertenece?}{\paramIn{c}{conjAvl(\alpha)}, \paramIn{\ a}{\alpha}}{res: bool}{true}{res \ $\igobs$ a \in c}{\Ode{log(n)}}{Devuelve true si y solo si $a$ pertenece al conjunto.}

\disDeclaraProc{Eliminar}{\paramInOut{c}{conjAvl(\alpha)}, \paramIn{\ a}{\alpha}}{c \ $\igobs$ c_{0}}{c \ $\igobs$ c \backslash \{ a \}}{\Ode{log(n)}}{Elimina $a$ de $c$ si es que estaba.}

%\disDeclaraFuncion{\puntito = \puntito}{\paramIn{c_{1}}{conjAvl(\alpha)}, \paramIn{\ c_{2}}{conjAvl(\alpha)}}{res: bool}{true}{res \ $\igobs$ c_{1} = c_{2} }{\Ode{}}{Compara a $c_{1}$ y $c_{2}$ por igualdad.}

\disPautasDeImplementacion

\disEstructuraDeRepresentacion

\disSeRepresentaCon{conjAvl(\alpha)}{estr}
\disDondeEs{estr}{diccAvl(\alpha, bool)}

\disJustificacionDeLaEstructuraElegida
{Para entender mejor la estructura damos una explicaci\'on: \\
Representamos conjAvl($\alpha$) mediante un diccAvl($\alpha$, bool) donde los elementos del conjunto van a ser las claves del diccionario, y elegimos arbitrariamente el tipo bool para el significado, pero no nos va a interesar.
}

\disInvarianteDeRepresentacion
\hspace*{\disSubSubSecMargen}\textbf{\textsf{Informal}}
\hspace*{\disSubSubSecMargen}
\begin{enumerate}
% HACK: SGA 20/06/2011. Para identar correctmente los items.
\setlength{\itemindent}{3em}
	\item No necesitamos pedir nada en el invariante de representacion pues cualquier diccAvl($\alpha$, bool) nos sirve para representar algun conjunto.
\end{enumerate}


\hspace*{\disSubSubSecMargen}\textbf{\textsf{Formal}}
\disRep{estr}{e}{true}


\disFuncionDeAbstraccion
\vspace*{-1em}
\disAbs{estr}{e}{conjAvl(\alpha)}{c}{($\forall$ d:$\alpha$) definido?(e, d) $\Longleftrightarrow$ d $\in$ c} 


\disAlgoritmos
%\hspace*{\disSubSubSecMargen}{Texto}
% HACK: SGA 28/05/2011. Para quitar el titulo Algorithm del caption \floatname{algorithm}{}
\floatname{algorithm}{}
% WARNING: SGA 27/05/2011. La opción [H] indica a LaTex que el algoritmo lo queremos AQUI!
% Ver 4.4.1 Placement of Algorithms de algorithms.pdf.

\begin{algorithm}\phantom{[H]}
\begin{algorithmic}[1]
\Function {\textsc{$i$Vacio}}{}{$\disFlecha$ res: estr}\Comment{$\Ode{1}$}
  \State res $\gets$ vacio() \Comment{$\Ode{1}$}
\EndFunction
\end{algorithmic}
\end{algorithm}

\begin{algorithm}\phantom{[H]}
\begin{algorithmic}[1]
\Function {\textsc{$i$Agregar}}{\paramInOut{c}{estr}, \paramIn{a}{\alpha}}\Comment{$\Ode{log(n)}$}
  \State definir(c, a, true) \Comment{$\Ode{log(n)}$}
\EndFunction
\end{algorithmic}
\end{algorithm}

\begin{algorithm}\phantom{[H]}
\begin{algorithmic}[1]
\Function {\textsc{$i$Vacio?}}{\paramIn{c}{estr}}{$\disFlecha$ res: bool} \Comment{$\Ode{1}$}
  \State res $\gets$ esVacio?(claves(c)) \Comment{$\Ode{1}$}
\EndFunction
\end{algorithmic}
\end{algorithm}

\begin{algorithm}\phantom{[H]}
\begin{algorithmic}[1]
\Function {\textsc{$i$Eliminar}}{\paramInOut{c}{estr}, \paramIn{a}{\alpha}}\Comment{$\Ode{log(n)}$}
  \If {definido?(c, a)} \Comment{$\Ode{log(n)}$}
  	\State borrar(c, a) \Comment{$\Ode{log(n)}$}
  \EndIf
\EndFunction
\end{algorithmic}
\end{algorithm}

\begin{algorithm}\phantom{[H]}
\begin{algorithmic}[1]
\Function {\textsc{$i$Pertence?}}{\paramIn{c}{estr}, \paramIn{a}{\alpha}}{$\disFlecha$ res: bool}\Comment{$\Ode{log(n)}$}
  \State res $\gets$ definido?(c, a) \Comment{$\Ode{log(n)}$}
\EndFunction
\end{algorithmic}
\end{algorithm}


%\begin{algorithm}\phantom{[H]}
%\begin{algorithmic}[1]
%\Function {\textsc{\puntito $=_{i}$ \puntito}}{\paramIn{c_{1}}{estr}, \paramIn{c_{2}}{estr}}{$\disFlecha$ res: bool}
%  \State res $\gets$ claves($c_{1}$) = claves($c_{2}$) 
%\EndFunction
%\end{algorithmic}
%\end{algorithm}






