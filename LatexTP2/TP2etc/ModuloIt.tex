% Dise�o del tipo T
\newpage

% Dise�o del Tipo
\disDisenio{}
% La especificaci�n
\disEspecificacion
\hspace*{\disSubSecMargen}
\disAspectosDeLaInterfaz

\disInterfaz

\disParametrosFormales{}


\disSeExplicaCon{}

\disGenero{}

\disOperaciones{}

\disDeclaraFuncion{crearIt}{\paramIn{d}{diccTrie($\kappa,\sigma$)}}{res : itDiccTrie(\kappa,\sigma)}{true}{alias(esPermutacion(secuSuby(res),d)) \land vacia?(anteriores(res))}{\Ode{1}}{Crea un iterador bidireccional del diccionario, de forma tal que HayAnterior eval�e a false (i.e.,que se pueda recorrer los elementos aplicando iterativamente Siguiente)}
\disComentAliasing{El iterador se invalida si y s�lo si se elimina el elemento siguiente del iterador sin utilizar la funci�n EliminarSiguiente. Adem�s, anteriores(res) y siguientes(res) podr�an cambiar completamente ante cualquier operaci�n que modifique d sin utilizar las funciones del iterador.}

\disDeclaraFuncion{haySiguiente}{\paramIn{it}{itDiccTrie($\kappa,\sigma$)}}{res : bool}{true}{res \igobs haySiguiente?(it)}{\Ode{1}}{Devuelve true si y s�lo si en el iterador todav�a quedan elementos para avanzar}
\disComentAliasing{}

\disDeclaraFuncion{hayAnterior}{\paramIn{it}{itDiccTrie($\kappa,\sigma$)}}{res : bool}{true}{res \igobs hayAnterior?(it)}{\Ode{1}}{Devuelve true si y s�lo si en el iterador todav�a quedan elementos para retroceder.}
\disComentAliasing{}

\disDeclaraFuncion{siguiente}{\paramIn{it}{itDiccTrie($\kappa,\sigma$)}}{res : tupla(\kappa,\sigma)}{haySiguiente?(it)}{alias(res \igobs siguiente(it))}{\Ode{1}}{devuelve el elemento siguiente del iterador.}
\disComentAliasing{res.significado es modificable si y s�lo si it es modificable. En cambio, res.clave no es modificable.}


\disDeclaraFuncion{siguienteClave}{\paramIn{it}{itDiccTrie($\kappa,\sigma$)}}{res : \kappa}{haySiguiente?(it)}{alias(res \igobs siguiente(it).clave)}{\Ode{1}}{devuelve la clave del elemento siguiente del iterador.}
\disComentAliasing{res no es modficable.}

\disDeclaraFuncion{siguienteSignificado}{\paramIn{it}{itDiccTrie($\kappa,\sigma$)}}{res : \sigma}{haySiguiente?(it)}{alias(res \igobs siguiente(it).significado)}{\Ode{1}}{devuelve el significado del elemento siguiente del iterador.}
\disComentAliasing{res es modificable si y s�lo si it es modficable.}


\disDeclaraFuncion{anterior}{\paramIn{it}{itDiccTrie($\kappa,\sigma$)}}{res : tupla(\kappa,\sigma)}{haySiguiente?(it)}{alias(res \igobs anterior(it))}{\Ode{1}}{devuelve el elemento anterior del iterador}
\disComentAliasing{res.significado es modificable si y s�lo si it es modificable. En cambio, res.clave no es modificable.}


\disDeclaraFuncion{anteriorClave}{\paramIn{it}{itDiccTrie($\kappa,\sigma$)}}{res : \kappa}{haySiguiente?(it)}{alias(res \igobs anterior(it).clave)}{\Ode{1}}{devuelve la clave del elemento anterior del iterador.}
\disComentAliasing{res no es modficable.}

\disDeclaraFuncion{anteriorSignificado}{\paramIn{it}{itDiccTrie($\kappa,\sigma$)}}{res : \sigma}{haySiguiente?(it)}{alias(res \igobs anterior(it).significado)}{\Ode{1}}{devuelve el significado del elemento anterior del iterador.}
\disComentAliasing{res es modificable si y s�lo si it es modficable.}

\disDeclaraFuncion{avanzar}{\paramInOut{it}{itDiccTrie($\kappa,\sigma$)}}{}{it = $it_{0}$ \land haySiguiente?(it)}{it \igobs avanzar($it_{0}$)}{\Ode{1}}{avanza a la posici�n siguiente del iterador.}
\disComentAliasing{}

\disDeclaraFuncion{retroceder}{\paramInOut{it}{itDiccTrie($\kappa,\sigma$)}}{}{it = $it_{0}$ \land hayAnterior?(it)}{it \igobs retroceder($it_{0}$)}{\Ode{1}}{retrocede a la posici�n siguiente del iterador.}
\disComentAliasing{}



