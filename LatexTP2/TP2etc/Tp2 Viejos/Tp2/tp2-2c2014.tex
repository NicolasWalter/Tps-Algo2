\documentclass[a4paper,10pt]{article}
\usepackage[paper=a4paper, hmargin=1.5cm, bottom=1.5cm, top=3.5cm]{geometry}
\usepackage[latin1]{inputenc}
\usepackage[T1]{fontenc}
\usepackage[spanish]{babel}
\usepackage{xspace}
\usepackage{xargs}
\usepackage{ifthen}
\usepackage{aed2-symb,aed2-itef,aed2-tad,caratula}
\usepackage{xcolor}

\newcommand{\moduloNombre}[1]{\textbf{#1}}
\newcommand{\bigO}{\mathcal{O}}
\newcommand{\complejidad}[2]{\hfill $#1 \bigO(#2)$}

\let\NombreFuncion=\textsc
\let\TipoVariable=\texttt
\let\ModificadorArgumento=\textbf
\newcommand{\res}{$res$\xspace}
\newcommand{\tab}{\hspace*{7mm}}

\newcommandx{\TipoFuncion}[3]{%
  \NombreFuncion{#1}(#2) \ifx#3\empty\else $\to$ \res\,: \TipoVariable{#3}\fi%
}
\newcommand{\In}[2]{\ModificadorArgumento{in} \ensuremath{#1}\,: \TipoVariable{#2}\xspace}
\newcommand{\Out}[2]{\ModificadorArgumento{out} \ensuremath{#1}\,: \TipoVariable{#2}\xspace}
\newcommand{\Inout}[2]{\ModificadorArgumento{in/out} \ensuremath{#1}\,: \TipoVariable{#2}\xspace}
\newcommand{\Aplicar}[2]{\NombreFuncion{#1}(#2)}

\newlength{\IntFuncionLengthA}
\newlength{\IntFuncionLengthB}
\newlength{\IntFuncionLengthC}
%InterfazFuncion(nombre, argumentos, valor retorno, precondicion, postcondicion, complejidad, descripcion, aliasing)
\newcommandx{\InterfazFuncion}[9][4=true,6,7,8,9]{%
  \hangindent=\parindent
  \TipoFuncion{#1}{#2}{#3}\\%
  \textbf{Pre} $\equiv$ \{#4\}\\%
  \textbf{Post} $\equiv$ \{#5\}%
  \ifx#6\empty\else\\\textbf{Complejidad:} #6\fi%
  \ifx#7\empty\else\\\textbf{Descripci�n:} #7\fi%
  \ifx#8\empty\else\\\textbf{Aliasing:} #8\fi%
  \ifx#9\empty\else\\\textbf{Requiere:} #9\fi%
}

\newenvironment{Interfaz}{%
  \parskip=2ex%
  \noindent\textbf{\Large Interfaz}%
  \par%
}{}

\newenvironment{Representacion}{%
  \vspace*{2ex}%
  \noindent\textbf{\Large Representaci�n}%
  \vspace*{2ex}%
}{}

\newenvironment{Algoritmos}{%
  \vspace*{2ex}%
  \noindent\textbf{\Large Algoritmos}%
  \vspace*{2ex}%
}{}


\newcommand{\titlex}[1]{
  \vspace*{1ex}\par\noindent\textbf{\large #1}\par
}

\newenvironmentx{Estructura}[2][2={estr}]{%
  \par\vspace*{2ex}%
  \TipoVariable{#1} \textbf{se representa con} \TipoVariable{#2}%
  \par\vspace*{1ex}%
}{%
  \par\vspace*{2ex}%
}%

\newboolean{EstructuraHayItems}
\newlength{\lenTupla}
\newenvironmentx{Tupla}[1][1={estr}]{%
    \settowidth{\lenTupla}{\hspace*{3mm}donde \TipoVariable{#1} es \TipoVariable{tupla}$($}%
    \addtolength{\lenTupla}{\parindent}%
    \hspace*{3mm}donde \TipoVariable{#1} es \TipoVariable{tupla}$($%
    \begin{minipage}[t]{\linewidth-\lenTupla}%
    \setboolean{EstructuraHayItems}{false}%
}{%
    $)$%
    \end{minipage}
}

\newcommandx{\tupItem}[3][1={\ }]{%
    %\hspace*{3mm}%
    \ifthenelse{\boolean{EstructuraHayItems}}{%
        ,#1%
    }{}%
    \emph{#2}: \TipoVariable{#3}%
    \setboolean{EstructuraHayItems}{true}%
}

\newcommandx{\RepFc}[3][1={estr},2={e}]{%
  \tadOperacion{Rep}{#1}{bool}{}%
  \tadAxioma{Rep($#2$)}{#3}%
}%

\newcommandx{\Rep}[3][1={estr},2={e}]{%
  \tadOperacion{Rep}{#1}{bool}{}%
  \tadAxioma{Rep($#2$)}{true \ssi #3}%
}%

\newcommandx{\Abs}[5][1={estr},3={e}]{%
  \tadOperacion{Abs}{#1/#3}{#2}{Rep($#3$)}%
  \settominwidth{\hangindent}{Abs($#3$) \igobs #4: #2 $\mid$ }%
  \addtolength{\hangindent}{\parindent}%
  Abs($#3$) \igobs #4: #2 $\mid$ #5%
}%

\newcommandx{\AbsFc}[4][1={estr},3={e}]{%
  \tadOperacion{Abs}{#1/#3}{#2}{Rep($#3$)}%
  \tadAxioma{Abs($#3$)}{#4}%
}%


\newcommand{\DRef}{\ensuremath{\rightarrow}}

\begin{document}

% Estos comandos deben ir antes del \maketitle
\materia{Algoritmos y Estructuras de Datos II} % obligatorio
\submateria{Segundo Cuatrimestre de 2014} % opcional
\titulo{Trabajo Pr�ctico 2} % obligatorio
\subtitulo{Dise�o: Ciudad Rob�tica} % opcional
\grupo{Grupo 12} % opcional 

\integrante{Juan Ignacio Dopazo}{602/06}{dopazo.juan@gmail.com} % obligatorio 
\integrante{Rodrigo Oscar Kapobel}{695/12}{rok\_35@live.com.ar} % obligatorio 
\integrante{Nicolas Hernandez}{122/13}{nicoh22@hotmail.com} % obligatorio 
\integrante{Mart�n Baigorria}{575/14}{martin.baigorria@gmail.com} % obligatorio 

\maketitle

% compilar 2 veces para actualizar las referencias
\tableofcontents

\pagebreak
\section{Extensiones}

\subsection{Secuencia($\alpha$)}
\tadOtrasOperaciones
\tadOperacion{iesimo}{secu($\alpha$)/s, nat/i}{$\alpha$}{}
\tadAxiomas
\tadAxioma{iesimo(s,i)}{
\textbf{if} i = 0 \textbf{then} prim(s) \textbf{else} iesimo(fin(s),n-1) \textbf{fi}
}

\subsection{Cola($\alpha$)}
\tadOtrasOperaciones
\tadOperacion{in?}{$\alpha$/elem, cola($\alpha$)/c}{bool}{}
\tadAxiomas
\tadAxioma{in?(vacia)}{false}
\tadAxioma{in?(e, encolar(elem, c))}{\textbf{if} e = elem \textbf{then} true \textbf{else} in?(e, c) \textbf{fi}}

\section{M�dulo Ciudad Robotica}

\begin{Interfaz}

  \textbf{se explica con}: \tadNombre{CiudadRobotica}, \tadNombre{Iterador Bidireccional($\alpha$)}.

  \textbf{g�neros}: \TipoVariable{ciudad}, \TipoVariable{itLista($\alpha$)}.

  \textbf{usa}:  \TipoVariable{bool}, \TipoVariable{mapa}, \TipoVariable{rur}, \TipoVariable{tag}, \TipoVariable{diccTrie($\sigma$)} 
  
  \titlex{Operaciones b�sicas de ciudad}

  \InterfazFuncion{Crear}{\In{m}{mapa}}{ciudad}%
  {$res \igobs$ crear($m$)}%
  [$\bigO(1)$]
  [genera una nueva Ciudad.]
  [el mapa se pasa por referencia. la ciudad se devuelve por referencia.]

  \InterfazFuncion{Entrar}{\In{ts}{secu(tag)}, \In{e}{estacion}, \Inout{c}{ciudad}}{}
  [$c \igobs c_0 \land e \in$ estaciones($c$)]
  {$c \igobs$ entrar($ts, e, c_0$)}
  [$\bigO(|e_{m}|*S*R + N_{total})$]
  [esta funcion ingresa un robot a la ciudad. un conjunto de tags $ts$ bien formado es un diccTrie(bool), donde si el robot tiene un tag, el diccTrie devuelve true con la operacion definido?(ts, tag)]
  []

  \InterfazFuncion{Mover}{\In{u}{rur}, \In{e}{estacion}, \Inout{c}{ciudad}}{}
  [$c \igobs c_0 \land e \in$ estaciones($c$) $\land u \in $ robots($c$)]
  {$c \igobs$ mover($u, e, c_0$)}
  [$\bigO(|e| + |estacion(c,u)| + log N_{e} + log N_{estacion(c,u)})$]
  [mueve un robot entre dos estaciones.]
  []

  \InterfazFuncion{Inspeccion}{\In{e}{estacion}, \Inout{c}{ciudad}}{}
  [$c \igobs c_0 \land e \in$ estaciones($c$) $\land$ robots(e,c) $\neq$ $\emptyset$]
  {$c \igobs$ inspeccion($e, c_0$)}
  [$\bigO(|e| + log N_e)$]
  [realiza una inspeccion en una estacion. para determinar que robot se elimina en una inspeccion, se toma el que cometio mas infracciones, desempatando por su RUR.]
  []

\pagebreak

 \InterfazFuncion{ProximoRUR}{\In{c}{ciudad}}{rur}
  {$res \igobs$ proximoRUR($c$)}
  [$\bigO(1)$]
  []
  []

  \InterfazFuncion{Mapa}{\In{c}{ciudad}}{mapa}
  {$res \igobs$ mapa($c$)}
  [$\bigO(1)$]
  []
  [el mapa se devuelve por referencia.]

  \InterfazFuncion{Robots}{\In{c}{ciudad}}{itLista(robot)}
  {$res \igobs$ CrearIt(robots($c$))}
  [$\bigO(1)$]
  [devuelve un iterador al conjunto de rur.]
  []

  \InterfazFuncion{Estacion}{\In{u}{rur}, \In{c}{ciudad}}{estacion}
  [$u \in$  robots($c$)]
  {$res \igobs$ estacion($u, c$)}
  [$\bigO(1)$]
  []
  []
  
  \InterfazFuncion{Tags}{\In{u}{rur}, \In{c}{ciudad}}{secu(tag)}
  [$u \in$ robots($c$)]
  {$res \igobs$ tags($e, c_0$)}
  [$\bigO(1)$]
  []
  [El conjunto de tags se devuelve por referencia.]

  \InterfazFuncion{$\#$Infracciones}{\In{u}{rur}, \In{c}{ciudad}}{nat}
  [$u \in$ robots($c$)]
  {$c \igobs$ $\#$infracciones($u, c$)}
  [$\bigO(1)$]
  []
  []
  
  \InterfazFuncion{Estaciones}{\In{c}{ciudad}}{itConj(estaciones)}
  {$res \igobs$ CrearIt(estaciones($c$))} %CrearIt es la del modulo conjunto
  [$\bigO(1)$]
  []
  []

  \InterfazFuncion{tagMasInvolucrado}{\In{c}{ciudad}}{tag}
  [$c \igobs c_0 \land (\exists t \in tagsHistoricos(c)) \yluego \#infractoresPorTag(t,c) > 0$]
  {$tag \igobs$ tagMasInvolucrado(c)}
  [$\bigO(1)$]
  [tag mas involucrado en infracciones a lo largo de la historia de la ciudad.]
  []

  \InterfazFuncion{$\#$inspecciones}{\In{e}{estacion}, \In{c}{ciudad}}{nat}
  [$c \igobs c_0 \land e \in$ estaciones($c$)]
  {$res \igobs$ inspecciones($e, c_0$)}
  [$\bigO(|e|)$]
  []
  []
  
  \InterfazFuncion{TagsHistoricos}{\In{c}{ciudad}}{secu(tag)}  
  {$res \igobs$ tagsHistoricos($c$)}
  [$\bigO(res*ts*R)$ \\
  res es la longitud de la secuencia devuelta, ts el conjunto de todos los tags posibles existentes en los robots y
  R la cantidad total de robots en la ciudad.]
  [La secuencia se devuelve por copia.]
  []

  \InterfazFuncion{$\#$InfraccionesXtag}{\In{t}{tag}, \In{c}{ciudad}}{nat}
  [$t \in$ tagsHistoricos($c$)]  
  {$c \igobs$ $\#$infraccionesXtag($t, c$)}
  [$\bigO(1)$]
  []
  []

\end{Interfaz}

\pagebreak

\subsection{Representaci�n}

  \begin{Estructura}{ciudad}[estr]
    \begin{Tupla}[estr]
      \tupItem{robots}{lista(robot)}%
      \tupItem{\\robRUR}{arreglo\_dimensionable(itLista(robot))}%
      \tupItem{\\robEstacion}{DiccTrie(colaPrio(itLista(robot)))}%
      \tupItem{\\mapa}{mapa}%
      \tupItem{\\proximoRUR}{nat}%
      \tupItem{\\tagMasInvolucrado}{tag}%
      \tupItem{\\tagsInvolucrados}{DiccTrie(tag)}%
      \tupItem{\\historialInspeccion}{DiccTrie(nat)}%
    \end{Tupla}
   
    \begin{Tupla}[robot]
      \tupItem{infr}{nat}%
      \tupItem{\\rur}{rur}%
      \tupItem{\\activo?}{bool}
      \tupItem{\\est}{estacion}%
      \tupItem{\\tags}{lista(tag)}%
      \tupItem{\\permisos}{DiccTrie(DiccTrie(bool))}%
      \tupItem{\\itEst}{it(colaPrio(itLista(robot))}%
    \end{Tupla}  
   
  \end{Estructura}

Para representar a la ciudad y cumplir con los ordenes de complejidad requeridos, decidimos utilizar una tupla con diferentes estructuras.

En primer lugar, utilizamos una lista de robots para luego poder obtener el siguiente rur en $\bigO$(1). Es relevante notar que todas las estructuras que de alguna manera apuntan a un robot, utilizan un iterador a esta lista para no estar manipulando la estructura interna de la misma de forma directa.

Por otro lado, con el objetivo de poder acceder a cualquier robot por medio de su RUR en $\bigO$(1), utilizamos un arreglo dimensionable. Dado que los RUR son consecutivos y comienzan en 0, utilizamos el indice del arreglo para identificar un RUR.

robEstacion permite buscar el conjunto de robots en una estacion. La busqueda se hace sobre un trie, y luego el conjunto de robots se representa con un heap. Esto se debe a que para que cierre la complejidad de inspeccion, tomar el robot mas infractor debe tener orden logaritmico.

\pagebreak

\subsection{Invariante de Representacion}
\subsubsection{El invariante en lenguaje natural}

\begin{enumerate}
  \item proximoRUR es igual a la longitud de la lista de robots mas uno.
  \item En robRUR, el indice del arreglo coincide con el RUR del robot.
  \item La estacion de los robots activos coincide con la estacion de robEstacion.
  \item No existe robot que pertenezca a la cola de alguna estacion que este inactivo o no pertenezca a la lista de robots.
  \item Un robot no puede estar en mas de una estacion a la vez.
  \item Si hay algun robot inactivo, existe algun significado de historialInspeccion positivo.
  \item Si alguna vez algun robot ha cometido una infraccion en la historia de la ciudad, el tagMasInvolucrado corresponde al significado maximo de tagsInvolucrados.
\end{enumerate}
  
\subsubsection{El invariante en lenguaje formal}

  \Rep[estr][c]{
  \begin{enumerate}
  \item proximoRUR = long(c.robots) + 1
  \item ($\forall$ i: nat) i < longitud(c.robots) \impluego c.robRUR[i].rur = i
  \item ($\forall$ r: rur) esta?(r,c.robots) \impluego ($\exists$ e: estacion) in?(r,obtener(e, c.robEstacion))
  \item ($\forall$ r: rur) $\neg$($\exists$ e: estacion) \\ esta?(r,obtener(e, c.robEstacion)) $\land$ (r.activo? = false $\lor$ $\neg$esta?(r,c.robots))
  \item ($\forall$ $r$: rur) $\neg$($\exists$ $e_1, e_2$: estacion) $e_1 \neq e_2$ \\ $\implies$ in?(r,obtener($e_1$, c.robEstacion)) $\land$ in?(r,obtener($e_2$, c.robEstacion))
  \item ($\exists$ r: rur) (r.activo? = false $\land$ esta?(r, c.robots)) \yluego \\ ($\exists$ e: estacion) def?(e, c.historialInspeccion) \yluego obtener(e, historialInspeccion) $>$ 0
  \item ($\exists$ r: rur) (r.infr > 0  $\land$ esta?(r, c.robots)) \yluego  $\neg$($\exists$ t: tag) def?(t, c.tagsInvolucrados) \yluego \\ obtener(t, c.tagsInvolucrados) $>$ obtener(c.tagMasInvolucrado, c.tagsInvolucrados)
  \end{enumerate} 
  }

\subsection{Funcion de abstraccion}

  \Abs[estr]{ciudad}[c]{ciu}{
  proximoRUR(ciu) = c.proximoRUR $\land$\\
  mapa(ciu) = c.mapa $\land$\\
  robots(ciu) = claves(c \DRef robRUR) $\land$\\
  ($\forall r$: rur) r $\in$ robots(ciu) \impluego estacion(r,c) = robRUR[rur].estacion  $\land$\\
  ($\forall r$: rur) r $\in$ robots(ciu) \impluego tags(rur, ciu) = claves(robRUR[rur].tags)  $\land$\\
  ($\forall r$: rur) r $\in$ robots(ciu) \impluego \#infracciones(r, ciu) = estacion(r,c) = robRUR[rur].infr
  }

\pagebreak

\subsection{Algoritmos}

\TipoFuncion{iMover}{\Inout{c}{ciudad}, \In{e}{estacion}, \In{u}{rur}}{} \complejidad{Complejidad: }{log(n_{e1}) + |e| + log (n_e)}\\
\indent var robot $\leftarrow$ Siguiente(c.robRUR[u]) \complejidad{}{1} \\
\indent ElimSig(robot.itEst) \complejidad{}{log(n_{estacion(u, c)})} \\
\indent if ($\neg$Significado(robot.est, Significado(e, robot.permisos)))\{ \complejidad{Condicion:}{|e| + |estacion(u, c)|)}\\
\indent \indent int i $\leftarrow$ 0  \complejidad{}{1}\\
\indent \indent int inf $\leftarrow$ 0  \complejidad{}{1}\\
\indent \indent while i <  Longitud(robot.permisos)  \{ \complejidad{}{1}\\
\indent \indent \indent if definido?(iesimo(robot.permisos, i), c.tagsInvolucrados) \{  \complejidad{}{1}\\
\indent \indent \indent \indent inf $\leftarrow$ obtener(iesimo(robot.permisos, i). c.tagsInvolucrados)++  \complejidad{}{1}\\
\indent \indent \indent \} else \{ \\
\indent \indent \indent \indent definir(iesimo(robot.permisos, i), 1, c.tagsInvolucrados)  \complejidad{}{1}\\
\indent \indent \indent \indent inf $\leftarrow$ 1  \complejidad{}{1}\\
\indent \indent \indent \} \\
\indent \indent \indent if tagMasInvolucrado == '' $\lor$ obtener(c.tagMasInvolucrado. c.tagsInvolucrados) < inf \{  \complejidad{}{1}\\
\indent \indent \indent \indent tagMasInvolucrado $\leftarrow$ iesimo(robot.permisos, i)  \complejidad{}{1}\\
\indent \indent \indent \} \\
\indent \indent \} \\
\indent \indent robot.infr++ \complejidad{}{1}\\
\indent \}\\
\indent robot.itEst $\leftarrow$ encolar(c.robRUR[u], Significado(e, c.robEstacion)) \complejidad{}{|e|+ log(n_e)}\\
\indent robot.est $\leftarrow$ e\\

\textbf{Justificacion de complejidad:} Debido a que la cantidad de tags esta acotada, dado que tienen a lo sumo 64 caracteres, la complejidad es: 
$log(n_{estacion(u, c)}) + 2*|e| + |estacion(u, c)| + log(n_e)$.\\
Pero 2*|e| $\in \bigO(|e|)$. Entonces la complejidad de mover $\in \bigO(log(n_{estacion(u, c)}) + |e| + |estacion(u, c)| + log(n_e))$ .\\

\TipoFuncion{i$\#$infracciones}{\In{c}{ciudad}, \In{u}{rur}}{nat} \complejidad{Complejidad: }{1}\\
\indent res $\leftarrow$ Siguiente(c.robRUR[u]).infr \complejidad{}{1}\\

\TipoFuncion{i$\#$infraccionesXtag}{\In{c}{ciudad}, \In{t}{tag}}{nat} \complejidad{Complejidad: }{1}\\
\indent res $\leftarrow$ significado(c.tagsInvolucrados, t) \complejidad{}{1}\\

\TipoFuncion{iTagsHistoricos}{\In{c}{ciudad}}{secu(tag)} \complejidad{}{long(res)*long(tags(siguiente(itRobots)))*long(c.robots)}\\
\indent itLista(robot) itRobots $\leftarrow$ CrearIt(c.robots) \complejidad{}{1}\\
\indent res $\leftarrow$ vacia \complejidad{}{1} \\
\indent itLista(tag) itRes $\leftarrow$ CrearIt(res) \complejidad{}{1} \\
\indent while haySiguiente(itRobots) \{ \complejidad{}{long(c.robots)}\\
\indent \indent itLista(tag) itTags $\leftarrow$ CrearIt(tags(siguiente(itRobots))) \complejidad{}{1}\\
\indent \indent while haySiguiente(itTags) \{ \complejidad{}{long(tags(siguiente(itRobots)))}\\
\indent \indent \indent int i $\leftarrow$ 0 \complejidad{}{1}\\
\indent \indent \indent boolean esta $\leftarrow$ false \complejidad{}{1}\\
\indent \indent \indent while i < long(res) \&\& esta != true \{ \complejidad{}{long(res)}\\
\indent \indent \indent \indent if res[i] == siguiente(itTags) \{  \complejidad{}{1}\\
\indent \indent \indent \indent \indent esta $\leftarrow$ true  \complejidad{}{1}\\
\indent \indent \indent \indent \} \\
\indent \indent \indent \} \\
\indent \indent \indent if esta == false \{ \complejidad{}{1}\\
\indent \indent \indent \indent agregarComoSiguiente(itRes, siguiente(itTags)) \complejidad{}{1}\\
\indent \indent \indent \} \\
\indent \indent \indent avanzar(itTags)  \complejidad{}{1}\\
\indent \indent \}  \\
\indent \indent avanzar(itRobots) \complejidad{}{1} \\
\indent \} \\

\pagebreak

\TipoFuncion{iEstacion}{\In{c}{ciudad}, \In{u}{rur}}{estacion} \complejidad{Complejidad: }{1}\\
\indent res $\leftarrow$ Siguiente(c.robRUR[u]).est  \complejidad{}{1}\\

\TipoFuncion{iTags}{\In{c}{ciudad}, \In{u}{rur}}{lista(tag)} \complejidad{Complejidad: }{1}\\
\indent res $\leftarrow$ Siguiente(c.robRUR[u]).tags \complejidad{}{1}\\

\TipoFuncion{iEstaciones}{\In{c}{ciudad}}{itLista(estacion)} \complejidad{Complejidad: }{1}\\
\indent res $\leftarrow$ estaciones(c.mapa)\complejidad{}{1}\\

\TipoFuncion{iRobots}{\In{c}{ciudad}}{itDicc(rur)} \complejidad{Complejidad: }{1}\\
\indent res $\leftarrow$ CrearIt(c.robots)\complejidad{}{1}\\

\TipoFuncion{iProximoRUR}{\In{c}{ciudad}}{rur} \complejidad{Complejidad: }{1}\\
\indent res $\leftarrow$ c.proximoRUR\complejidad{}{1}\\

\TipoFuncion{iMapa}{\In{c}{ciudad}}{mapa} \complejidad{Complejidad: }{1}\\
\indent res $\leftarrow$ c.mapa \complejidad{}{1}\\

\TipoFuncion{iInspeccion}{\In{c}{ciudad}, \In{e}{estacion}}{} \complejidad{Complejidad: }{|e| + log(n_e)}\\
\indent r $\leftarrow$ Desencolar(Significado(e, c.robEstacion))\complejidad{}{|e| + log(n_e)} \\
\indent r.activo? $\leftarrow$ false \complejidad{}{1}\\
\indent if definido?(c.historialInspeccion, e) \{ \complejidad{}{|e|}\\
\indent \indent significado(c.historialInspeccion, e)++ \complejidad{}{|e|}\\
\indent \} else \{ \\
\indent \indent definir(c.historialInspeccion, e, 1) \complejidad{}{|e|}\\
\indent \} \\

\TipoFuncion{iTagMasInvolucrado}{\In{c}{ciudad}}{tag} \complejidad{Complejidad: }{1}\\
\indent res $\leftarrow$ c.tagMasInvolucrado \complejidad{}{1}\\


\TipoFuncion{i$\#$inspecciones}{\In{e}{estacion}, \In{c}{ciudad}}{nat} \complejidad{}{1}\\
\indent if def?(e, c.historialInspeccion) \{ \complejidad{}{1}\\
\indent \indent res $\leftarrow$ significado(c.historialInspeccion, e) \complejidad{}{1}\\
\indent \} else \{ \\
\indent \indent res $\leftarrow$ 0 \complejidad{}{1}\\
\indent \} \\

\pagebreak

\TipoFuncion{iEntrar}{\In{ts}{lista(tag)}, \In{c}{ciudad}, \In{u}{rur}}{} \complejidad{Complejidad:}{|e_{m}|*S*R + N_{total}}\\
\indent var im $\leftarrow$ CrearIt(c.mapa) \complejidad{}{1}\\
\indent var permisos $\leftarrow$ vacio() //DiccTrie(DiccTrie(bool)) \complejidad{}{1}\\
\indent while(HaySiguiente?(im)) \{ \complejidad{}{S}\\
\indent \indent var senda $\leftarrow$  Siguiente(im) \complejidad{}{1}\\
\indent \indent var b $\leftarrow$ verifica?(senda.restriccion, ts) \complejidad{}{R}\\
\indent \indent definir(permisos, senda.$e_{1}$, vacio) \complejidad{}{|e_m|}\\
\indent \indent definir(significado(permisos, senda.$e_{1}$), $e_{2}$, b) \complejidad{}{|e_m|}\\
\indent \indent //Definir permutacion del permiso. \\
\indent \indent definir(permisos, senda.$e_{2}$, vacio)\} \complejidad{}{|e_m|}\\
\indent \indent definir(significado(permisos, senda.$e_{2}$), $e_{1}$, b) \complejidad{}{|e_m|}\\
\indent \indent Avanzar(im) \complejidad{}{1}\\
\indent \}\\
\indent var rob $\leftarrow$ < 0, c.proximoRUR, true, e, ts, permisos, NULL > \complejidad{}{1}\\
\indent itRob $\leftarrow$ AgregarAdelante(c.robots, rob) \complejidad{}{1}\\
\indent var Nuevo $\leftarrow$ CrearArreglo(c.proximoRUR) \complejidad{}{1}\\
\indent var i $\leftarrow$ 0 \complejidad{}{1}\\
\indent while (i $<$ c.proximoRUR -1) \{ \complejidad{}{N_{total}}\\
\indent \indent Nuevo[i] $\leftarrow$ c.robRUR[i] \complejidad{}{1}\\ 
\indent \indent i++ \complejidad{}{1}\\
\indent \}\\
\indent c.robRUR[rob.rur] $\leftarrow$ itRob \complejidad{}{1}\\
\indent c.robRUR $\leftarrow$ Nuevo \complejidad{}{1}\\
\indent var it $\leftarrow$ encolar(Significado(c.robEstacion, e), itRob) \complejidad{}{1}\\
\indent Siguiente(itRob).itEst $\leftarrow$ it \complejidad{}{1}\\
\indent c.proximoRUR++  \complejidad{}{1}\\

\textbf{Justificacion de complejidad:} Al entrar un robot, en primer lugar debemos armar su 'matriz' de permisos. Eso se hace con un DiccTrie(DiccTrie(bool)). El objetivo de esto es que luego sea mas facil mover a un robot, para saber si comete una infraccion o no al pasar de una estacion a otra. Para ver si un robot puede pasar de la estacion 1 a la estacion 2, primero se busca en el primer diccionario la estacion 1, y luego la estacion 2 para tener como resultado un bool. Esto tiene costo S*(R + 4|$e_m$|) que esta en $\bigO$(|$e_m$| * S * R).
Luego debemos agregar el robot al comienzo de la lista de robots, que esta en $\bigO$(1), y a su vez hay que agregar el robot al arreglo dimensionable, que esta en $\bigO$($N_{total}$).

\pagebreak

\subsection{Servicios Usados}

\TipoVariable{Lista Enlazada($\alpha$)}:
\begin{itemize}
	\item Siguiente(itLista(robot)) en $\bigO(1)$
	\item AgregarAdelante(lista(robot), robot) en $\bigO(1)$
\end{itemize}

\TipoVariable{colaPrio(itLista(robot))}:
\begin{itemize}
	\item encolar(colaPrio c, itLista it) en $\bigO(log(n))$ donde n es la cantidad de elementos en la cola
	\item desencolar(colaPrio c) en $\bigO(log(n))$ donde n es la cantidad de elementos en la cola
	\item ElimSig(itCola) en $\bigO(log(n))$ donde n es la cantidad de elementos en la cola iterada
\end{itemize}

\TipoVariable{mapa}:
\begin{itemize}
	\item Estaciones(mapa) en $\bigO(1)$
	\item CrearIt(mapa) en $\bigO(1)$
	\item Siguiente(itMapa) en $\bigO(1)$
	\item HaySiguiente(itMapa) en $\bigO(1)$
	\item Avanzar(itMapa) en $\bigO(1)$
\end{itemize}

\TipoVariable{DiccTrie($\alpha$)}:
\begin{itemize}
	\item Definir(DiccTrie, string s, $\alpha$) en $\bigO(|s|)$ donde |s| es la longitud del string 
	\item Significado(DiccTrie, string) en $\bigO(|s|)$ donde |s| es la longitud del string
	\item Definido?(DiccTrie, string) en $\bigO(|s|)$ donde |s| es la longitud del string 
\end{itemize}

\TipoVariable{Restriccion}:
\begin{itemize}
	\item Verifica?(restriccion, ListaEnlazada(tags)) en $\bigO(R)$ siendo R la cantidad de nodos del arbol restriccion.
\end{itemize}


\pagebreak
\section{Modulo Mapa}

\begin{Interfaz}

  \textbf{se explica con}: \tadNombre{Mapa}, \tadNombre{Iterador Bidireccional($senda$)}.

  \textbf{g�neros}: \TipoVariable{mapa}, \TipoVariable{itMapa}.

  \textbf{usa}: \TipoVariable{restriccion}, \TipoVariable{estacion}, \TipoVariable{lista($\alpha$)}
  
  \titlex{Operaciones b�sicas del mapa}
  
  \InterfazFuncion{Vacio}{}{mapa}%
  {$res \igobs$ vacio}%
  [$\bigO(1)$]
  [genera un nuevo mapa vacio.]
  
  \InterfazFuncion{Agregar}{estacion $e$, mapa $m$}{mapa}%
  [$e$ $\not\in$ estaciones($m$)]  
  {$res \igobs$ agregar(e,m)}%
  [$\bigO(1)$]
  [Agrega una nueva estacion al mapa.]
  [La estacion se agrega por copia.]

  \InterfazFuncion{Conectar}{estacion $e_1$, estacion $e_2$, restriccion $r$, mapa m}{mapa}%
  [\{e1, e2\} $\subseteq$ estaciones(m) $\land_L$ $\neg$(conectadas?($e_1$, $e_1$, $m$))]  
  {$res \igobs$ conectar($e_1$, $e_2$, $r$, $m$)}%
  [$\bigO(1)$]
  [Conecta dos estaciones del mapa y genera una nueva senda.]
  [Se copian los punteros]

  \InterfazFuncion{Estaciones}{mapa $m$}{itLista(estacion)}%
  {$res \igobs$ CrearItBi(toSecu(estaciones($m$)))}%
  [$\bigO(1)$]
  [Devuelve el iterador a las estaciones del mapa.]
  [alias(res, siguientes(itLista(estacion)))]

  \InterfazFuncion{Conectadas?}{estacion $e_1$, estacion $e_2$, mapa $m$}{bool}%
  [\{e1, e2\} $\subseteq$ estaciones(m)]
  {$res \igobs$ Conectadas?($e_1$, $e_2$, $m$)}%
  [$\bigO(S)$ \\
  Donde S es la cantidad de sendas actuales]
  [Devuelve $true$ si y solo si las estaciones estan conectadas.]
  
  \InterfazFuncion{Restriccion}{estacion $e_1$, estacion $e_2$, mapa $m$}{restriccion}%
  [\{e1, e2\} $\subseteq$ estaciones(m) $\land_L$ concetadas?($e_1$, $e_2$, $m$)]
  {$res \igobs$ Restriccion($e_1$, $e_2$, $m$)}%
  [$\bigO(S)$ \\
  Donde S es la cantidad de sendas actuales]
  [Devuelve la restriccion de la senda especificada.]
  
\pagebreak  
  
  \titlex{Operaciones b�sicas del iterador} 
  
  \InterfazFuncion{CrearIt}{\In{m}{mapa}}{itMapa}%  
  {$res$ $\leftarrow$ CrearItBi(secuSendas($m$, toSecu(estaciones($m$))))}%
  [$\bigO(1)$]
  [Crea un iterador bidireccional a las sendas del mapa \\
  alias(res,siguientes(itMapa)).]
    
  \InterfazFuncion{Avanzar}{\Inout{it}{itMapa(senda)}}{}%  
  [$it = it_0$ $\land$ haySiguiente?(it)]
  {$it \igobs$ avanzar($it_0$)}%
  [$\bigO(1)$]
  [Avanza a la posici�n siguiente del iterador.]  
  
  \InterfazFuncion{Siguiente}{\In{it}{itMapa}}{senda}%  
  [haySiguiente?(it)]
  {$res \igobs$ siguiente($it$)}%
  [$\bigO(1)$]
  [devuelve el elemento siguiente a la posici�n del iterador.] 
  [res es modificable si y solo si it es modificable.]  
  
  \InterfazFuncion{HaySiguiente?}{\In{it}{itMapa}}{bool}%  
  {$res \igobs$ haySiguiente?($it$)}%
  [$\bigO(1)$]
  [devuelve $true$ si y solo si en el iterador quedan elementos para avanzar.] 
  
\end{Interfaz}

\subsection{Representacion del Mapa}

 \subsubsection{Representacion}
   
 \begin{Estructura}{Mapa}[vec]
  \begin{Tupla}[vec]
   \tupItem{sendas}{lista(senda)}
   \tupItem{estaciones}{lista(estacion)}
  \end{Tupla}
 \end{Estructura}
 \begin{Estructura}{senda}[s]
  \begin{Tupla}[s]
  \tupItem{$e_1$}{estacion}
  \tupItem{$e_2$}{estacion}
  \tupItem{$r$}{restriccion}
  \end{Tupla}
 \end{Estructura}
 
 \subsubsection{Invariante de Representacion}
 
Se permite que las estaciones esten conectadas a si mismas dado que el TAD lo permite. \\
 
 \Rep[vec][v]{if long(vec.sendas) $>$ 0 then \\
 	noHaySendasRepetidas(vec.sendas, vec.sendas)
 	fi $\land$ \\
 	noHayEstacionesRepetidas(vec.estaciones, vec.estaciones) $\land$ \\
 ($\forall$ $e_1$: estacion) esta?(dameEstaciones(vec.sendas), $e_1$) $\Rightarrow_L$ \\ 
 ($\exists$ $e_2$: estacion) esta?(vec.estaciones, $e_2$) $\land_L$ $e_1$ = $e_2$)}\mbox{}\

\subsection{Funccion de abstraccion del Mapa.}

  \AbsFc[vec]{mapa}[v]{esPermutacion(v.estaciones, estaciones(mapa)) $\land_L$ \\ 
  ($\forall$ estacion $e_1$, $e_2$) esSenda($e_1$, $e_2$, v.sendas) $\land_L$ conectadas($e_1$, $e_2$, $m$) $\land$ \\ restriccion($e_1$, $e_2$, $m$) = dameRestriccion($e_1$, $e_2$, v.sendas)
   } 

 ~  

 \tadOperacion{secuSendas}{$mapa/m$, $secu(estacion)/se$}{s : secu(<estacion $e_1$, estacion $e_2$,  restriccion $r$>)}{}
  \tadAxioma{secuSendas($m$, $se$)}{
	if long(se) > 0 then \\
		conectadasA(prim(se),fin(se), m) \& secuSendas($m$,fin(se))\\
    else $< >$ fi
  }
 
 ~  

 \tadOperacion{conectadasA}{$estacion/e$, $secu(estacion)/se$, $mapa/m$}{s : secu(<estacion $e_1$, estacion $e_2$,  restriccion $r$>)}{}
  \tadAxioma{conectadasA($e$, $se$, $m$)}{
	if long(se) > 0 then \\
		if conectadas?(prim(se), e, m) then \\
		<prim(se), e, restriccion(prim(se), e, m)> fi \argumento conectadasA(e, fin(se), m)
    else $< >$ fi
  }

 ~

 \tadOperacion{toSecu}{$conj(estacion)/ce$}{s : secu(estacion)}{}
  \tadAxioma{toSecu($ce$)}{
	if \#(ce) > 0 then \\
		dameUno(ce) \argumento toSecu(sinUno(ce))\\
    else $< >$ fi
  }
 
 ~ 

 \tadOperacion{dameRestriccion}{$estacion/e_1$, $estacion/e_2$, secu(senda)/cn}{r : restriccion}{}
  \tadAxioma{dameRestriccion($e_1$, $e_2$, $cn$)}{if $long(cn) \not= 0$ then \\ 
  if ((prim(cn)).$e_1$ = $e_1$ $\land$ (prim(cn)).$e_2$ = $e_2$) $\vee$ \\
  ((prim(cn)).$e_2$ = $e_1$ $\land$ (prim(cn)).$e_1$ = $e_2$) then \\
   (prim(cn)).r
  else \\
   dameRestriccion($e_1$, $e_2$, fin($cn$))
  fi \\
  fi}
 
 ~  

 \tadOperacion{esSenda}{$estacion/e_1$, $estacion/e_2$, secu(senda)/cn}{b : bool}{}
  \tadAxioma{esSenda($e_1$, $e_2$, $cn$)}{if $long(cn) \not= 0$ then \\ 
  if ((prim(cn)).$e_1$ = $e_1$ $\land$ (prim(cn)).$e_2$ = $e_2$) $\vee$ \\
  ((prim(cn)).$e_2$ = $e_1$ $\land$ (prim(cn)).$e_1$ = $e_2$) then \\
   true
  else \\
   esSenda($e_1$, $e_2$, fin($cn$))
  fi \\
  fi}
 
 ~  

 \tadOperacion{dameEstaciones}{secu(senda))/cn}{s : secu(estacion)}{}
  \tadAxioma{dameEstaciones($cn$)}{if $long(cn) = 0$ then \\ 
  $< >$ else \\
  (prim(cn).$e_1$ \argumento (prim(cn).$e_2$ \argumento dameEstaciones(fin($cn$)) \\
  fi}
 
 ~   
  
  \tadOperacion{noHayEstacionesRepetidas}{$secu(estacion)/cn_1$, $secu(estacion)/cn_2$}{b : bool}{}
  \tadAxioma{noHayEstacionesRepetidas($cn_1$, $cn_2$)}{
  if long($cn_1$) = 0 then true else \\  
  if aparicionesEstacion($cn_1$, prim($cn_2$)) $\not=$ 1 then \\ 
   false else \\
    noHayEstacionesRepetidas(fin($cn_2$), $cn_1$)\\
  fi \\ fi}

 ~  

  \tadOperacion{noHaySendasRepetidas}{$secu(senda)/cn_1$, $secu(senda)/cn_2$}{b : bool}{}
  \tadAxioma{noHaySendasRepetidas($cn_1$, $cn_2$)}{
  if long($cn_1$) = 0 then true else \\  
  if aparicionesSenda($cn_1$, prim($cn_2$)) $\not=$ 1 then \\ 
   false else \\
    noHaySendasRepetidas(fin($cn_2$), $cn_1$)\\
  fi \\ fi}

 ~  

 \tadOperacion{aparicionesEstacion}{secu(estacion)/cn , estacion/e}{n : nat}{}
  \tadAxioma{aparicionesEstacion($cn$, e)}{
  if long(cn) $\not=$ 0 then \\  
  if esta?(cn, e) then \\ 
   1 + aparicionesEstacion(fin(cn), e) else \\
   aparicionesEstacion(fin(cn), e) \\
  fi \\
  fi}

 ~    
  
 \tadOperacion{aparicionesSenda}{secu(senda)/cn , senda/s}{n : nat}{}
  \tadAxioma{aparicionesSenda($cn$, $s$)}{
  if long(cn) $\not=$ 0 then \\  
  if (s.$e_1$ \igobs (prim(senda)).$e_1$ $\land$ s.$e_2$ \igobs (prim(senda)).$e_2$)  $\vee$ \\
     (s.$e_2$ \igobs (prim(senda)).$e_1$ $\land$ s.$e_1$ \igobs (prim(senda)).$e_2$) then \\ 
   1 + aparicionesSenda(fin(cn), e) else \\
   aparicionesSenda(fin(cn), e) \\
  fi \\
  fi}
  
\subsection{Representacion del Iterardor}  
  
\subsubsection{Representacion.}

\begin{Estructura}{itMapa}[im]
  \begin{Tupla}[im]
   \tupItem{lista}{itLista(senda)}
  \end{Tupla}
 \end{Estructura}
 
\subsubsection{Invariante de Representacion}
 
 \Rep[im][im]{true} 

\subsection{Funcion de abstraccion del Iterador}

 \AbsFc[im]{itBi(senda)}[v]{it $\|$ Anteriores(im.lista) \igobs Anteriores(it) $\land$\\
 Siguientes(im.lista) \igobs Siguientes(it) }
 
\pagebreak

\subsection{Algoritmos}

  \subsubsection{Algoritmos del Mapa}
  
  ~ 
  
  \TipoFuncion{iVacio}{}{vec v} \complejidad{Complejidad: }{1}\\
  \indent\indent v.sendas $\leftarrow$ Vacia() \complejidad{}{1}\\ 
  \indent\indent v.estaciones $\leftarrow$ Vacia() \complejidad{}{1}\\
  \indent\indent res $\leftarrow$ v  

  ~  
  
  \TipoFuncion{iAgregar}{estacion e,vec v}{vec}  \complejidad{Complejidad: }{1}  \\
  \indent\indent AgregarAdelante(v.estaciones, e) \complejidad{}{1}\\
  \indent\indent res $\leftarrow$ v \complejidad{}{1}\\ 
  
  ~  
  
  \TipoFuncion{iConectar}{estacion a, estacion b, restriccion r, vec v}{vec} \complejidad{Complejidad: }{1}\\ 
  \indent\indent senda.$e_1$ $\leftarrow$ a  \complejidad{}{1}\\ 
  \indent\indent senda.$e_2$ $\leftarrow$ b  \complejidad{}{1}\\ 
  \indent\indent senda.$r$ $\leftarrow$ r \complejidad{}{1}\\ 
  \indent\indent AgregarAdelante(v.sendas, senda) \complejidad{}{1}\\ 
  \indent\indent res $\leftarrow$ v \complejidad{}{1}
  
  ~  
  
  \TipoFuncion{iEstaciones}{vec v}{itLista(estacion)} \complejidad{Complejidad: }{1}\\ 
  \indent\indent res $\leftarrow$ crearItBi(vec.estaciones) \complejidad{}{1}\\ 
  
  ~

  \TipoFuncion{iConectadas?}{estacion a, estacion b, vec v}{bool} \complejidad{Complejidad: }{S}\\ 
  \indent\indent itLista(senda) it $\leftarrow$ CrearIt(v.sendas) \complejidad{}{1}\\ 
  \indent\indent bool conectadas $\leftarrow$ false \complejidad{}{1}\\ 
  \indent\indent while haySiguiente(it) \{ \complejidad{}{S}\\ 
  \indent\indent\indent senda aux $\leftarrow$ Siguiente(it) \complejidad{}{1}\\ 
  \indent\indent\indent if (aux.e1 = a \& aux.e2 = b) || (aux.e1 = b \& aux.e2 = a) \complejidad{}{1}\\ 
   \indent\indent\indent\indent conectadas $\leftarrow$ true \complejidad{}{1}\\ 
   \indent\indent\indent \} \complejidad{}{1}\\ 
   \indent\indent\indent Avanzar(it) \complejidad{}{1}\\ 
   \indent\indent \} \\
   \indent\indent res $\leftarrow$ conectadas \complejidad{}{1}
   
  ~
  
  \TipoFuncion{iRestriccion}{estacion a, estacion b, vec v}{restriccion} \complejidad{Complejidad: }{S}\\ 
  \indent\indent itLista(senda) it $\leftarrow$ CrearIt(v.sendas) \complejidad{}{1}\\ 
  \indent\indent restriccion restr \complejidad{}{1}\\ 
  \indent\indent while haySiguiente(it) \{ \complejidad{}{S}\\ 
  \indent\indent\indent senda aux $\leftarrow$ Siguiente(it) \complejidad{}{1}\\ 
  \indent\indent\indent if (aux.e1 = a \& aux.e2 = b) || (aux.e1 = b \& aux.e2 = a) \complejidad{}{1}\\ 
   \indent\indent\indent\indent restr $\leftarrow$ aux.r \complejidad{}{1}\\ 
   \indent\indent\indent \} \complejidad{}{1}\\ 
   \indent\indent\indent Avanzar(it) \complejidad{}{1}\\ 
   \indent\indent \} \\
   \indent\indent res $\leftarrow$ restr \complejidad{}{1}
  
\subsubsection{Algoritmos del Iterador}

\TipoFuncion{iCrearIt}{mapa m}{itMapa} \complejidad{}{1}\\
\indent\indent return CrearIt(m.sendas)

\TipoFuncion{iAvanzar}{itMapa m}{} \complejidad{}{1}\\
\indent\indent Avanzar(m.lista)

\TipoFuncion{iSiguiente}{itMapa m}{} \complejidad{}{1}\\
\indent\indent Siguiente(m.lista)

\TipoFuncion{iHaySiguiente?}{itMapa m}{bool} \complejidad{}{1}\\
\indent\indent HaySiguiente?(m.lista)

\subsection{Servicios Usados}

\TipoVariable{Lista Enlazada($\alpha$)}:
\begin{itemize}
	\item Vacia() en $\bigO(1)$
	\item Siguiente(itLista($\alpha$)) en $\bigO(1)$
	\item AgregarAdelante(lista($\alpha$), $\alpha$) en $\bigO(1)$
	\item CrearIt en $\bigO(1)$
	\item HaySiguiente(itLista($\alpha$)) en $\bigO(1)$
	\item Avanzar(itLista($\alpha$)) en $\bigO(1)$
\end{itemize}
%\pagebreak
%\section{M�dulo Robot}

\begin{Interfaz}
  
  \textbf{par�metros formales}\hangindent=2\parindent\\
  \parbox{1.7cm}{\textbf{g�neros}} $robot$\\
  \textbf{se explica con}: \tadNombre{Robot}

  \textbf{g�nero}: \TipoVariable{robot}

  \textbf{usa}:  \TipoVariable{bool}, \TipoVariable{estacion}, \TipoVariable{rur}, \TipoVariable{diccTrie($\sigma$)} 

  \titlex{Operaciones b�sicas del robot}

  \InterfazFuncion{crear}{\In{rur}{rur}, \In{est}{estacion}, \In{infr}{nat}, \In{tags}{DiccTrie(bool)}, \In{perm}{DiccTrie(DiccTrie(bool))}}{robot}
  [true]
  {$res = crear(rur,est,infr,tags,perm)$}
  [$\bigO(1)$]
  []
  [devuelve un nuevo robot por referencia.]

  \InterfazFuncion{$\bullet > \bullet$}{\In{r_1}{robot}, \In{r_2}{robot}}{bool}
  [true]
  {$res = esMayor?(r_1,r_2)$}
  [$\bigO(1)$]
  [devuelve true si $r_1$ es 'mayor' a $r_2$, definido por numero de infracciones y luego por RUR.]
  [devuelve al robot por referencia.]

  \InterfazFuncion{activo?}{\In{r}{robot}}{bool}
  [true]
  {$res = activo(r)$}
  [$\bigO(1)$]
  [bool que muestrar si el robot esta actualmente activo]
  []

  \InterfazFuncion{estacion}{\In{r}{robot}}{estacion}
  [true]
  {$res = estacion(r)$}
  [$\bigO(1)$]
  [devuelve string que representa la estacion actual donde se encuentrar el robot.]
  []

  \InterfazFuncion{infracciones}{\In{r}{robot}}{nat}
  [true]
  {$res = infracciones(r)$}
  [$\bigO(1)$]
  []
  []

  \InterfazFuncion{permisos}{\In{r}{robot}}{DiccTrie(bool)}
  [true]
  {$res = permisos(r)$}
  [$\bigO(1)$]
  [devuelve la referencia al conj de tags en un DiccTrie.]
  []

  \InterfazFuncion{tieneTag?}{\In{r}{robot}, \In{tag}{string}}{bool}
  [true]
  {$res = tieneTag?(r)$}
  [$\bigO(|tag|)$]
  [devuelve true si el robot tiene cierto tag.]
  []

\end{Interfaz}

\pagebreak

\subsection{Representacion}
  
  \titlex{Representaci�n del Robot}

  \begin{Estructura}{robot}[datosRobot]
    \begin{Tupla}[datosRobot]
      \tupItem{rur}{rur}%
      \tupItem{\\presente?}{bool}%
      \tupItem{\\est}{estacion}%
      \tupItem{\\infr}{nat}%
      \tupItem{\\tags}{DiccTrie(bool)}%
      \tupItem{\\permisos}{DiccTrie(DiccTrie(bool)}%
    \end{Tupla}
  \end{Estructura}

Representamos a un robot con una tupla debido a que el modulo en si es necesario para definir la relacion de orden entre robots. Esta relacion es luego utilizada por el modulo cola de prioridad para definir que robot sale de circulacion en una inspeccion. Para evitar tener que hacer el modulo de conjunto, para los tags utilizamos un DiccTrie(bool).

\subsection{Invariante de Representacion}
\subsubsection{El invariante en lenguaje natural.}

\begin{enumerate}
  \item Un robot no tiene tags repetidos.
  \item Los tags tienen menos de 64 caracteres.
\end{enumerate}
  
  \subsection{Funcion de abstraccion}

  \Abs[estr]{robot}[r]{robot}{rur(r) = rur(robot) $\land$ presente?(r) = presente?(robot) $\land$ estacion(r) = estacion(robot) $\land$ infr(r) = infr(robot) $\land$ ($\forall t$: tag) tag $\in$ tags(r) $\iff$ tieneTag?(robot, tag)}

\subsection{Algoritmos}

Dado que los getters son triviales, solo voy a hacer el pseudocodigo de la relacion de orden y tieneTag?.

~

\TipoFuncion{$\bullet > \bullet$}{\In{r_1}{robot}, \In{r_2}{robot}}{bool} \complejidad{Complejidad:}{1} \\
\indent if infracciones($r_1$) $>$ infracciones($r_2$) \{ \complejidad{}{1} \\
\indent \indent res $\leftarrow$ $true$ \complejidad{}{1} \\
\indent \} else \{ \\
\indent \indent if infracciones($r_2$) > infracciones($r_1$) \{ \complejidad{}{1} \\
\indent \indent \indent res $\leftarrow$ $false$ \complejidad{}{1} \\
\indent \indent \} else \{ \\
\indent \indent \indent if rur($r_1$) $>$ rur($r_2$) \{ \complejidad{}{1} \\
\indent \indent \indent \indent res $\leftarrow$ $true$ \complejidad{}{1} \\
\indent \indent \indent \} else \{ \\
\indent \indent \indent \indent res $\leftarrow$ $false$ \complejidad{}{1} \\
\indent \indent \indent \} \\
\indent \indent \} \\
\indent \} \\

~

\TipoFuncion{tieneTag?}{\In{r}{robot}, \In{tag}{string}}{bool} \complejidad{Complejidad:}{|tag|} \\
\indent res $\leftarrow$ definido?(robot \DRef tags, tag) \complejidad{}{|tag|}
\pagebreak
\section{Modulo Restriccion}

\begin{Interfaz}

  \textbf{se explica con}: \tadNombre{Restriccion}

  \textbf{g�neros}: \TipoVariable{restriccion}

  \textbf{usa}: \TipoVariable{puntero($\alpha$)}, \TipoVariable{tag}, \TipoVariable{bool}


  \InterfazFuncion{NuevoTag}{\In{t}{tag}}{restriccion}%
  {$res \igobs$ <t>}%
  [$\bigO(1)$]
  [genera una restriccion de un solo tag.]

  \InterfazFuncion{And}{\In{r1}{restriccion}, \In{r2}{restriccion}}{restriccion}%
  {$res \igobs$ $r1$ AND $r2$}%
  [$\bigO(1)$]
  []
  []

  \InterfazFuncion{Or}{\In{r1}{restriccion}, \In{r2}{restriccion}}{restriccion}%
  {$res \igobs$ $r1$ OR $r2$}%
  [$\bigO(1)$]
  []
  []
  
  \InterfazFuncion{Not}{\Inout{r}{restriccion}}{}% este capaz que es inout
  {$res \igobs$ NOT $r$}%
  [$\bigO(1)$]
  []
  []  
  
  \InterfazFuncion{Verifica?}{\In{ts}{secu(tag)}, \In{r}{restriccion}}{bool}%
  {$res \igobs$ verifica?(ts, r)}%
  [$\bigO(R)$]
  [R es el tama�o de la restriccion mas grande vigente en las sendas de la ciudad.]
  []  

\end{Interfaz}

\subsection{Representacion}
 
 \begin{Estructura}{restriccion}[puntero(nodo)]
  \begin{Tupla}[nodo]
   \tupItem{tag}{tag}
   \tupItem{\\tipo}{log}
   \tupItem{\\izq}{puntero(nodo)}
   \tupItem{\\der}{puntero(nodo)}
  \end{Tupla}
  
  \indent \indent donde log es enum\{AND, OR, NOT, CAR\}
 \end{Estructura}

Las restricciones estan representadas por arboles binarios, donde cada nodo indica si se trata de una operacion logica o de un tag. Las caracteristicas son hojas del arbol, mientras que los operadores logicos son nodos internos. Para ver si se verifica cierta restriccion, el arbol se evalua de forma recursiva, para respetar el orden de los operadores logicos.

\pagebreak 
 
\subsection{Invariante de Representacion}
\subsubsection{El invariante en lenguaje natural}

\begin{enumerate}
  \item El arbol binario no posee ciclos.
  \item Todo nodo que representa los operadores logicos AND y OR tiene ambos hijos definidos.
  \item Todo nodo que representa el operador logico NOT tiene el hijo derecho definido.
  \item Todo nodo que representa un tag es una hoja.

\end{enumerate}

\subsubsection{El invariante en lenguaje formal}

 \Rep[rtr][r]{
  \begin{enumerate}
  \item ($\forall p$: puntero(nodo)) (p $\in$ punteros(r.raiz)) $\implies$ \#(p, punteros(r.raiz)) \igobs 1)
  \item ($\forall p$: puntero(nodo)) (p $\in$ punteros(r.raiz) $\land$ (p.tipo = AND $\lor$ p.tipo = OR) $\implies$ \\ p.izq $\neq$ NULL $\land$ p.der $\neq$ NULL
  \item ($\forall p$: puntero(nodo)) (p $\in$ punteros(r.raiz) $\land$ p.tipo = NOT) $\implies$ p.der $\neq$ NULL
  \item ($\forall p$: puntero(nodo)) (p $\in$ punteros(r.raiz) $\land$ p.tipo = CAR) $\implies$ \\ p.izq = NULL $\land$ p.der = NULL
  \end{enumerate} 
  }

~

\tadOperacion{punteros}{puntero(nodo)/p}{multiconj(punteros(nodo))}{}
 \tadAxioma{punteros(p)}{
    \LIF{ p = NULL} \LTHEN{ $\emptyset$} \LELSE{ Ag(r, punteros(r.izq)) $\bigcup$ punteros(r.der)} \LFI
  }
  
\subsection{Funcion de abstraccion}

\Abs[rtr]{restriccion}[r]{rest}{($\forall$ ts : conj(tag) verifica?(ts, rest) $=$ Verifica?(ts, r)}

~

\tadOperacion{Verifica?}{conj(tag)/ts, puntero(nodo)/p}{bool}{}
 \tadAxioma{Verifica?(ts,p)}{
\LIF{ p.tipo = CAR } \LTHEN{ p.tag $\in$ ts} \LELSE{ \\
\LIF{ p.tipo = AND } \LTHEN{ Verifica?(ts, p.izq) $\land$ Verifica?(ts, p.der)} \LELSE{ \\
\LIF{ p.tipo = OR } \LTHEN{ Verifica?(ts, p.izq) $\lor$ Verifica?(ts, p.der)} \LELSE{ \\
 $\neg$Verifica?(ts, p.der) }}}\LFI
  }

\pagebreak

\subsection{Algoritmos}

\TipoFuncion{iNuevoTag}{\In{t}{tag}}{restriccion} \complejidad{Complejidad: }{1}\\
 \indent\indent puntero(nodo) n $\leftarrow$ new nodo \complejidad{}{1}\\
 \indent\indent n.tag $\leftarrow$ t \complejidad{}{1}\\
 \indent\indent n.tipo $\leftarrow$ CAR \complejidad{}{1}\\
 \indent\indent n.izq $\leftarrow$ NULL \complejidad{}{1}\\
 \indent\indent n.der $\leftarrow$ NULL \complejidad{}{1}\\
 \indent\indent res.raiz $\leftarrow$ n \complejidad{}{1}

 ~

\TipoFuncion{iNot}{\Inout{r}{restriccion}}{} \complejidad{Complejidad: }{1}\\
 \indent\indent puntero(nodo) n $\leftarrow$ new nodo \complejidad{}{1}\\
 \indent\indent n.tipo $\leftarrow$ NOT \complejidad{}{1}\\
 \indent\indent n.der $\leftarrow$ r.raiz \complejidad{}{1}\\
 \indent\indent res.raiz $\leftarrow$ n \complejidad{}{1}

 ~

\TipoFuncion{iAnd}{\In{r1}{restriccion}, \In{r2}{restriccion}}{restriccion} \complejidad{Complejidad: }{1}\\
 \indent\indent puntero(nodo) n $\leftarrow$ new nodo \complejidad{}{1}\\
 \indent\indent n.tipo $\leftarrow$ AND \complejidad{}{1}\\
 \indent\indent n.izq $\leftarrow$ r1.raiz \complejidad{}{1}\\
 \indent\indent n.der $\leftarrow$ r2.raiz \complejidad{}{1}\\
 \indent\indent res.raiz $\leftarrow$ n \complejidad{}{1}

 ~

\TipoFuncion{iOr}{\In{r1}{restriccion}, \In{r2}{restriccion}}{restriccion} \complejidad{Complejidad: }{1}\\
 \indent\indent puntero(nodo) n $\leftarrow$ new nodo \complejidad{}{1}\\
 \indent\indent n.tipo $\leftarrow$ OR \complejidad{}{1}\\
 \indent\indent n.izq $\leftarrow$ r1.raiz \complejidad{}{1}\\
 \indent\indent n.der $\leftarrow$ r2.raiz \complejidad{}{1}\\
 \indent\indent res.raiz $\leftarrow$ n \complejidad{}{1}

 ~

\TipoFuncion{iVerifica? }{\In{r}{restriccion}, \In{ts}{lista(tag)}}{bool}  \complejidad{Complejidad: }{R}\\
\indent\indent res $\leftarrow$ verificaAux(r.raiz, ts)  \complejidad{}{R}

 ~

\TipoFuncion{iverificaAux}{\In{n}{nodo}, \In{ts}{lista(tag)}}{bool} \complejidad{Complejidad: }{R * long(ts)}\\
\indent\indent bool aux \complejidad{}{1}\\
\indent\indent if n.tipo == CAR \complejidad{}{1}\\
\indent\indent \indent aux $\leftarrow$ esta?(n.tag, ts) \complejidad{}{1}\\
\indent\indent else if n.tipo == AND \complejidad{}{1}\\
\indent\indent \indent aux $\leftarrow$ verificaAux(n.izq, ts) $\land$ verificaAux(n.der, ts) \complejidad{}{R}\\ 
\indent\indent else if n.tipo == OR \complejidad{}{1}\\
\indent\indent \indent else aux $\leftarrow$ verificaAux(n.izq, ts) $\lor$ verificaAux(n.der, ts) \complejidad{}{R}\\
\indent\indent else //NOT \complejidad{}{1}\\
\indent\indent \indent else aux $\leftarrow$ $\lnot$verificaAux(n.der, ts) \complejidad{}{R}\\
\indent\indent res $\leftarrow$ aux \complejidad{}{1}\\

Recordar que los tags tienen a lo sumo 64 caracteres. esta? toma $\bigO$(k) (siendo k la cantidad de elementos de la lista), pero como en este caso la cantidad de tags posibles esta acotada, toma $\bigO$(1).\\
\\
Debido a esto la ecuacion de complejidad ($T(n)$) es de la forma:\\

$\theta(1$) cuando $n = 1$\\
$2*T(n/2) + f(n)$  cuando $n > 1$\\
\\
a y c son 2 porque realizo dos llamadas recursivas (uno al subarbol derecho y otra al izquierdo) y cada uno de estos tiene la mitad del tama�o del original.
f(n) es $\bigO(1)$. Aplicando teorema maestro, vemos que cae en el primer caso:\\
\\
$\exists \epsilon = 1 / f(n) \in \bigO(n^{log_2(2 - \epsilon)})$\\
$f(n) \in \bigO(n^{0})$\\
$f(n) \in \bigO(1)$\\
\\
Entonces $T(n) = \theta(n^{log_22}) = \theta(n)$\\

\subsection{Servicios Usados}

\begin{itemize}
	\item esta?(Lista Enlazada(tag), tag) en $\bigO(n)$ siendo n la longitud de la lista
\end{itemize}


\pagebreak
\section{Modulo Cola de Prioridad}

\begin{Interfaz}
  \textbf{se explica con}: \tadNombre{Cola de prioridad(itLista(robot))}, \tadNombre{itMod($\alpha$)}

  \textbf{g�neros}: \TipoVariable{colaPrio(itLista(robot))}, \TipoVariable{itCola($\alpha$)}

  \textbf{usa}: \TipoVariable{bool}, \TipoVariable{Lista Enlazada($\alpha$)}, \TipoVariable{puntero($\alpha$)}, \TipoVariable{nat}
	

  \InterfazFuncion{Vacia}{}{colaPrio(itLista(robot))}%
  {$res \igobs$ vacia}%
  [$\bigO(1)$]
  [genera una cola vacia.]
  
  \InterfazFuncion{Encolar}{\In{a}{itLista}, \Inout{c}{colaPrio(itLista(robot))}}{itCola($\alpha$)}%
  [$c \igobs c_0$]
  {$c \igobs$ encolar(a, $c_0$) $\land res \igobs CrearItMod(<>,a \bullet <>)$ )}%
  [$\bigO(log(n))$]
  [agrega a la cola el elemento $a$.]
  []
    
  \InterfazFuncion{Vacia?}{\In{c}{colaPrio(itLista(robot))}}{bool}%
  {$res \igobs$ vacia?(c)}%
  [$\bigO(1)$]
  [checkea si la cola esta vacia.]
  
  \InterfazFuncion{Desencolar}{\Inout{c}{colaPrio(itLista(robot))}}{itLista}%
  [$c \igobs c_0 \land \neg$ vacia?($c_0$)]%
  {$res$ \igobs proximo($c_0$) $\land$ c \igobs desencolar($c_0$)}
  [$\bigO(log(n))$]
  [elimina el proximo de la cola y retorna el elemento.] 
  
\titlex{Operaciones del iterador}

  \InterfazFuncion{Siguiente}{\In{it}{itCola(itLista(robot))}}{itLista(robot)}%  
  [haySiguiente?(it)]
  {$res \igobs$ siguiente($it$)}%
  [$\bigO(1)$]
  [devuelve el elemento siguiente a la posici�n del iterador.] 
  [res no es modificable.]  

  \InterfazFuncion{ElimSig}{\Inout{it}{itCola(itLista(robot))}}{}%  
  [haySiguiente?(it)]
  {$it \igobs$ Eliminar($it$)}%
  [$\bigO( log(n) )$]
  [Elimina el elemento siguiente de la cola que esta siendo iterada.] 
  []  
  
\end{Interfaz}

\pagebreak

\subsection{Representacion de la Cola}

\subsubsection{Representacion}
\titlex{Representacion de la cola}
\begin{Estructura}{colaPrio($\alpha$)}[c]
  \begin{Tupla}[c]
   \tupItem{raiz}{puntero(Nodo)}
   \tupItem{padreUlt}{puntero(Nodo)}
   \tupItem{cant}{nat}
  \end{Tupla}
  
  \begin{Tupla}[Nodo]
   \tupItem{elem}{itLista(robot)}   
   \tupItem{\\padre}{puntero(Nodo)}
   \tupItem{\\izq}{puntero(Nodo)}
   \tupItem{\\der}{puntero(Nodo)}
  \end{Tupla}

    \begin{Tupla}[robot]
      \tupItem{infr}{nat}%
      \tupItem{\\rur}{rur}%
      \tupItem{\\est}{estacion}%
      \tupItem{\\tags}{DiccTrie(bool)}%
      \tupItem{\\permisos}{DiccTrie(DiccTrie(bool))}%
      \tupItem{\\itEst}{it(colaPrio(itLista(robot))}%
    \end{Tupla}  

\end{Estructura}

\subsubsection{Invariante de Representacion}

Esta es una cola de prioridad implementada sobre un arbol binario. Elegimos esta estructura en ve de la comunmente usada representacion en arreglo por que nos ahorramos el costo lineal de redefinir el vector/arreglo cuando se llena. 

\subsubsection{El invariante en lenguaje natural}

\begin{enumerate}
  \item La prioridad de cada nodo es mayor que la de sus hijos
  \item[] es izquierdista
  \item[] esta balanceado (el subarbol derecho puede tener uno menos de altura)
  \item[] si un elemento es hijo de N, su puntero padre apunta a N
  \item[] cada subarbol tambien cumple lo anterior
  \item cant es la cantidad de nodos
  \item padreUlt apunta al ultimo nodo
  \item no hay nodos repetidos ni ciclos
\end{enumerate}

\subsubsection{El invariante en lenguaje formal}

  \Rep[estr][c]{
  \begin{enumerate}
  \item esHeap?(c.raiz) \yluego
  \item c.cant \igobs cantNodos(c.raiz) \yluego
  \item c.cant > 1 \impluego c.padreUlt \igobs buscarPadreUlt(c.raiz)
  \item ($\forall p$: puntero(nodo)) (p $\in$ punteros(r.raiz)) $\implies$ \#(p, punteros(r.raiz)) \igobs 1)
  \end{enumerate} 
  }
  

  \tadOperacion{esHeap?}{puntero(Nodo)/n}{bool}{}
  \tadAxioma{esHeap?(n)}{
  
  \LIF{ n = NULL } \LTHEN{ true } \LELSE{ \\
    $\neg$ (n\DRef der \igobs NULL) \impluego $\neg$ (n\DRef izq \igobs NULL) \yluego\\
    \LIF{ n\DRef izq \igobs NULL }\LTHEN{ true }\LELSE{\\
    masPrio?(n\DRef elem, n\DRef izq\DRef elem) $\land$ n\DRef izq\DRef padre \igobs n \yluego \\
    \LIF{ n\DRef der \igobs NULL }\LTHEN{ true }\LELSE{\\
    masPrio?(n\DRef elem, n\DRef der\DRef elem) $\land$ n\DRef der\DRef padre \igobs n
    }\\ \LFI 
    }\\ \LFI \yluego \\
    altura(n\DRef izq) - altura(n\DRef der) <= 1  \yluego \\
    esHeap?(n\DRef izq) $\land$ esHeap?(n\DRef der)  
    } \\ \LFI\\
  }

\tadOperacion{masPrio?}{itLista/r1, itLista/r2}{bool}{}
\tadAxioma{masPrio?(r1, r2)}{\LIF{ Siguiente(r1)\DRef infr \igobs Siguiente(r2)\DRef infr \\}\LTHEN{
 Siguiente(r1)\DRef rur >Siguiente(r2)\DRef rur \\}\LELSE{ Siguiente(r1)\DRef infr > Siguiente(r2)\DRef infr \\} \LFI}


  \tadOperacion{cantNodos}{puntero(Nodo)/n}{nat}{}
  \tadAxioma{cantNodos(n)}{
  
  \LIF{ n = NULL } \LTHEN{ 0 } \LELSE{ \\
    1 + cantNodos(n\DRef izq) + cantNodos(n\DRef der)
    } \\ \LFI\\
  }
  
  \tadOperacion{altura}{puntero(Nodo)/n}{nat}{}
  \tadAxioma{altura(n)}{
  
  \LIF{ n = NULL } \LTHEN{ 0 } \LELSE{ \\
    1 + max(altura(n\DRef izq), altura(n\DRef der))
    } \\ \LFI\\
  }
  
  \tadOperacion{buscarPadreUlt}{puntero(Nodo)/n}{puntero(nodo)}{cantNodos(n) > 1 $\land$ esHeap?(n)}
  \tadAxioma{buscarPadreUlt(n)}{
  
  \LIF{ n\DRef der \igobs NULL $\lor$ altura(n\DRef der) \igobs 1} \LTHEN{ n } \LELSE{ \\
    \LIF{ altura(n\DRef der) + 1 \igobs altura(n\DRef izq)} \LTHEN{\\
     buscarUlt(n\DRef izq) } 
    \LELSE{buscarUlt(n\DRef der) } \LFI \\ 
    } \\ \LFI
  }
  
  \tadOperacion{punteros}{puntero(Nodo)/r}{multiconj(punteros(nodo))}{}
 \tadAxioma{punteros(d)}{
    \LIF{ r = NULL} \LTHEN{ $\emptyset$} \LELSE{ Ag(r, punteros(r.izq)) $\bigcup$ punteros(r.der)} \LFI
  }
  

\subsection{Funcion de abstraccion de la Cola}

\tadOperacion{Abs}{estr/c}{colaPrior($\alpha$)}{Rep(c)}
  \tadAxioma{Abs(c)}{\LIF{ c.cant = 0 }\LTHEN{ vacia }\LELSE{\\
   encoladoRecursivo(colaSecu(c.raiz)) }\\ \LFI}

\tadOperacion{encoladoRecursivo}{secu($\alpha$)/n}{colaPrior($\alpha$)}{}
\tadAxioma{encoladoRecursivo(n)}{\LIF{ vacia?(n)}\LTHEN{ vacia }\LELSE{\\ 
encolar(prim(n),encoladoRecursivo(fin(n)))
}\LIF}

\tadOperacion{colaSecu}{puntero(Nodo)/n}{secu($\alpha$)/n}{}
\tadAxioma{colaSecu(c)}{\LIF{ n \igobs NULL } \LTHEN{$ < > $} 
\LELSE{\\n\DRef elem \argumento colaSecu(n\DRef izq) \& colaSecu(n\DRef der)}\\ \LFI}

\pagebreak

\subsection{Representaci�n del Iterador}

\subsubsection{Representaci�n}

Este iterador no es m�s que un puntero al elemento siguiente y una cola para permitir un eliminado rapido del elemento siguiente.
 
 
 \begin{Estructura}{itCola($\alpha$)}[itTupla]
  \begin{Tupla}[itTupla]
   \tupItem{cola}{puntero(colaPrio)}
   \tupItem{pSig}{puntero(Nodo)}   
  \end{Tupla}
 \end{Estructura}

\subsubsection{Invariante de Representaci�n}

 \Rep[itTupla][it]{ it.pSig $\in$ punteros(it.cola\DRef raiz) 
 }\mbox{} \\

\subsection{Funcion de abstraccion del Iterador}

  \Abs[itTupla]{itMod($\alpha$)}[it]{im}{Siguientes($im$) $= a \bullet <> \land$ \\ 
   Anteriores($im$) $= <>$}\\

\pagebreak

\subsection{Algoritmos} 
\subsubsection*{Algoritmos de la Cola}
\TipoFuncion{iVacia}{}{colaPrio($\alpha$)} \complejidad{Complejidad:}{1} \\
\indent res.raiz $\leftarrow$ NULL\\
\indent res.cant $\leftarrow$ 0\\
\indent res.padreUlt $\leftarrow$ NULL\\
\\
\TipoFuncion{iVacia?}{\In{c}{colaPrio($\alpha$)}}{bool} \complejidad{Complejidad:}{1}\\
\indent res $\leftarrow$ c.cant $==$ 0 \complejidad{}{1}\\
\\
\TipoFuncion{iEncolar}{\Inout{c}{colaPrio($\alpha$)}, \In{a}{$\alpha$}, \In{n}{nat}}{}\complejidad{Complejidad:}{log(n)} \\
\indent c.cant++ \complejidad{}{1} \\
\indent var n $\leftarrow$ New Nodo \complejidad{}{1}\\
\indent n.elem $\leftarrow$ a \complejidad{}{1}\\
\indent if c.raiz = NULL \{ c.raiz $\leftarrow$ \& n\} \complejidad{Condicion:}{1}\\
\indent else \{ \\ 
\indent \indent if c.padreUlt $==$ NULL\{\complejidad{Condicion:}{1}\\
\indent \indent c.padreUlt $\leftarrow$ c.raiz \complejidad{}{1}\\ 
\indent \indent n.padre $\leftarrow$ c.raiz \complejidad{}{1}\\
\indent \indent c.raiz$\rightarrow$izq $\leftarrow$ \& n  \complejidad{}{1} \\
\indent \indent \} else \{\\
\indent \indent \indent if c.padreUlt$\rightarrow$der $==$ NULL\{ \complejidad{Condicion:}{1}\\
\indent \indent \indent \indent n.padre $\leftarrow$ c.padreUlt \complejidad{}{1}\\
\indent \indent \indent \indent c.padreUlt$\rightarrow$der $\leftarrow$ \& n \complejidad{}{1}\\
\indent \indent \indent \}else\{ \\
\indent \indent \indent \indent if $log_2(cant) == int(log_2(cant))$ \{\complejidad{Condicion:}{1}\\
\indent \indent \indent \indent \indent c.padreUlt $\leftarrow$ busqIzq(c) \complejidad{}{log(n)}\\
\indent \indent \indent \indent \indent n.padre $\leftarrow$ c.padreUlt\\
\indent \indent \indent \indent \indent c.padreUlt$\rightarrow$izq $\leftarrow$ \& n\\
\indent \indent \indent \indent \}else\{\\
\indent \indent \indent \indent \indent c.padreUlt $\leftarrow$ busqTransversal(c.padreUlt, c) \complejidad{}{log(n)}\\
\indent \indent \indent \indent \indent n.padre $\leftarrow$ c.padreUlt \complejidad{}{1}\\
\indent \indent \indent \indent \indent c.padreUlt$\rightarrow$izq $\leftarrow$ \& n \complejidad{}{1}\\
\indent \indent \indent \indent \}\\
\indent \indent \indent \}\\
\indent \indent \}\\
\indent \}\\
\indent subir(n)\\
\indent res $\leftarrow$ CrearIt(\& c, \& n) \complejidad{}{1}\\
\\
\textbf{Justificacion de complejidad:} Primero contemplamos el caso que la cola este vacia. Si no lo est� nos fijamos a donde apunta el puntero padreUlt, si esta NULL significa que solo existe la raiz en la cola, entonces agregamos el elemento a su izquierda. Una vez descartados estos casos agregamos el elemento en la izquierda de padreUlt si es posible, sino en su izquierda. De estar ocupado ambos lugares preguntamos si el arbol esta completo (el logaritmo de su cantidad de elementos debe ser entero). Todos estos pasos tienen complejidad constante\\
Si esta completo usamos BusqIzq que nos da el nodo m�s a la izquierda del arbol en complejidad logaritmica. si no usamos BusqTransversal(tiempo logaritmico) que nos da el "hermano" de padreUlt, es decir la siguiente posicion si fuera un heap sobre arreglo.\\
Una vez determinada la posicion del nuevo nodo utilizamos la funcion subir que reacomoda los nodos segun su prioridad en tiempo logaritmico.\\
Por lo tanto encolar termina siendo una suma de operaciones en tiempo constante o en tiempo logaritmico, sin ciclos.\\

\pagebreak

\TipoFuncion{iDesencolar}{\Inout{c}{colaPrio($\alpha$)}}{$\alpha$}\complejidad{}{log(n)}\\
\indent res $\leftarrow$ c.raiz$\rightarrow$elem\\
\indent var ult \complejidad{}{1}\\
\indent if c.padreUlt$\rightarrow$der != NULL \{ \complejidad{Condicion:}{1}\\
\indent \indent ult $\leftarrow$ c.padreUlt$\rightarrow$der \complejidad{}{1}\\
\indent \indent c.padreUlt$\rightarrow$der $\leftarrow$ NULL \complejidad{}{1}\\
\indent \}else \{\\
\indent \indent ult $\leftarrow$ c.padreUlt$\rightarrow$izq \complejidad{}{1}\\
\indent \indent c.padreUlt$\rightarrow$izq $\leftarrow$ NULL \complejidad{}{1}\\
\indent \}\\
\indent ult$\rightarrow$padre $\leftarrow$ NULL \complejidad{}{1}\\
\indent ult$\rightarrow$izq $\leftarrow$ c.raiz$\rightarrow$izq \complejidad{}{1}\\
\indent ult$\rightarrow$der $\leftarrow$ c.raiz$\rightarrow$der \complejidad{}{1}\\
\indent ult$\rightarrow$der$\rightarrow$padre $\leftarrow$ ult \complejidad{}{1}\\
\indent ult$\rightarrow$izq$\rightarrow$padre $\leftarrow$ ult \complejidad{}{1}\\
\indent delete c.raiz \complejidad{}{1}\\
\indent c.raiz $\leftarrow$ ult \complejidad{}{1}\\ %tengo en c.raiz y (ult) el nodo a bajar 
\indent bajar(ult)\\
\indent c.cant--\\
\\
\textbf{Justificacion de complejidad:} Determino la posicion del ultimo nodo (gracias al puntero padreUlt) y lo intercambio con la raiz en tiempo constante. Elimino la anterior raiz y reacomodo los punteros del padre del ultimo y los hijos de la raiz Tambien en tiempo constante.
Luego utilizo la op�racion bajar sobre la nueva raiz en tiempo logaritmico.\\
\\
\TipoFuncion{subir}{\Inout{n}{Nodo}}{} \complejidad{Complejidad:}{log(n)}\\
\indent var parent$\leftarrow$ n$\rightarrow$padre\\
\indent while(parent!=NULL $\land$ mayorPrio?(n$\rightarrow$elem, parent$\rightarrow$elem)) \{\\
\indent \indent swapPadreHijo(parent, n)\\
\indent \indent parent $\leftarrow$ n$\rightarrow$padre\\
\indent \}\\
\\
\textbf{Justificacion de complejidad:} Todas las operaciones usadas son en tiempo constante. Subir toma un nodo y lo compara con su padre. De tener su padre menor prioridad, los intercambia. Esto se hace hasta que vemos un elemento padre con mayor prioridad o nos chocamos con la raiz. El peor caso seria subir hasta la raiz, intercambiamos nodos una cantidad igual a la altura del arbol. Al ser nuestro arbol balanceado la altura es logaritmica con respecto a la cantidad de elementos.\\
\\
\TipoFuncion{bajar}{\Inout{n}{puntero(Nodo)}, \Inout{c}{colaPrio}}{} \complejidad{Complejidad:}{log(n)}\\
\indent while(n$\rightarrow$izq != NULL $\land$ mayorPrio?(n$\rightarrow$elem, n$\rightarrow$izq$\rightarrow$elem) $\lor$\\
\indent n$\rightarrow$der != NULL $\land$ mayorPrio?(n$\rightarrow$elem, n$\rightarrow$der$\rightarrow$elem) $\lor$ mayorPrio?(n$\rightarrow$elem, n$\rightarrow$izq$\rightarrow$elem) )\{\\
\indent \indent if (n$\rightarrow$der != NULL $\land$ mayorPrio?(n$\rightarrow$der$\rightarrow$elem, n$\rightarrow$izq$\rightarrow$elem)\{\\
\indent \indent \indent if n $==$ c.raiz \{c.raiz $\leftarrow$ n.der\} \complejidad{}{1}\\ 
\indent \indent \indent swapPadreHijo(n, n$\rightarrow$der)\\
\indent \indent \}else \{\\
\indent \indent \indent if n $==$ c.raiz \{c.raiz $\leftarrow$ n$\rightarrow$izq\} \complejidad{}{1}\\ 
\indent \indent \indent swapPadreHijo(n, n$\rightarrow$izq)\\
\indent \indent \}\\ 
\indent \}\\
\\
\textbf{Justificacion de complejidad:} Todas las operaciones usadas son en tiempo constante. Bajar toma un nodo y lo compara con sus hijos. De tener algun hijo mayor prioridad, lo intercambia con el de mayor prioridad. Esto se hace hasta queambos hijos tienen menor prioridad o somos una hoja. El peor caso seria bajar desde la raiz hasta ser una hoja, intercambiamos nodos una cantidad igual a la altura del arbol. Al ser nuestro arbol balanceado la altura es logaritmica con respecto a la cantidad de elementos.\\

\pagebreak

\TipoFuncion{mayorPrio?}{\In{it1}{itLista}, \In{it2}{itLista}}{bool} \complejidad{Complejidad:}{1}\\
\indent var r1 $\leftarrow$ Siguiente(it1) \complejidad{}{1}\\
\indent var r2 $\leftarrow$ Siguiente(it2) \complejidad{}{1}\\
\indent if (r1.infr == r2.infr)\{\complejidad{Condicion:}{1}\\
\indent \indent res $\leftarrow$ r1.rur > r2.rur \complejidad{}{1}\\
\indent \} else\{\\
\indent \indent r1.infr > r2.infr \complejidad{}{1}\\
\indent \}\\
\indent return res \complejidad{}{1}\\
\\
\TipoFuncion{swapPadreHijo}{\Inout{parent}{puntero(Nodo)}, \Inout{hijo}{puntero(Nodo)}}{} \complejidad{}{1}\\
\indent var aux \\
\indent if (parent$\rightarrow$padre != NULL) \{\\
\indent \indent if parent$\rightarrow$padre$\rightarrow$der == parent\{\\
\indent \indent \indent parent$\rightarrow$padre$\rightarrow$der $\leftarrow$ hijo \complejidad{}{1}\\
\indent \indent \} else\{\\
\indent \indent \indent parent$\rightarrow$padre$\rightarrow$izq $\leftarrow$ hijo \complejidad{}{1}\\
\indent \indent \}\\
\indent \}\\
\indent hijo$\rightarrow$padre $\leftarrow$ parent$\rightarrow$padre \complejidad{}{1}\\
\indent parent$\rightarrow$padre $\leftarrow$ hijo \complejidad{}{1}\\
\indent if (hijo$\rightarrow$izq != NULL) \{\\
\indent \indent hijo$\rightarrow$izq$\rightarrow$padre $\leftarrow$ parent \complejidad{}{1}\\
\indent \}\\
\indent if (hijo$\rightarrow$der != NULL) \{\\
\indent \indent hijo$\rightarrow$der$\rightarrow$padre $\leftarrow$ parent \complejidad{}{1}\\
\indent \}\\
\\
\indent if (parent$\rightarrow$der == hijo)\{ \complejidad{Condicion:}{1}\\
\indent \indent aux $\leftarrow$ parent$\rightarrow$izq \complejidad{}{1}\\
\indent \indent aux$\rightarrow$padre $\leftarrow$ hijo \complejidad{}{1}\\
\indent \indent parent$\rightarrow$der $\leftarrow$ hijo.der \complejidad{}{1}\\
\indent \indent parent$\rightarrow$izq $\leftarrow$ hijo.izq \complejidad{}{1}\\
\indent \indent hijo.der $\leftarrow$ parent \complejidad{}{1}\\
\indent \indent hijo.izq $\leftarrow$ aux \complejidad{}{1}\\
\indent \} else \{\\
\indent \indent aux $\leftarrow$ parent$\rightarrow$der \complejidad{}{1}\\
\indent \indent aux$\rightarrow$padre $\leftarrow$ hijo \complejidad{}{1}\\
\indent \indent parent$\rightarrow$der $\leftarrow$ hijo.der \complejidad{}{1}\\
\indent \indent parent$\rightarrow$izq $\leftarrow$ hijo.izq \complejidad{}{1}\\
\indent \indent hijo.izq $\leftarrow$ parent \complejidad{}{1}\\
\indent \indent hijo.der $\leftarrow$ aux \complejidad{}{1}\\
\indent \}\\
\\
\TipoFuncion{ibusqIzq}{\In{c}{colaPrio($\alpha$)}}{Nodo} \complejidad{Complejidad:}{log(n)}\\
\indent var actual $\leftarrow$ c.raiz \complejidad{}{1}\\
\indent while(actual$\rightarrow$izq != NULL)\{ \complejidad{}{log(n)}\\
\indent \indent actual $\leftarrow$ actual$\rightarrow$izq \complejidad{}{1}\\
\indent \}\\
\indent res $\leftarrow$ actual \complejidad{}{1}\\

\pagebreak

\TipoFuncion{ibusqTransversal}{\In{c}{colaPrio($\alpha$)}, \In{n}{Nodo}}{Nodo} \complejidad{Complejidad:}{log(n)}\\
\indent var parent $\leftarrow$ n.padre \complejidad{}{1}\\
\indent var act $\leftarrow$ n \complejidad{}{1}\\
\indent while(parent!= NULL $\land$ parent$\rightarrow$der == act )\{ //subo Hasta Ser Raiz o hijo Izquierdo \complejidad{}{log(n)}\\
\indent \indent act $\leftarrow$ parent \complejidad{}{1}\\
\indent \indent parent $\leftarrow$ parent$\rightarrow$padre \complejidad{}{1}\\
\indent \}\\
\indent act $\leftarrow$ parent$\rightarrow$der \complejidad{}{1}\\
\indent while(act$\rightarrow$izq =! NULL)\{ \complejidad{}{log(n)}\\
\indent \indent act $\leftarrow$ act$\rightarrow$izq \complejidad{}{1}\\
\indent \}\\
\indent res $\leftarrow$ act \complejidad{}{1}\\
\\
\TipoFuncion{ibusqDer}{\In{c}{colaPrio($\alpha$)}}{Nodo}\\
\indent var actual $\leftarrow$ c.raiz \complejidad{}{1}\\
\indent while(actual$\rightarrow$der != NULL)\{ \complejidad{}{log(n)}\\
\indent \indent actual $\leftarrow$ actual$\rightarrow$der \complejidad{}{1}\\
\indent \}\\
\indent res $\leftarrow$ actual \complejidad{}{1}\\
\\
\textbf{Justificacion de complejidad:}  Esta funcion trabaja correctamente en tiempo logaritmico a la cantidad de elementos si el nodo en cuestion es el ultimo agregado y es hijo derecho de alguien cuando el arbol no est� completo. Subimos en el arbol buscando un nodo que sea hijo izquierdo de alguien. Esto me toma un tiempo $\bigO(log(n))$. Cambio al hijo derecho del padre del nodo.
Luego bajo por la izquierda hasta toparme con un nodo sin hijo izquierdo. Esto ultimo en caso peor toma tambien $\bigO(log(n))$. $\bigO(2log(n)) \in \bigO(log(n))$.\\

\pagebreak

\subsubsection*{Algoritmos del Iterador}

\TipoFuncion{CrearIt}{\In{c}{colaPrio($\alpha$)}, \In{p}{puntero(Nodo)}}{itCola($\alpha$)} \complejidad{Complejidad:}{1}\\
\indent res.pSig $\leftarrow$ p \complejidad{}{1}\\
\indent res.cola $\leftarrow$ \& c \complejidad{}{1}\\

\TipoFuncion{Siguiente}{\In{it}{itCola($\alpha$)}}{$\alpha$} \complejidad{Complejidad:}{1}\\
\indent res $\leftarrow$ pSig$\rightarrow$elem\\

\TipoFuncion{ElimSig}{\Inout{it}{itCola($\alpha$)}}{} \complejidad{}{log(n)}\\
\indent var ult \complejidad{}{1}\\
\indent var c $\leftarrow$ it\DRef cola \complejidad{}{1}\\
\indent if c\DRef raiz == it.pSig \{ iDesencolar(c) \} else \{ \complejidad{}{log(n)}\\
\indent \indent if c\DRef padreUlt\DRef der == NULL\{ \complejidad{}{1}\\
\indent \indent \indent ult$\leftarrow$ c\DRef padreUlt\DRef izq \complejidad{}{1}\\
\indent \indent \} else\{\\
\indent \indent \indent ult$\leftarrow$ c\DRef padreUlt\DRef der \complejidad{}{1}\\
\indent \indent \} \\
\indent \indent var deboRestaurar? $\leftarrow$ false \complejidad{}{1}\\
\indent \indent if $\neg$ esDescendiente(ult, it.pSig) \{ \complejidad{Condicion:}{log(n)}\\
\indent \indent \indent ult$\leftarrow$ buscarDer(it.pSig) \complejidad{}{log(n)}\\
\indent \indent \indent ult\DRef padre\DRef der$\leftarrow$NULL \complejidad{}{1}\\
\indent \indent \indent deboRestaurar? $\leftarrow$ true \complejidad{}{1}\\
\indent \indent \} \\
\indent \indent ult$\rightarrow$padre $\leftarrow$ it.pSig\DRef padre \complejidad{}{1}\\
\indent \indent ult$\rightarrow$izq $\leftarrow$ it.pSig$\rightarrow$izq \complejidad{}{1}\\
\indent \indent ult$\rightarrow$der $\leftarrow$ it.pSig$\rightarrow$der \complejidad{}{1}\\
\indent \indent ult$\rightarrow$der$\rightarrow$padre $\leftarrow$ ult \complejidad{}{1}\\
\indent \indent ult$\rightarrow$izq$\rightarrow$padre $\leftarrow$ ult \complejidad{}{1}\\
\indent \indent delete it.pSig \complejidad{}{1}\\
\indent \indent var colaUlt \complejidad{}{1}\\
\indent \indent if deboRestaurar? \{ colaUlt $\leftarrow$ buscarDer(ult)\} \complejidad{}{log(n)}\\
\indent if deboRestaurar? \{\\
\indent \indent if c\DRef padreUlt\DRef der==NULL \{ \complejidad{}{1}\\
\indent \indent \indent colaUlt\DRef der $\leftarrow$ c\DRef padreUlt\DRef izq \complejidad{}{1}\\
\indent \indent \indent c\DRef padreUlt\DRef izq $\leftarrow$  NULL \complejidad{}{1}\\
\indent \indent \} else\{\\
\indent \indent \indent colaUlt\DRef der $\leftarrow$ c\DRef padreUlt\DRef der  \complejidad{}{1}\\
\indent \indent \indent c\DRef padreUlt\DRef der $\leftarrow$ NULL \complejidad{}{1}\\
\indent \indent \}\\
\indent \indent subir(colaUlt\DRef der)\complejidad{}{log(n)}\\
\indent \indent bajar(ult) \complejidad{}{log(n)}\\
\\
\textbf{Justificacion de complejidad:} Si el iterador apunta a la raiz uso Desencolar en $\bigO(log(n))$, sino considero el "subheap" con el nodo en cuestion como raiz. Contemplo el caso del que el ultimo de la cola sea hijo del nodo (me cuesta $\bigO(log(n))$).Si lo es, en ult guardo el ultimo nodo. Si no lo es uso la funcion buscarDer($\bigO(log(n))$) y lo asigno a la variable ult. \\
Cambio de lugar mi nodo y el de la variable ult (asignar punteros en tiempo constante). Borro el nodo al que apunta el iterador.
Me quedo un "hueco" en la ultima posicion del subheap. Agarro el ultimo de la cola y lo coloco ahi en tiempo constante. Luego uso la operacion subir sobre el ultimo del subheap y aplico bajar en $\bigO(log(n))$ al nodo ult.\\
Estas son operaciones en tiempo constante y logaritmico en relacion a la cantidad de elementos de la cola. al no haber ciclos podemos afirmar que la suma de estas pertenece a $\bigO(log(n))$.\\

\pagebreak

\TipoFuncion{esDescendiente}{\In{n}{puntero(Nodo)}, \In{p}{puntero(Nodo)}}{bool} \complejidad{Complejidad}{log(n)}\\
\indent res $\leftarrow$ false \complejidad{}{1}\\
\indent while n\DRef padre != NULL \{ \complejidad{}{log(n)}\\
\indent \indent if p == n\{ res $\leftarrow$ true\} \complejidad{}{1}\\
\indent \indent n$\leftarrow$ n\DRef padre \complejidad{}{1}\\
 \indent \}\\

\subsection{Servicios Usados}

\TipoVariable{lista($\alpha$)}:
\begin{itemize}
	\item Siguiente en $\bigO(1)$
\end{itemize}


%\pagebreak
%\section{Modulo DiccArreglo(nat, significado)}

\begin{Interfaz}
  \textbf{par�metros formales}\hangindent=2\parindent\\
  \parbox{1.7cm}{\textbf{g�neros}} nat, significado\\
  
  \textbf{se explica con}: \tadNombre{dicc($clave, significado$)}, \tadNombre{Iterador Unidireccional($\alpha$)}.\\
  \indent\textbf{g�neros}: \TipoVariable{DiccArreglo(nat, significado)}, \TipoVariable{itDicc(nat)}.

  \titlex{Operaciones b�sicas del diccionario arreglo}
  
  \InterfazFuncion{Vacio}{}{DiccArreglo(nat, significado)}%
  {$res \igobs$ vacio()}%
  [$\Theta(1)$]
  [genera un nuevo diccionario arreglo.]
  
  \InterfazFuncion{Definir}{\Inout{DA}{DiccArreglo(nat, significado)}, \In{s}{significado}}{}%
  [$DA_0 \igobs DA$]
  {$DA \igobs$ definir($\#claves(DA_0) + e$, s, $DA_0$)}%
  [$\Theta(n)$]
  [Define una nueva clave con su significado. Donde e es la cantidad de elementos eliminados del diccionario 	historicamente\\
   n es la cantidad de elementos en el diccionario en el estado anterior a Definir.] 

  \InterfazFuncion{Def?}{\In{a}{nat}, \In{DA}{DiccArreglo(nat, significado)}}{bool}%
  {$res \igobs$ def?(a, DA)}%
  [$\Theta(1)$]
  [Devuelve $True$ si el la clave est� definida.]

  \InterfazFuncion{Obtener}{\In{a}{nat}, \In{DA}{DiccArreglo(nat, significado)}}{significado}%
  [Def?(a, DA)]  
  {$res \igobs$ obtener(a, DA)}%
  [$\Theta(1)$]
  [Devuelve el significado de la clave a.]
  
  \InterfazFuncion{Borrar}{\In{a}{nat}, \Inout{DA}{DiccArreglo(nat, significado)}}{}%
  [Def?(a, DA) $\land$ $DA_0 \igobs DA$]  
  {$DA \igobs$ borrar(a, $DA_0$)}%
  [$\Theta(1)$]
  [Borra el significado y su clave a.]
  
  \InterfazFuncion{Claves}{\Inout{DA}{DiccArreglo(nat, significado)}}{itDicc(nat)}%  
  {$res \igobs$ CrearItUni(toSecu(claves($DA$)))}%
  [$\Theta(1)$]
  [Devuelve el conjunto de claves del diccionario.]
  
\pagebreak  
  
  \titlex{Operaciones b�sicas del iterador}
  
  \InterfazFuncion{CrearIt}{\In{DA}{DiccArreglo(nat, significado)}}{itDicc(nat)}%  
  {$res$ $\leftarrow$ CrearItUni(toSecu(claves($DA$)))}%
  [$\Theta(1)$]
  [Crea un iterador unidireccional del conjunto de claves \\
  de forma tal que siguiente devuelva la siguiente clave del diccionario.]
    
  \InterfazFuncion{Avanzar}{\Inout{it}{itDicc(nat)}}{}%  
  [$it = it_0$ $\land$ hayMas?(it)]
  {$it \igobs$ avanzar($it_0$)}%
  [$\Theta(1)$]
  [Avanza a la posici�n siguiente del iterador.]  
  
  \InterfazFuncion{Actual}{\In{it}{itDicc(nat)}}{nat}%  
  [hayMas?(it)]
  {$res \igobs$ actual($it$)}%
  [$\Theta(1)$]
  [devuelve el elemento siguiente a la posici�n del iterador.] 
  [res es modificable si y solo si it es modificable.]  
  
  \InterfazFuncion{HayMas?}{\In{it}{itDicc(nat)}}{bool}%  
  {$res \igobs$ hayMas?($it$)}%
  [$\Theta(1)$]
  [devuelve $true$ si y solo si en el iterador quedan elementos para avanzar.]   
    
\end{Interfaz}

\subsection{Representacion}

\subsubsection{Representaci�n del diccionario arreglo}

El objetivo de este modulo es implementar un arreglo de elementos que son una tupla con un significado y un booleano.
La idea es que si el booleano vale True significa que el elemento est� borrado. 

 \begin{Estructura}{DiccArreglo(nat, significado)}[vec : vector(valor)]
  \begin{Tupla}[valor]
   \tupItem{sig}{significado}
   \tupItem{esta?}{bool}
  \end{Tupla}
 \end{Estructura}

  \Rep[vec][v]{true}\mbox{}\

  \AbsFc[vector(valor)]{dicc(nat, significado)}[vec]{\IF long($vec$) $=$ $0$ THEN vacio ELSE 
  \textbf{if} prim($vec$).esta? = $true$ \textbf{then}\\ definir(long($vec$) $-$ 1, prim(($vec$).significado, Abs(fin($vec$))) \\\textbf{else} \\
   Abs(fin($vec$)) \\\textbf{fi} FI} 

\pagebreak

\subsubsection{Representaci�n del iterador}
Este iterador recorre las claves que son los naturales en el rango del vector. Para esto mantiene una variable\\
actualizada con la posici�n siguiente. Este iterador se indefine cada vez que el vector se redefine. En ese caso deber� crearse un nuevo iterador.

 \begin{Estructura}{itDicc(nat)}[itTupla]
  \begin{Tupla}[itTupla]
   \tupItem{posicion}{nat}
   \tupItem{pVec}{puntero(vector(valor))}   
  \end{Tupla}
 \end{Estructura}
 
 Fijarse que en el rep no puedo indexar porque pVec en el mundo de TAD�s es una secuencia!.
 \Rep[itTupla][it]{$\neg$($it$.pVec  $=$  NULL) $\yluego$ \\ 
 ($it$.posicion $<$ long(*($i$.pVec)) \&\& 0 $\leq$ $it$.posicion) $\yluego$ \\ 
 buscar((*($it$.pVec)), $it$.posicion).esta? = true) 
 }\mbox{} \\

  \Abs[itTupla]{itUni(nat)}[it]{iu}{siguientes($iu$) $=$ convertir(long(*($it$.pVec))) $\land$ \\ 
   siguiente($iu$) $=$ $it$.posici�n}\\

 ~  

 \tadOperacion{convertir}{nat/n}{s : secu(nat)}{}
  \tadAxioma{convertir($n$)}{if $n = 0$ then \\ 
  $< >$ else \\
  convertir(n - 1)) o (n - 1) \\
  fi} 
  
 ~    
  
 \tadOperacion{buscar}{vector(valor)/$v$, nat/$n$}{va : valor}{n $\leq$ tam($v$)}
  \tadAxioma{buscar($v$,$n$)}{if $n = 0$ then \\ 
  prim(v) else \\
  buscar(fin(v), n-1) \\
  fi}

\pagebreak

\subsection{Algoritmos}

  \subsubsection{Algoritmos del diccionario arreglo}

  \TipoFuncion{iVacio}{}{vector(valor)} \\
  \indent\indent res $\leftarrow$ Vac�a();\\
  
  \TipoFuncion{iDefinir}{\Inout{vec}{vector(valor)}, \In{s}{significado}}{} \\
  \indent\indent\indent\indent res  $\leftarrow$ AgregarAtras(vec, s);\\
  
  \TipoFuncion{iDef?}{\In{a}{nat}, \Inout{vec}{vector(valor)}}{bool} \\
  \indent\indent res $\leftarrow$ false; \\ 
  \indent\indent if a $<$ Longitud(vec) \&\& a $>$ 0 \{ \\
  \indent\indent\indent\indent if vec[a].esta? true \{ \\ 
  \indent\indent\indent\indent\indent\indent res $\leftarrow$ true; \\
  \indent\indent\indent\indent \} \\ 
  \indent\indent\} \\

  \TipoFuncion{iObtener}{\In{a}{nat}, \Inout{vec}{vector(valor)}}{valor} \\
  \indent\indent res $\leftarrow$ vec[a].significado; \\

  \TipoFuncion{iBorrar}{\In{a}{nat}, \Inout{vec}{vector(valor)}}{} \\
  \indent\indent vec[a].esta? $\leftarrow$ false; \\ 
  
  \TipoFuncion{iClaves}{\Inout{vec}{vector(valor)}}{itDicc(nat)} \\
  \indent\indent res $\leftarrow$ iCrearIt(vec);
 
  \subsubsection{Algoritmos del iterador}
  
  \TipoFuncion{iCrearIt}{\In{vec}{vector(valor)}}{itTupla} \\
  \indent\indent res.posicion $\leftarrow$ 0; \\
  \indent\indent res.pVec $\leftarrow$ \&vec;\\
  
  \TipoFuncion{iAvanzar}{\Inout{it}{itTupla}}{} \\
  \indent\indent while iHaySiguiente(it) \&\& iObtener(i,*(it.pVec)).esta? != true \{ \\
  \indent\indent\indent\indent it.posicion++; \\
  \indent\indent \} \\
 
  \TipoFuncion{iActual}{\In{it}{itTupla}}{nat} \\
  \indent\indent res $\leftarrow$ it.posicion; \\

  \TipoFuncion{iHayMas?}{\In{it}{itTupla}}{bool} \\
  \indent\indent if it.posicion $<$ Longitud(*(it.pVec)) \{ \\
  \indent\indent\indent\indent res $\leftarrow$ true; \\
  \indent\indent \} else \{ \\
  \indent\indent\indent\indent res $\leftarrow$ false; \\
  \indent\indent \}
\pagebreak
\section{M�dulo DiccTrie($\sigma$)}

\begin{Interfaz}
  
  \textbf{par�metros formales}\hangindent=2\parindent\\
  \parbox{1.7cm}{\textbf{g�neros}} $\sigma$\\
  % notar que ya existe el tad diccionario, este es otro. por ahora lo dejo asi, pero lo mejor seria darle un renombre.
  \textbf{se explica con}: \tadNombre{Diccionario($\sigma$)}.

  \textbf{g�neros}: \TipoVariable{diccTrie($\sigma$)}.

  \titlex{Operaciones b�sicas de diccionario}

  \InterfazFuncion{Vac�o}{}{dicc($\sigma$)}
  {$res \igobs vacio$}
  [$\bigO(1)$]
  [genera un diccionario vac�o.]

  \InterfazFuncion{Definir}{\Inout{d}{dicc($\sigma$)}, \In{k}{string}, \In{s}{$\sigma$}}{}
  [$d \igobs d_0$]
  {$d = definir(d, k, s)$}
  [$\bigO(long(k))$]
  [define la clave $k$ con el significado s en el diccionario.]
  [la clave se define por copia, pero el significado se define por referencia.]

  \InterfazFuncion{Definido?}{\In{d}{dicc($\sigma$)}, \In{k}{string}}{bool}
  [long(k) $>$ 0]
  {$res \igobs def?(d,k)$}
  [$\bigO(long(k))$]
  [devuelve true si k esta definido en el diccionario.]

  \InterfazFuncion{Significado}{\In{d}{dicc($\sigma$)}, \In{k}{string}}{$\sigma$}
  [$def?(d,k)$]
  {$res \igobs obtener(k,d)$}
  [$\bigO(long(k))$]
  [devuelve la referencia al significado de la clave k en d.]

\end{Interfaz}

\subsection{Representacion}
  
  \titlex{Representaci�n del diccionario}

  \begin{Estructura}{dicc$(\sigma)$}[puntero(nodo)]
    \begin{Tupla}[nodo]
      \tupItem{caracteres}{arreglo[256] de puntero(nodo)}%
      \tupItem{significado}{puntero($\sigma$)}%
    \end{Tupla}
  \end{Estructura}

Representamos este diccionario sobre un trie, compuesto por nodos. Cada nodo puede redireccionar a otro nodo en O(1), dado que usamos un arreglo con punteros a otros nodos. Esto se debe a que la cantidad de caracteres posibles esta acotada por 256, por lo que un arreglo estatico alcanza para todas las redirecciones posibles. A su vez, este nodo tiene un puntero al significado, dado que no necesariamente debe apuntar una estructura primitiva.

\pagebreak

\subsection{Invariante de Representacion}
\subsubsection{El invariante en lenguaje natural}

\begin{enumerate}
  \item Como este diccionario esta implementado sobre un trie, debemos garantizar que no existan ciclos. Caso contrario se esta limitando innecesariamente el espacio de significados. Como el TAD mapa en ningun momento especifica que no puede existir una ciudad sin nombre, no debemos olvidar este caso, representado con la raiz del diccionario.
  \item A excepcion de la raiz, un nodo que no redirecciona a cualquier otro nodo necesariamente debe tener un significado. Caso contrario no tiene sentido que ese nodo exista.
\end{enumerate}

\subsubsection{El invariante en lenguaje formal}

  \Rep[estr][d]{
  \begin{enumerate}
  \item ($\forall p, q$: puntero(nodo)) (p $\in$ Ag(d, punteros(d)) $\land$ q $\in$ Ag(d, punteros(d)) - \{p\}) $\implies$ p $\neq$ q)
  \item ($\forall p$: puntero(nodo)) (p $\in$ punteros(d) $\land$ esHoja(p)) $\implies$ p \DRef significado $\neq$ NULL
  \end{enumerate} 
  }

  ~
  
  \tadOperacion{punteros}{dicc/d}{multiconj(punteros(nodo))}{}
  \tadAxioma{punteros(d)}{
    \LIF{ d = NULL} \LTHEN{ $\emptyset$} \LELSE{ punterosAux(d, 0)} \LFI
  }

  ~

  \tadOperacion{punterosAux}{dicc/d, nat/n}{multiconj(punteros(nodo))}{}
  \tadAxioma{punterosAux(d, n)}{
  
  \LIF{ n = long(d->caracteres) } \LTHEN{ $\emptyset$ } \LELSE{ \\
    \LIF{ d \DRef caracteres[n] = NULL} \LTHEN{ $\emptyset$ } 
    \LELSE{ Ag(d \DRef caracteres[n], punteros(d \DRef caracteres[n], 0) } \LFI \\ $\bigcup$ punteros(d, n+1) 
    } \\ \LFI
  }


  ~

  \tadOperacion{esHoja}{dicc/d}{bool}{}
  \tadAxioma{esHoja(d, n)}{ esHojaAux(d,0) }

  ~

  \tadOperacion{esHojaAux}{dicc/d, nat/n}{bool}{}
  \tadAxioma{esHojaAux(d, n)}{
  
  \LIF{ n $<$ 256 $\land$ d $\neq$ NULL} \\ \LTHEN{ d \DRef significado $\neq$ NULL $\land$ d \DRef caracteres[n] = NULL $\land$ esHojaAux(d, n+1)} 
  \LELSE{ true} \LFI
  }
  
  \subsection{Funcion de abstraccion}

  \Abs[estr]{dicc($\sigma$)}[d]{dic}{  
  ($\forall s$: string) def?(dic, s) = estaDefinido?(s, d) $\land$ \\ (def?(dic, s) \impluego ($\forall k$: string) obtener(k, dic) = dameSignificado(s, d))
  }
  
  ~

  \tadOperacion{dameSignificado}{string/s, puntero(nodo)/d}{puntero($\sigma$)}{}
  \tadAxioma{dameSignificado(s, d)}{
  
  \LIF{ d = NULL } \LTHEN{ NULL} \LELSE{ \\
    \LIF{ long(s) = 0} \LTHEN{ d\DRef significado }\\
    \LELSE{ dameSignificado(fin(s), d\DRef caracteres[ord(prim(s))-1])} \LFI
     } \\ \LFI
  }

  \tadOperacion{estaDefinido?}{string/s, puntero(nodo)/p}{bool}{}
  \tadAxioma{estaDefinido?(s, p)}{dameSignificado(s, p) $\neq$ NULL}

\newpage

\subsection{Algoritmos}

\TipoFuncion{iVacio}{}{dicc($\sigma$)} \complejidad{Complejidad:}{1} \\
\indent res $\leftarrow$ <caracteres: arreglo[256] de puntero(nodo), significado: puntero($\sigma$)> \complejidad{}{1}

~

\TipoFuncion{nuevoNodo}{}{nodo} \complejidad{Complejidad:}{1}\\
\indent res $\leftarrow$ <ad(puntero(nodo)), puntero($\sigma$)> \complejidad{}{1}

~

\TipoFuncion{iDefinir}{\Inout{d}{dicc($\sigma$)}, \In{k}{string}, \In{s}{$\sigma$}}{} \complejidad{Complejidad:}{|k|} \\
\indent var nodoActual $\leftarrow$ d \complejidad{}{1} \\
\indent var i $\leftarrow$ 0 \complejidad{}{1} \\
\indent while i $<$ long(k) \{ \complejidad{}{|k|} \\
\indent \indent if *nodoActual.campo$_1$[ord(k[i])] == NULL \complejidad{}{1} \\
\indent \indent \indent *nodoActual.campo$_1$[ord(k[i])] $\leftarrow$ nuevoNodo() \complejidad{}{1} \\
\indent \indent \} \\
\indent \indent nodoActual $\leftarrow$ *nodoActual.campo$_1$[ord(k[i])] \complejidad{}{1} \\
\indent \indent i++ \complejidad{}{1}\\
\indent \} \\
\indent *nodoActual.campo$_2$ $\leftarrow$ \&s \complejidad{}{1}

~

\TipoFuncion{iDefinido?}{\In{d}{dicc($\sigma$)}, \In{k}{string}}{bool} \complejidad{Complejidad:}{|k|}\\
\indent var nodoActual $\leftarrow$ d \complejidad{}{1}\\
\indent var i $\leftarrow$ 0 \complejidad{}{1}\\
\indent var seguirBuscando $\leftarrow$ true \complejidad{}{1}\\
\indent while i $<$ long(k) \&\& seguirBuscando \{ \complejidad{}{|k|}\\
\indent \indent if *nodoActual.campo$_1$[ord(k[i])] == NULL \{ \complejidad{}{1}\\
\indent \indent \indent seguirBuscando $\leftarrow$ false \complejidad{}{1}\\
\indent \indent \} else \{ \\
\indent \indent \indent nodoActual $\leftarrow$ *nodoActual.campo$_1$[ord(k[i])] \complejidad{}{1}\\
\indent \indent \indent i++ \complejidad{}{1}\\
\indent \indent \} \\
\indent \} \\
\indent res $\leftarrow$ false \complejidad{}{1}\\
\indent if i == long(k)  \{ \complejidad{}{1}\\
\indent \indent  res $\leftarrow$ (*nodoActual.campo$_2$[ord(k[i])] $\neq$ NULL) \complejidad{}{1}\\
\indent \} \\

~

\TipoFuncion{iSignificado}{\In{d}{dicc($\sigma$)}, \In{k}{string}}{$\sigma$} \complejidad{Complejidad:}{|k|}\\
\indent var nodoActual $\leftarrow$ d \complejidad{}{1}\\
\indent var i $\leftarrow$ 0 \complejidad{}{1}\\
\indent while i $<$ long(k) \{ \complejidad{}{|k|}\\
\indent \indent nodoActual $\leftarrow$ *nodoActual.campo$_1$[ord(k[i])] \complejidad{}{1}\\
\indent \indent i++ \complejidad{}{1}\\
\indent \} \\
\indent res $\leftarrow$ *nodoActual.campo$_2$ \complejidad{}{1}\\



\end{document}
