\section{Extensiones}

\subsection{Secuencia($\alpha$)}
\tadOtrasOperaciones
\tadOperacion{iesimo}{secu($\alpha$)/s, nat/i}{$\alpha$}{}
\tadAxiomas
\tadAxioma{iesimo(s,i)}{
\textbf{if} i = 0 \textbf{then} prim(s) \textbf{else} iesimo(fin(s),n-1) \textbf{fi}
}

\subsection{Cola($\alpha$)}
\tadOtrasOperaciones
\tadOperacion{in?}{$\alpha$/elem, cola($\alpha$)/c}{bool}{}
\tadAxiomas
\tadAxioma{in?(vacia)}{false}
\tadAxioma{in?(e, encolar(elem, c))}{\textbf{if} e = elem \textbf{then} true \textbf{else} in?(e, c) \textbf{fi}}

\section{M�dulo Ciudad Robotica}

\begin{Interfaz}

  \textbf{se explica con}: \tadNombre{CiudadRobotica}, \tadNombre{Iterador Bidireccional($\alpha$)}.

  \textbf{g�neros}: \TipoVariable{ciudad}, \TipoVariable{itLista($\alpha$)}.

  \textbf{usa}:  \TipoVariable{bool}, \TipoVariable{mapa}, \TipoVariable{rur}, \TipoVariable{tag}, \TipoVariable{diccTrie($\sigma$)} 
  
  \titlex{Operaciones b�sicas de ciudad}

  \InterfazFuncion{Crear}{\In{m}{mapa}}{ciudad}%
  {$res \igobs$ crear($m$)}%
  [$\bigO(1)$]
  [genera una nueva Ciudad.]
  [el mapa se pasa por referencia. la ciudad se devuelve por referencia.]

  \InterfazFuncion{Entrar}{\In{ts}{secu(tag)}, \In{e}{estacion}, \Inout{c}{ciudad}}{}
  [$c \igobs c_0 \land e \in$ estaciones($c$)]
  {$c \igobs$ entrar($ts, e, c_0$)}
  [$\bigO(|e_{m}|*S*R + N_{total})$]
  [esta funcion ingresa un robot a la ciudad. un conjunto de tags $ts$ bien formado es un diccTrie(bool), donde si el robot tiene un tag, el diccTrie devuelve true con la operacion definido?(ts, tag)]
  []

  \InterfazFuncion{Mover}{\In{u}{rur}, \In{e}{estacion}, \Inout{c}{ciudad}}{}
  [$c \igobs c_0 \land e \in$ estaciones($c$) $\land u \in $ robots($c$)]
  {$c \igobs$ mover($u, e, c_0$)}
  [$\bigO(|e| + |estacion(c,u)| + log N_{e} + log N_{estacion(c,u)})$]
  [mueve un robot entre dos estaciones.]
  []

  \InterfazFuncion{Inspeccion}{\In{e}{estacion}, \Inout{c}{ciudad}}{}
  [$c \igobs c_0 \land e \in$ estaciones($c$) $\land$ robots(e,c) $\neq$ $\emptyset$]
  {$c \igobs$ inspeccion($e, c_0$)}
  [$\bigO(|e| + log N_e)$]
  [realiza una inspeccion en una estacion. para determinar que robot se elimina en una inspeccion, se toma el que cometio mas infracciones, desempatando por su RUR.]
  []

\pagebreak

 \InterfazFuncion{ProximoRUR}{\In{c}{ciudad}}{rur}
  {$res \igobs$ proximoRUR($c$)}
  [$\bigO(1)$]
  []
  []

  \InterfazFuncion{Mapa}{\In{c}{ciudad}}{mapa}
  {$res \igobs$ mapa($c$)}
  [$\bigO(1)$]
  []
  [el mapa se devuelve por referencia.]

  \InterfazFuncion{Robots}{\In{c}{ciudad}}{itLista(robot)}
  {$res \igobs$ CrearIt(robots($c$))}
  [$\bigO(1)$]
  [devuelve un iterador al conjunto de rur.]
  []

  \InterfazFuncion{Estacion}{\In{u}{rur}, \In{c}{ciudad}}{estacion}
  [$u \in$  robots($c$)]
  {$res \igobs$ estacion($u, c$)}
  [$\bigO(1)$]
  []
  []
  
  \InterfazFuncion{Tags}{\In{u}{rur}, \In{c}{ciudad}}{secu(tag)}
  [$u \in$ robots($c$)]
  {$res \igobs$ tags($e, c_0$)}
  [$\bigO(1)$]
  []
  [El conjunto de tags se devuelve por referencia.]

  \InterfazFuncion{$\#$Infracciones}{\In{u}{rur}, \In{c}{ciudad}}{nat}
  [$u \in$ robots($c$)]
  {$c \igobs$ $\#$infracciones($u, c$)}
  [$\bigO(1)$]
  []
  []
  
  \InterfazFuncion{Estaciones}{\In{c}{ciudad}}{itConj(estaciones)}
  {$res \igobs$ CrearIt(estaciones($c$))} %CrearIt es la del modulo conjunto
  [$\bigO(1)$]
  []
  []

  \InterfazFuncion{tagMasInvolucrado}{\In{c}{ciudad}}{tag}
  [$c \igobs c_0 \land (\exists t \in tagsHistoricos(c)) \yluego \#infractoresPorTag(t,c) > 0$]
  {$tag \igobs$ tagMasInvolucrado(c)}
  [$\bigO(1)$]
  [tag mas involucrado en infracciones a lo largo de la historia de la ciudad.]
  []

  \InterfazFuncion{$\#$inspecciones}{\In{e}{estacion}, \In{c}{ciudad}}{nat}
  [$c \igobs c_0 \land e \in$ estaciones($c$)]
  {$res \igobs$ inspecciones($e, c_0$)}
  [$\bigO(|e|)$]
  []
  []
  
  \InterfazFuncion{TagsHistoricos}{\In{c}{ciudad}}{secu(tag)}  
  {$res \igobs$ tagsHistoricos($c$)}
  [$\bigO(res*ts*R)$ \\
  res es la longitud de la secuencia devuelta, ts el conjunto de todos los tags posibles existentes en los robots y
  R la cantidad total de robots en la ciudad.]
  [La secuencia se devuelve por copia.]
  []

  \InterfazFuncion{$\#$InfraccionesXtag}{\In{t}{tag}, \In{c}{ciudad}}{nat}
  [$t \in$ tagsHistoricos($c$)]  
  {$c \igobs$ $\#$infraccionesXtag($t, c$)}
  [$\bigO(1)$]
  []
  []

\end{Interfaz}

\pagebreak

\subsection{Representaci�n}

  \begin{Estructura}{ciudad}[estr]
    \begin{Tupla}[estr]
      \tupItem{robots}{lista(robot)}%
      \tupItem{\\robRUR}{arreglo\_dimensionable(itLista(robot))}%
      \tupItem{\\robEstacion}{DiccTrie(colaPrio(itLista(robot)))}%
      \tupItem{\\mapa}{mapa}%
      \tupItem{\\proximoRUR}{nat}%
      \tupItem{\\tagMasInvolucrado}{tag}%
      \tupItem{\\tagsInvolucrados}{DiccTrie(tag)}%
      \tupItem{\\historialInspeccion}{DiccTrie(nat)}%
    \end{Tupla}
   
    \begin{Tupla}[robot]
      \tupItem{infr}{nat}%
      \tupItem{\\rur}{rur}%
      \tupItem{\\activo?}{bool}
      \tupItem{\\est}{estacion}%
      \tupItem{\\tags}{lista(tag)}%
      \tupItem{\\permisos}{DiccTrie(DiccTrie(bool))}%
      \tupItem{\\itEst}{it(colaPrio(itLista(robot))}%
    \end{Tupla}  
   
  \end{Estructura}

Para representar a la ciudad y cumplir con los ordenes de complejidad requeridos, decidimos utilizar una tupla con diferentes estructuras.

En primer lugar, utilizamos una lista de robots para luego poder obtener el siguiente rur en $\bigO$(1). Es relevante notar que todas las estructuras que de alguna manera apuntan a un robot, utilizan un iterador a esta lista para no estar manipulando la estructura interna de la misma de forma directa.

Por otro lado, con el objetivo de poder acceder a cualquier robot por medio de su RUR en $\bigO$(1), utilizamos un arreglo dimensionable. Dado que los RUR son consecutivos y comienzan en 0, utilizamos el indice del arreglo para identificar un RUR.

robEstacion permite buscar el conjunto de robots en una estacion. La busqueda se hace sobre un trie, y luego el conjunto de robots se representa con un heap. Esto se debe a que para que cierre la complejidad de inspeccion, tomar el robot mas infractor debe tener orden logaritmico.

\pagebreak

\subsection{Invariante de Representacion}
\subsubsection{El invariante en lenguaje natural}

\begin{enumerate}
  \item proximoRUR es igual a la longitud de la lista de robots mas uno.
  \item En robRUR, el indice del arreglo coincide con el RUR del robot.
  \item La estacion de los robots activos coincide con la estacion de robEstacion.
  \item No existe robot que pertenezca a la cola de alguna estacion que este inactivo o no pertenezca a la lista de robots.
  \item Un robot no puede estar en mas de una estacion a la vez.
  \item Si hay algun robot inactivo, existe algun significado de historialInspeccion positivo.
  \item Si alguna vez algun robot ha cometido una infraccion en la historia de la ciudad, el tagMasInvolucrado corresponde al significado maximo de tagsInvolucrados.
\end{enumerate}
  
\subsubsection{El invariante en lenguaje formal}

  \Rep[estr][c]{
  \begin{enumerate}
  \item proximoRUR = long(c.robots) + 1
  \item ($\forall$ i: nat) i < longitud(c.robots) \impluego c.robRUR[i].rur = i
  \item ($\forall$ r: rur) esta?(r,c.robots) \impluego ($\exists$ e: estacion) in?(r,obtener(e, c.robEstacion))
  \item ($\forall$ r: rur) $\neg$($\exists$ e: estacion) \\ esta?(r,obtener(e, c.robEstacion)) $\land$ (r.activo? = false $\lor$ $\neg$esta?(r,c.robots))
  \item ($\forall$ $r$: rur) $\neg$($\exists$ $e_1, e_2$: estacion) $e_1 \neq e_2$ \\ $\implies$ in?(r,obtener($e_1$, c.robEstacion)) $\land$ in?(r,obtener($e_2$, c.robEstacion))
  \item ($\exists$ r: rur) (r.activo? = false $\land$ esta?(r, c.robots)) \yluego \\ ($\exists$ e: estacion) def?(e, c.historialInspeccion) \yluego obtener(e, historialInspeccion) $>$ 0
  \item ($\exists$ r: rur) (r.infr > 0  $\land$ esta?(r, c.robots)) \yluego  $\neg$($\exists$ t: tag) def?(t, c.tagsInvolucrados) \yluego \\ obtener(t, c.tagsInvolucrados) $>$ obtener(c.tagMasInvolucrado, c.tagsInvolucrados)
  \end{enumerate} 
  }

\subsection{Funcion de abstraccion}

  \Abs[estr]{ciudad}[c]{ciu}{
  proximoRUR(ciu) = c.proximoRUR $\land$\\
  mapa(ciu) = c.mapa $\land$\\
  robots(ciu) = claves(c \DRef robRUR) $\land$\\
  ($\forall r$: rur) r $\in$ robots(ciu) \impluego estacion(r,c) = robRUR[rur].estacion  $\land$\\
  ($\forall r$: rur) r $\in$ robots(ciu) \impluego tags(rur, ciu) = claves(robRUR[rur].tags)  $\land$\\
  ($\forall r$: rur) r $\in$ robots(ciu) \impluego \#infracciones(r, ciu) = estacion(r,c) = robRUR[rur].infr
  }

\pagebreak

\subsection{Algoritmos}

\TipoFuncion{iMover}{\Inout{c}{ciudad}, \In{e}{estacion}, \In{u}{rur}}{} \complejidad{Complejidad: }{log(n_{e1}) + |e| + log (n_e)}\\
\indent var robot $\leftarrow$ Siguiente(c.robRUR[u]) \complejidad{}{1} \\
\indent ElimSig(robot.itEst) \complejidad{}{log(n_{estacion(u, c)})} \\
\indent if ($\neg$Significado(robot.est, Significado(e, robot.permisos)))\{ \complejidad{Condicion:}{|e| + |estacion(u, c)|)}\\
\indent \indent int i $\leftarrow$ 0  \complejidad{}{1}\\
\indent \indent int inf $\leftarrow$ 0  \complejidad{}{1}\\
\indent \indent while i <  Longitud(robot.permisos)  \{ \complejidad{}{1}\\
\indent \indent \indent if definido?(iesimo(robot.permisos, i), c.tagsInvolucrados) \{  \complejidad{}{1}\\
\indent \indent \indent \indent inf $\leftarrow$ obtener(iesimo(robot.permisos, i). c.tagsInvolucrados)++  \complejidad{}{1}\\
\indent \indent \indent \} else \{ \\
\indent \indent \indent \indent definir(iesimo(robot.permisos, i), 1, c.tagsInvolucrados)  \complejidad{}{1}\\
\indent \indent \indent \indent inf $\leftarrow$ 1  \complejidad{}{1}\\
\indent \indent \indent \} \\
\indent \indent \indent if tagMasInvolucrado == '' $\lor$ obtener(c.tagMasInvolucrado. c.tagsInvolucrados) < inf \{  \complejidad{}{1}\\
\indent \indent \indent \indent tagMasInvolucrado $\leftarrow$ iesimo(robot.permisos, i)  \complejidad{}{1}\\
\indent \indent \indent \} \\
\indent \indent \} \\
\indent \indent robot.infr++ \complejidad{}{1}\\
\indent \}\\
\indent robot.itEst $\leftarrow$ encolar(c.robRUR[u], Significado(e, c.robEstacion)) \complejidad{}{|e|+ log(n_e)}\\
\indent robot.est $\leftarrow$ e\\

\textbf{Justificacion de complejidad:} Debido a que la cantidad de tags esta acotada, dado que tienen a lo sumo 64 caracteres, la complejidad es: 
$log(n_{estacion(u, c)}) + 2*|e| + |estacion(u, c)| + log(n_e)$.\\
Pero 2*|e| $\in \bigO(|e|)$. Entonces la complejidad de mover $\in \bigO(log(n_{estacion(u, c)}) + |e| + |estacion(u, c)| + log(n_e))$ .\\

\TipoFuncion{i$\#$infracciones}{\In{c}{ciudad}, \In{u}{rur}}{nat} \complejidad{Complejidad: }{1}\\
\indent res $\leftarrow$ Siguiente(c.robRUR[u]).infr \complejidad{}{1}\\

\TipoFuncion{i$\#$infraccionesXtag}{\In{c}{ciudad}, \In{t}{tag}}{nat} \complejidad{Complejidad: }{1}\\
\indent res $\leftarrow$ significado(c.tagsInvolucrados, t) \complejidad{}{1}\\

\TipoFuncion{iTagsHistoricos}{\In{c}{ciudad}}{secu(tag)} \complejidad{}{long(res)*long(tags(siguiente(itRobots)))*long(c.robots)}\\
\indent itLista(robot) itRobots $\leftarrow$ CrearIt(c.robots) \complejidad{}{1}\\
\indent res $\leftarrow$ vacia \complejidad{}{1} \\
\indent itLista(tag) itRes $\leftarrow$ CrearIt(res) \complejidad{}{1} \\
\indent while haySiguiente(itRobots) \{ \complejidad{}{long(c.robots)}\\
\indent \indent itLista(tag) itTags $\leftarrow$ CrearIt(tags(siguiente(itRobots))) \complejidad{}{1}\\
\indent \indent while haySiguiente(itTags) \{ \complejidad{}{long(tags(siguiente(itRobots)))}\\
\indent \indent \indent int i $\leftarrow$ 0 \complejidad{}{1}\\
\indent \indent \indent boolean esta $\leftarrow$ false \complejidad{}{1}\\
\indent \indent \indent while i < long(res) \&\& esta != true \{ \complejidad{}{long(res)}\\
\indent \indent \indent \indent if res[i] == siguiente(itTags) \{  \complejidad{}{1}\\
\indent \indent \indent \indent \indent esta $\leftarrow$ true  \complejidad{}{1}\\
\indent \indent \indent \indent \} \\
\indent \indent \indent \} \\
\indent \indent \indent if esta == false \{ \complejidad{}{1}\\
\indent \indent \indent \indent agregarComoSiguiente(itRes, siguiente(itTags)) \complejidad{}{1}\\
\indent \indent \indent \} \\
\indent \indent \indent avanzar(itTags)  \complejidad{}{1}\\
\indent \indent \}  \\
\indent \indent avanzar(itRobots) \complejidad{}{1} \\
\indent \} \\

\pagebreak

\TipoFuncion{iEstacion}{\In{c}{ciudad}, \In{u}{rur}}{estacion} \complejidad{Complejidad: }{1}\\
\indent res $\leftarrow$ Siguiente(c.robRUR[u]).est  \complejidad{}{1}\\

\TipoFuncion{iTags}{\In{c}{ciudad}, \In{u}{rur}}{lista(tag)} \complejidad{Complejidad: }{1}\\
\indent res $\leftarrow$ Siguiente(c.robRUR[u]).tags \complejidad{}{1}\\

\TipoFuncion{iEstaciones}{\In{c}{ciudad}}{itLista(estacion)} \complejidad{Complejidad: }{1}\\
\indent res $\leftarrow$ estaciones(c.mapa)\complejidad{}{1}\\

\TipoFuncion{iRobots}{\In{c}{ciudad}}{itDicc(rur)} \complejidad{Complejidad: }{1}\\
\indent res $\leftarrow$ CrearIt(c.robots)\complejidad{}{1}\\

\TipoFuncion{iProximoRUR}{\In{c}{ciudad}}{rur} \complejidad{Complejidad: }{1}\\
\indent res $\leftarrow$ c.proximoRUR\complejidad{}{1}\\

\TipoFuncion{iMapa}{\In{c}{ciudad}}{mapa} \complejidad{Complejidad: }{1}\\
\indent res $\leftarrow$ c.mapa \complejidad{}{1}\\

\TipoFuncion{iInspeccion}{\In{c}{ciudad}, \In{e}{estacion}}{} \complejidad{Complejidad: }{|e| + log(n_e)}\\
\indent r $\leftarrow$ Desencolar(Significado(e, c.robEstacion))\complejidad{}{|e| + log(n_e)} \\
\indent r.activo? $\leftarrow$ false \complejidad{}{1}\\
\indent if definido?(c.historialInspeccion, e) \{ \complejidad{}{|e|}\\
\indent \indent significado(c.historialInspeccion, e)++ \complejidad{}{|e|}\\
\indent \} else \{ \\
\indent \indent definir(c.historialInspeccion, e, 1) \complejidad{}{|e|}\\
\indent \} \\

\TipoFuncion{iTagMasInvolucrado}{\In{c}{ciudad}}{tag} \complejidad{Complejidad: }{1}\\
\indent res $\leftarrow$ c.tagMasInvolucrado \complejidad{}{1}\\


\TipoFuncion{i$\#$inspecciones}{\In{e}{estacion}, \In{c}{ciudad}}{nat} \complejidad{}{1}\\
\indent if def?(e, c.historialInspeccion) \{ \complejidad{}{1}\\
\indent \indent res $\leftarrow$ significado(c.historialInspeccion, e) \complejidad{}{1}\\
\indent \} else \{ \\
\indent \indent res $\leftarrow$ 0 \complejidad{}{1}\\
\indent \} \\

\pagebreak

\TipoFuncion{iEntrar}{\In{ts}{lista(tag)}, \In{c}{ciudad}, \In{u}{rur}}{} \complejidad{Complejidad:}{|e_{m}|*S*R + N_{total}}\\
\indent var im $\leftarrow$ CrearIt(c.mapa) \complejidad{}{1}\\
\indent var permisos $\leftarrow$ vacio() //DiccTrie(DiccTrie(bool)) \complejidad{}{1}\\
\indent while(HaySiguiente?(im)) \{ \complejidad{}{S}\\
\indent \indent var senda $\leftarrow$  Siguiente(im) \complejidad{}{1}\\
\indent \indent var b $\leftarrow$ verifica?(senda.restriccion, ts) \complejidad{}{R}\\
\indent \indent definir(permisos, senda.$e_{1}$, vacio) \complejidad{}{|e_m|}\\
\indent \indent definir(significado(permisos, senda.$e_{1}$), $e_{2}$, b) \complejidad{}{|e_m|}\\
\indent \indent //Definir permutacion del permiso. \\
\indent \indent definir(permisos, senda.$e_{2}$, vacio)\} \complejidad{}{|e_m|}\\
\indent \indent definir(significado(permisos, senda.$e_{2}$), $e_{1}$, b) \complejidad{}{|e_m|}\\
\indent \indent Avanzar(im) \complejidad{}{1}\\
\indent \}\\
\indent var rob $\leftarrow$ < 0, c.proximoRUR, true, e, ts, permisos, NULL > \complejidad{}{1}\\
\indent itRob $\leftarrow$ AgregarAdelante(c.robots, rob) \complejidad{}{1}\\
\indent var Nuevo $\leftarrow$ CrearArreglo(c.proximoRUR) \complejidad{}{1}\\
\indent var i $\leftarrow$ 0 \complejidad{}{1}\\
\indent while (i $<$ c.proximoRUR -1) \{ \complejidad{}{N_{total}}\\
\indent \indent Nuevo[i] $\leftarrow$ c.robRUR[i] \complejidad{}{1}\\ 
\indent \indent i++ \complejidad{}{1}\\
\indent \}\\
\indent c.robRUR[rob.rur] $\leftarrow$ itRob \complejidad{}{1}\\
\indent c.robRUR $\leftarrow$ Nuevo \complejidad{}{1}\\
\indent var it $\leftarrow$ encolar(Significado(c.robEstacion, e), itRob) \complejidad{}{1}\\
\indent Siguiente(itRob).itEst $\leftarrow$ it \complejidad{}{1}\\
\indent c.proximoRUR++  \complejidad{}{1}\\

\textbf{Justificacion de complejidad:} Al entrar un robot, en primer lugar debemos armar su 'matriz' de permisos. Eso se hace con un DiccTrie(DiccTrie(bool)). El objetivo de esto es que luego sea mas facil mover a un robot, para saber si comete una infraccion o no al pasar de una estacion a otra. Para ver si un robot puede pasar de la estacion 1 a la estacion 2, primero se busca en el primer diccionario la estacion 1, y luego la estacion 2 para tener como resultado un bool. Esto tiene costo S*(R + 4|$e_m$|) que esta en $\bigO$(|$e_m$| * S * R).
Luego debemos agregar el robot al comienzo de la lista de robots, que esta en $\bigO$(1), y a su vez hay que agregar el robot al arreglo dimensionable, que esta en $\bigO$($N_{total}$).

\pagebreak

\subsection{Servicios Usados}

\TipoVariable{Lista Enlazada($\alpha$)}:
\begin{itemize}
	\item Siguiente(itLista(robot)) en $\bigO(1)$
	\item AgregarAdelante(lista(robot), robot) en $\bigO(1)$
\end{itemize}

\TipoVariable{colaPrio(itLista(robot))}:
\begin{itemize}
	\item encolar(colaPrio c, itLista it) en $\bigO(log(n))$ donde n es la cantidad de elementos en la cola
	\item desencolar(colaPrio c) en $\bigO(log(n))$ donde n es la cantidad de elementos en la cola
	\item ElimSig(itCola) en $\bigO(log(n))$ donde n es la cantidad de elementos en la cola iterada
\end{itemize}

\TipoVariable{mapa}:
\begin{itemize}
	\item Estaciones(mapa) en $\bigO(1)$
	\item CrearIt(mapa) en $\bigO(1)$
	\item Siguiente(itMapa) en $\bigO(1)$
	\item HaySiguiente(itMapa) en $\bigO(1)$
	\item Avanzar(itMapa) en $\bigO(1)$
\end{itemize}

\TipoVariable{DiccTrie($\alpha$)}:
\begin{itemize}
	\item Definir(DiccTrie, string s, $\alpha$) en $\bigO(|s|)$ donde |s| es la longitud del string 
	\item Significado(DiccTrie, string) en $\bigO(|s|)$ donde |s| es la longitud del string
	\item Definido?(DiccTrie, string) en $\bigO(|s|)$ donde |s| es la longitud del string 
\end{itemize}

\TipoVariable{Restriccion}:
\begin{itemize}
	\item Verifica?(restriccion, ListaEnlazada(tags)) en $\bigO(R)$ siendo R la cantidad de nodos del arbol restriccion.
\end{itemize}

