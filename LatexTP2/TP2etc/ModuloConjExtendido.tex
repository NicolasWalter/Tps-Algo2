% Diseño del tipo T
\newpage

% Diseño del Tipo
\disDisenio{Conjunto Lineal Extendido($\alpha)$}
% La especificación

\disAspectosDeLaInterfaz

\disInterfaz
Se extiende la interfaz del conjunto lineal dada en el apunte de m\'odulos b\'asicos.

\disOperaciones{b\'asicas de conjunto}

\disDeclaraProc{Unir}{\paramInOut{$c_{1}$}{conj($\alpha$)}, \paramIn{$c_{2}$}{conj($\alpha$)}}{$c_{1} \igobs c_{0}$}{$c_{1} \igobs c_{0} \cup c_{2}$}{\Ode{n}\ donde\ n\ es\ el\ cardinal\ del\ conjunto.}{Devuelve la uni\'on entre 2 conjuntos.}

\disDeclaraFuncion{DameUno}{\paramIn{$c_{1}$}{conj($\alpha$)}}{res: \alpha}{$\¬$vacio(c)}{res \ $\igobs$ dameUno(c)}{\Ode{1}}{Devuelve un elemento del conjunto c.}

\disDeclaraProc{SinUno}{\paramInOut{c}{conj($\alpha$)}}{$\¬$vacio(c) $\land$ c $\igobs$ c_{0}}{res \ $\igobs$ sinUno(c_{0})}{\Ode{1}}{Devuelve el conjunto c con un elemento menos.}

\disAlgoritmos
%\hspace*{\disSubSubSecMargen}{Texto}
% HACK: SGA 28/05/2011. Para quitar el titulo Algorithm del caption \floatname{algorithm}{}
\floatname{algorithm}{}
% WARNING: SGA 27/05/2011. La opción [H] indica a LaTex que el algoritmo lo queremos AQUI!
% Ver 4.4.1 Placement of Algorithms de algorithms.pdf.
\begin{algorithm}\phantom{[H]}
\begin{algorithmic}[1]
\Function {\textsc{$i$Unir}}{\paramInOut{$c_{1}$}{conj($\alpha$)}, \paramIn {$c_{2}$}{conj($\alpha$)}}{}  \Comment{$\Ode{n}$}
  \State var $itConj(\alpha)$ it $\gets$ crearIt($c_{2}$) \Comment{$\Ode{1}$}
  \While{(haySiguiente(it))}                      \Comment{$\Ode{n}$}
    \If{($\¬$ pertenece?($c_{1}$, siguiente(it)))} \Comment{$\Ode{1}$}
      \State agregar($c_{1}$,siguiente(it))       \Comment{$\Ode{1}$}
      \State avanzar(it)                          \Comment{$\Ode{1}$}
    \EndIf
  \EndWhile     
\EndFunction
\end{algorithmic}
\end{algorithm}


\begin{algorithm}\phantom{[H]}
\begin{algorithmic}[1]
\Function {\textsc{$i$DameUno}}{\paramIn{c}{conj($\alpha$)}}{$\disFlecha$ res : $\alpha$}{} \Comment{$\Ode{1}$}
  \State res $\gets$ siguiente(crearIt(c))  \Comment{$\Ode{1}$}                        
\EndFunction
\end{algorithmic}
\end{algorithm}

\begin{algorithm}\phantom{[H]}
\begin{algorithmic}[1]
\Function {\textsc{$i$SinUno}}{\paramInOut{c}{conj($\alpha$)}}{}  \Comment{$\Ode{1}$}
  \State eliminarSiguiente(crearIt($c_{1}$))  \Comment{$\Ode{1}$}
\EndFunction
\end{algorithmic}
\end{algorithm}



