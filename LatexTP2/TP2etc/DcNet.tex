


\newpage
\disAlgoritmos
%\hspace*{\disSubSubSecMargen}{Texto}
% HACK: SGA 28/05/2011. Para quitar el titulo Algorithm del caption \floatname{algorithm}{}
\floatname{algorithm}{}
% WARNING: SGA 27/05/2011. La opci�n [H] indica a LaTex que el algoritmo lo queremos AQUI!
% Ver 4.4.1 Placement of Algorithms de algorithms.pdf.




\begin{algorithm}\phantom{[H]}
\begin{algorithmic}[1]
\Function {\textsc{inicar}}{\paramIn{r}{red}}{$\disFlecha$ res : estr}
  \State compu masEnvio = $<$ $>$ // falta completar
  \State diccTrie(compu,tupla$<$conjAvl, ColaHeap, diccAvl, nat) paquetes = vacio($)$ 
  \State diccAvl(compu,diccAvl(compu,compu)) ipSiguiente = vacio()
  \State tupla$<$compu,nat$>$ masEnviados = $<$masEnvio,0$>$
  \State iterado$r$ computador1 = crearIterador(mo$s$trarComputadoras(r))
  \While {haySiguiente(computador)}
    \State iterador computador2 = crearIterador(mostrarComputadoras(r)) 
    \State compu compu1 = siguiente(computador1)
    \State diccAvl(compu,compu) ipSiguiente2 = vacio()
    \While {haySiguiente(computador2)}
      \State compu compu2 = siguiente(computador2) 
      \State lista camino = dameUno(caminoMasCorto(r),compu1,compu2) // si caminoMasCorto es vacio, se indefine
      \If {!vacio(camino)} // nunca va a ser vacio, minimo 2 elementos 
        \State compu compu3 = $camino[1]$
        \State definir(ipSiguiente2,compu2,compu3)
      \EndIf
      \State avanzar(computador2)
    \EndWhile  
    \State definir(ipSiguiente,compu1,ipSiguiente2)
    \State avanzar(computador1)
  \EndWhile
  \State iterador computador3 = crearIterador(mostrarComputadoras(r))
  \While {haySiguiente(computador3)}
    \State conjAvl(paquete) paquetes2 = vacio()
    \State conjHeap(paquete) paquetes3 = vacio()
    \State nat paquetesEnviados = 0
    \State diccAvl caminosRecorridos = vacio()
    \State definir(paquetes, siguente(compu3), $<$paquetes2, paquetes3, caminosRecorridos, paquetesEnviados$>$)
    \State avanzar(compu3)
  \EndWhile
\EndFunction
\end{algorithmic}
\end{algorithm}




\begin{algorithm}\phantom{[H]}
\begin{algorithmic}[1]
\Function {\textsc{anadirPaquete}}{\paramInOut{dc}{estr}, \paramIn{p}{paquete}}
  \State puntero actual = &obtener(dc.paquetes,p.origen)
  \State agregar(actual$\rightarrow$xId, p, true) // true??
  \State encolar(actual$\rightarrow$xPrior,p)
  \State lista(compu) caminoRecorrido = vacio()
  \State agregarAtras(caminoRecorrido, p.origen)
  \State definir(actual$\rightarrow$caminos, caminoRecorrido)
\EndFunction
\end{algorithmic}
\end{algorithm}


\begin{algorithm}\phantom{[H]}
\begin{algorithmic}[1]
\Function {\textsc{verRed}}{\paramIn{dc}{estr}}{$\disFlecha$ res: red}
  \State res $\gets$ dc.red
\EndFunction
\end{algorithmic}
\end{algorithm}



\begin{algorithm}\phantom{[H]}
\begin{algorithmic}[1]
\Function {\textsc{recorrido}}{\paramIn{dc}{estr}, \paramIn{p}{paquete}}{$\disFlecha$ res: lista(compu)}
  \State bool noEncontrado = true
  \State iterador it = crearIterador(mostrarComputadoras(dc.red))
  \State puntero actual 
  \While {noEncontrado}
    \State actual = &obtener(dc.paquetes, siguiente(it))
    \If {definido?(actual$\rightarrow$xId, paquete)} // en el avl se busca por id...
      \State noEncontrado = false
      \State res $\gets$ obtener(actual$\rightarrow$caminos,paquete)
    \Else
      \State avanzar(it)
    \EndIf
  \EndWhile
\EndFunction
\end{algorithmic}
\end{algorithm}




\begin{algorithm}\phantom{[H]}
\begin{algorithmic}[1]
\Function {\textsc{enviados}}{\paramIn{dc}{estr}, \paramIn{c}{compu}}{$\disFlecha$ res: nat}
  \State puntero actual = &obtener(dc.paquetes,c)
  \State res $\gets$ actual$\rightarrow$enviados
\EndFunction
\end{algorithmic}
\end{algorithm}


\begin{algorithm}\phantom{[H]}
\begin{algorithmic}[1]
\Function {\textsc{paquetes}}{\paramIn{dc}{estr}, \paramIn{c}{compu}}{$\disFlecha$ res: conjAvl(paquete)}
  \State puntero actual = &obtener(dc.paquetes, c)
  \State res $\gets$ actual$\rightarrow$paquetes // se repiten nombres
\EndFunction
\end{algorithmic}
\end{algorithm}



\begin{algorithm}\phantom{[H]}
\begin{algorithmic}[1]
\Function {\textsc{avanzarSegundo}}{\paramInOut{dc}{estr}}
  \State iterador it = crearIterador(mostrarComputadoras(dc.red))
  \State lista(tupla$<$paquete,lista(compu$>$, compu) aEnviar = vacio()
  \While {haySiguiente(it)}
    \State puntero actual = &obtener(dc.paquetes,siguiente(it))
    \If {!vacio?(actual$\rightarrow$xId)}
      \State paquete p = desencolar(actual$\rightarrow$xPrior)
      \State lista(compu) l = obtener(actual$\rightarrow$caminos)
      \State borrar(actual$\rightarrow$xId, p)
      \State borrar(actual$\rightarrow$caminos, p)
      \State puntero aux = &obtener(siguientes, siguiente(it))
      \State compu comput = obtener(*aux, p.destino)
      \If {comput != p.destino} 
        \State agregarAtras(l, comput)
        \State agregarAtras(aEnviar, $<$p,l$>$, comput)
      \EndIf
      \State actual$\rightarrow$enviados = actual$\rightarrow$enviados + 1
      \If {$\pi_{2}$(dc.masEnviados) $<$ actual$\rightarrow$enviados}
        \State dc.masEnviados = $<$siguiente(it), actual$\rightarrow$enviados$>$
      \EndIf      
    \EndIf
  \EndWhile
  \State iterador itP = crearIterador(aEnviar)
  \While {haySiguiente(itP)}
    \State puntero actual2 = &obtener(dc.paquetes, $\pi_{3}$(siguiente(itP)))
    \State agregar(actual2$\rightarrow$xId, $\pi_{1}$(siguiente(itP)))
    \State encolar(actual2$\rightarrow$xPrior, $\pi_{1}$(siguiente(itP)))
    \State definir(actual2$\rightarrow$caminos, $\pi_{1}$(siguiente(itP)), $\pi_{2}$(siguiente(itP)))
    \State avanzar(itP)
  \EndWhile
\EndFunction
\end{algorithmic}
\end{algorithm}

